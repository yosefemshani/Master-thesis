\chapter{Results}
\label{sec:Results}
This chapter focuses on verifying the proton beam trajectory calculation with increasing complexity (see \autoref{sec:VerifyBeam}). First of all, a vacuum state with different magnetic field strengths is analyzed (see \autoref{sec:vacuum}). Secondly, a water phantom will be prepared, which will be irradiated with both our trajectory implementation and TOPAS (see \autoref{sec:waterphantom}). Same parameter settings are used for further irradiations of a bone phantom (see \autoref{sec:bonephantom}). Finally, given prostate patient CT will be used as a last candidate verification step (see \autoref{sec:prostate}).
~\\

After verifying the proton beam trajectory simulation, the prior discussed open source software matRad (see \autoref{sec:matRad}) is investigated (see \autoref{sec:matRadPlan}). A simple treatment plan for the given prostate patient is created (see \autoref{sec:treatment}). Furthermore, needed energy values and beam positions are exported and implemented in our proton beam simulation for verification purposes (see \autoref{sec:comparison}). Moreover, a magnetic field is introduced and the difference of bixel positions is visualized (see \autoref{sec:matradmagneticfield}). Finally, prior explained gradient descent algorithm (see \autoref{sec:gradientdescentmaterial}) is used for converging shifted beam positions towards shown initial conditions (see \autoref{sec:matradgradientdescent}).

\newpage
\section{Verification of proton stopping position}
\label{sec:VerifyBeam}
\subsection{Vacuum state}
\label{sec:vacuum}
Before verifying the energy loss calculation for different material complexities, the magnetic field influence is examined. For this investigation, a fixed energy value is used with four different magnetic field strengths and vice versa. The following figure \ref{fig:vacuumexample} visualizes an example proton beam trajectory calculated in MATLAB under influence of a magnetic field.
%\begin{figure}[h]
%    \begin{minipage}{0.5\textwidth}
%        \centering
%        \includegraphics[width=8cm]{../Screenshots/vacuum/VacuumB15E50MeV.png}
%        \label{fig:VacuumB15E50MeV}
%    \end{minipage}
%    \hfill
%    \begin{minipage}[c]{0.5\textwidth}
%        \centering
%        \includegraphics[width=8cm]{../Screenshots/vacuum/VacuumB3E100MeV.png}
%        \label{fig:VacuumB3E100MeV}
%    \end{minipage}
%    \caption{The left figure is ight figure.}
%    \label{fig:VacuumExample}
%\end{figure}
\begin{figure}[h!]
    \centering
    \includegraphics[height=9cm]{../Screenshots/vacuum/VacuumB15E50MeV.png}
    \caption{Example MATLAB proton trajectory simulation in a vacuum under influence of a magnetic field with E = 50 MeV and B = 1.5 T is displayed. Highlighted position (red dot) marks the diameter of the circular trajectory. Half of this value is needed.}
    \label{fig:vacuumexample}
\end{figure}
\newpage
The following table \ref{tab:compareMATLABanalytical} compares MATLAB with analytically calculated results of the prior discussed \autoref{eqn:radiusmagnet}.
~\\
\begin{table}[h!]
    \centering
    \caption{Comparison of resulting radii for MATLAB ($\symup{r_{MATLAB}}$) and analytical ($\symup{r_{Analytical}}$) proton beam trajectory calculations under influence of a magnetic field in a vacuum state. The first table has a fixed magnetic field strength with four different energy values, while the second table shows the results for a fixed energy value and four different magnetic field strengths. Differences between those radii are also listed.}
    \label{tab:compareMATLABanalytical}
    \begin{tabular}{c c c c }
        
        \multicolumn{4}{c}{\textbf{B = 1.5 T}} \\
        \hline
        E [MeV] & $\symup{r_{MATLAB}}$ [cm] & $\symup{r_{Analytical}}$ [cm] & Difference [\%] \\
        \hline
        50  & 65.76 & 65.53 & 0.35 \\
        
        100 & 89.63 & 89.35 & 0.31 \\
        
        150 & 106.03 & 105.71 & 0.30 \\
        
        200 & 118.46 & 118.14 & 0.27 \\
        \hline
    \end{tabular}
    
    \vspace{1cm}

    \begin{tabular}{c c c c}
        
        \multicolumn{4}{c}{\textbf{E = 100 MeV}} \\
        \hline
        B [T] & $\symup{r_{MATLAB}}$ [cm] & $\symup{r_{Analytical}}$ [cm] & Difference [\%] \\
        \hline
        0.5 & 268.89 & 268.04 & 0.32 \\
        
        1   & 134.45 & 134.02 & 0.32 \\
        
        2   & 67.23  & 67.01  & 0.33 \\
        
        3   & 44.82  & 44.67  & 0.34 \\
        \hline
    \end{tabular}
\end{table}

\newpage

\subsection{Water phantom}
\label{sec:waterphantom}
\subsubsection{B = 0 T}
Before activating a magnetic field, energy loss calculation and its' effect on proton range will be investigated. A simple candidate case for this investigation is water.

~\\As explained in section \ref{sec:TOPAS}, \gls{topas} will be used as a reference for given MATLAB results. Both simulations are intuitively carried out under the same conditions. These conditions include, for example, the inital energy, the starting position of our beam source and the same \gls{hulut}.

~\\The following figures visualize, on the one hand, the trajectories for both MATLAB and \gls{topas} (see figure \ref{fig:waterb0e200mev}), and on the other hand, the \gls{topas} depth dose curve with marked $\symup{R_{80}}$ position for \gls{csda} range comparison (see figure \ref{fig:waterb0e200mevddc}) for a candidate energy value.
\begin{figure}[h!]
    \centering
    \includegraphics[height=9cm]{../Screenshots/water/Water_B0_E200MeV.png}
    \caption{\gls{topas} and MATLAB simulation in water for B = 0 T and E = 200 MeV. The white line shows the trajectory calculated in MATLAB, while the heatmap dose distribution refers to the \gls{topas} proton beam trajectory.}
    \label{fig:waterb0e200mev}
\end{figure}

\begin{figure}[!h]
    \centering
    \includegraphics[height=9.5cm]{../Screenshots/water/Water_B0_E200MeV_ddc.png}
    \caption{\gls{topas} simulation in water for B = 0 T and E = 200 MeV. Resulting depth dose curve and marked $\symup{R_{80}}$ range can be seen.}
    \label{fig:waterb0e200mevddc}
\end{figure}
~\\
Results for MATLAB \gls{csda} ranges and \gls{topas} $\symup{R_{80}}$ ranges, with their differences, for four different energy values are listed in the following table \ref{tab:MATLABTOPASwaterB0}.
\begin{table}[h!]
    \centering
    \caption{Comparison of \gls{csda} ranges calculated analytically in MATLAB ($\symup{r_{ana}}$) and $\symup{R_{80}}$ ranges extracted from a \gls{mc} simulation depth dose curve in \gls{topas} ($\symup{r_{MC}}$) for a water phantom. An example of a depth dose curve constructed with a \gls{topas} simulation is shown in figure \ref{fig:waterb0e200mevddc}. Four different energy values are analyzed and the resulting differences between the two ranges can be seen.}
    \label{tab:MATLABTOPASwaterB0}
    \begin{tabular}{c c c c }
        
        \multicolumn{4}{c}{\textbf{B = 0 T}} \\
        \hline
        E [MeV] & $\symup{r_{ana}}$ [mm] & $\symup{r_{MC}}$ [mm] & Difference [\%] \\
        \hline
        50  & 22.19 & 21.26 & 4.37 \\
        
        100 & 76.92 & 76.28 & 0.84 \\
        
        150 & 157.21 & 156.87 & 0.22 \\
        
        200 & 258.63 & 258.64 & 0.0039 \\
        \hline
    \end{tabular}
\end{table}

\subsubsection{B > 0 T}
The following \autoref{fig:waterb0e200mevdistance} illustrates the resulting relative absorbed dose value for the same candidate energy value used in \autoref{fig:waterb0e200mevddc} with consideration of the distance.
\begin{figure}[!h]
    \centering
    \includegraphics[height=9.5cm]{../Screenshots/water/Water_B0_200MeV_distance.png}
    \caption{TOPAS simulation in water B = 0 T and E = 200 MeV with calculated distance. Prior calculated $\symup{R_{80}}$ range for the same energy value is marked with the corresponding relative absorbed dose value.}
    \label{fig:waterb0e200mevdistance}
\end{figure}
~\\Multiple relative absorbed dose values for four different energy values with the calculation of $\symup{R_{80}}$ are listed in table \autoref{tab:waterrelativedoser80}
\begin{table}[h!]
    \centering
    \caption{Relative dose values for a \gls{topas} simulation in water with B = 0 T and four different energy values calculated with prior determined $\symup{R_{80}}$ ranges and distances.}
    \label{tab:waterrelativedoser80}
    \begin{tabular}{c c}
        %\multicolumn{4}{c}{\textbf{B = 0 T}} \\
        \hline
        E [MeV] &  Rel. Dose [\%] \\
        \hline
        50  & 79.90 \\
        
        100 & 81.60 \\
        
        150 & 82.60 \\
        
        200 & 83.69 \\
        \hline
    \end{tabular}
\end{table}

~\\This relative dose value of 83.69 \% and its corresponding distance value for $\symup{B = 1.5 \, \, T}$ is analyzed and visualized in the following \autoref{fig:waterb15e200mevdistance}.
\begin{figure}[!h]
    \centering
    \includegraphics[height=9.5cm]{../Screenshots/water/Water_B15_200MeV_distance.png}
    \caption{TOPAS simulation in water B = 1.5 T and E = 200 MeV with calculated distance. Prior determined relative dose value and its corresponding distance is marked.}
    \label{fig:waterb15e200mevdistance}
\end{figure}
~\\
The distances for MATLAB and \gls{topas} with their differences are listed in table \autoref{tab:compareMATLABTOPASdistancewater}.
\begin{table}[h!]
    \centering
    \caption{Comparison of distances calculated analytically in MATLAB ($\symup{s_{ana}}$) and with a \gls{mc} simulation in \gls{topas} ($\symup{s_{MC}}$) for a water phantom. Different energy values and magnetic field strengths analogous to \autoref{tab:compareMATLABanalytical} are analyzed with their differences listed.}
    \label{tab:compareMATLABTOPASdistancewater}
    \begin{tabular}{c c c c }
        
        \multicolumn{4}{c}{\textbf{B = 1.5 T}} \\
        \hline
        E [MeV] & $\symup{s_{ana}}$ [mm] & $\symup{s_{MC}}$ [mm] & Difference [\%] \\
        \hline
        50  & 22.19 & 21.26 & 4.37 \\
        
        100 & 76.92 & 76.22 & 0.92 \\
        
        150 & 157.21 & 157.01 & 0.13 \\
        
        200 & 258.63 & 258.29 & 0.13 \\
        \hline
    \end{tabular}
    
    \vspace{1cm}

    \begin{tabular}{c c c c}
        
        \multicolumn{4}{c}{\textbf{E = 100 MeV}} \\
        \hline
        B [T] & $\symup{s_{ana}}$ [mm] & $\symup{s_{MC}}$ [mm] & Difference [\%] \\
        \hline
        0.5 & 76.92 & 76.26 & 0.87 \\
        
        1   & 76.92 & 76.27 & 0.85 \\
        
        2   & 76.92  & 76.24  & 0.89 \\
        
        3   & 76.92  & 76.32  & 0.79 \\
        \hline
    \end{tabular}
\end{table}
~\\

In addition, the following \autoref{tab:compareMATLABTOPAScoordinateswater} lists the coordinates for the final positions calculated in MATLAB and \gls{topas} for the analyzed water phantom.
%\begin{table}[h!]
%    \centering
%    \caption{Comparison of \(\vec{X}_{\symup{MATLAB}}\) and \(\vec{X}_{\symup{TOPAS}}\) for a water phantom with \(\vec{X} = (x \, \, \, y)^{\symup{T}}\). Different energy values and magnetic field strengths are analyzed with their differences listed.}
%    \label{tab:compareMATLABTOPAScoordinateswater}
%    \begin{tabular}{c c c c }
%        \multicolumn{4}{c}{\textbf{B = 1.5 T}} \\
%        \hline
%        E [MeV] & \(\vec{X}_{\symup{MATLAB}}\) [mm] & \(\vec{X}_{\symup{TOPAS}}\) [mm] & Difference [\%] \\
%        \hline
%        50  & 
%        \(\begin{pmatrix} 22.19 \\ 122.87 \end{pmatrix}\) & 
%        \(\begin{pmatrix} 21.26 \\ 122.89 \end{pmatrix}\) & 
%        \(\begin{pmatrix} 4.37 \\ 0.016 \end{pmatrix} \) \\
%        \vspace{0.005cm} \\
%        
%        100 & 
%        \(\begin{pmatrix} 76.83 \\ 125.79 \end{pmatrix}\) & 
%        \(\begin{pmatrix} 76.11 \\ 125.87 \end{pmatrix}\) & 
%        \(\begin{pmatrix} 0.94 \\ 0.064 \end{pmatrix} \) \\
%        \vspace{0.005cm} \\
%        
%        150 & 
%        \(\begin{pmatrix} 156.63 \\ 134.13 \end{pmatrix}\) & 
%        \(\begin{pmatrix} 156.38 \\ 133.97 \end{pmatrix}\) & 
%        \(\begin{pmatrix} 0.16 \\ 0.12 \end{pmatrix} \) \\
%        \vspace{0.005cm} \\
%        
%        200 & 
%        \(\begin{pmatrix} 256.58 \\ 150.62 \end{pmatrix}\) & 
%        \(\begin{pmatrix} 256.21 \\ 149.24 \end{pmatrix}\) & 
%        \(\begin{pmatrix} 0.14 \\ 0.92 \end{pmatrix} \) \\
%        \hline
%    \end{tabular}
%    
%    \vspace{1cm}
%    
%    \begin{tabular}{c c c c}
%        \multicolumn{4}{c}{\textbf{E = 100 MeV}} \\
%        \hline
%        B [T] & \(\vec{X}_{\symup{MATLAB}}\) [mm] & \(\vec{X}_{\symup{TOPAS}}\) [mm] & Difference [\%] \\
%        \hline
%        0.5 & 
%        \(\begin{pmatrix} 76.91 \\ 123.60 \end{pmatrix}\) & 
%        \(\begin{pmatrix} 76.24 \\ 123.64 \end{pmatrix}\) & 
%        \(\begin{pmatrix} 0.87 \\ 0.032 \end{pmatrix} \) \\
%        \vspace{0.005cm} \\
%        
%        1   & 
%        \(\begin{pmatrix} 76.88 \\ 124.70 \end{pmatrix}\) & 
%        \(\begin{pmatrix} 76.22 \\ 124.76 \end{pmatrix}\) & 
%        \(\begin{pmatrix} 0.86 \\ 0.048 \end{pmatrix} \) \\
%        \vspace{0.005cm} \\
%        
%        2   & 
%        \(\begin{pmatrix} 76.76 \\ 126.89 \end{pmatrix}\) & 
%        \(\begin{pmatrix} 76.04 \\ 126.99 \end{pmatrix}\) & 
%        \(\begin{pmatrix} 0.94 \\ 0.079 \end{pmatrix} \) \\
%        \vspace{0.005cm} \\
%        
%        3   & 
%        \(\begin{pmatrix} 76.55 \\ 129.09 \end{pmatrix}\) & 
%        \(\begin{pmatrix} 75.89 \\ 129.21 \end{pmatrix}\) & 
%        \(\begin{pmatrix} 0.86 \\ 0.093 \end{pmatrix} \) \\
%        \hline
%    \end{tabular}
%\end{table}
\begin{table}[h!]
    \centering
    \caption{Comparison of end positions calculated analytically in MATLAB (\(\vec{V}_{\symup{ana}}\)) and in a \gls{mc} simulation using \gls{topas} (\(\vec{V}_{\symup{MC}}\)) for a water phantom with \(\vec{V} = (x, \, y)\). Different energy values and magnetic field strengths are analyzed with their differences are calculated ($\symup{Difference} = \sqrt{(x_{\symup{MC}} - x_{\symup{ana}})^2 + (y_{\symup{MC}} - y_{\symup{ana}})^2}$) and listed.}
    \label{tab:compareMATLABTOPAScoordinateswater}
    \begin{tabular}{c c c c }
        \multicolumn{4}{c}{\textbf{B = 1.5 T}} \\
        \hline
        E [MeV] & \(\vec{V}_{\symup{ana}}\) [mm] & \(\vec{V}_{\symup{MC}}\) [mm] & Difference [mm] \\
        \hline
        50  & 
        \((22.19, 122.87)\) & 
        \((21.26, 122.89)\) & 
        0.93 \\
        
        100 & 
        \((76.83, 125.79)\) & 
        \((76.11, 125.87)\) & 
        0.90 \\
        
        150 & 
        \((156.63, 134.13)\) & 
        \((156.38, 133.97)\) & 
        0.30 \\
        
        200 & 
        \((256.58, 150.62)\) & 
        \((256.21, 149.24)\) & 
        1.38 \\
        \hline
    \end{tabular}
    
    \vspace{1cm}
    
    \begin{tabular}{c c c c}
        \multicolumn{4}{c}{\textbf{E = 100 MeV}} \\
        \hline
        B [T] & \(\vec{V}_{\symup{ana}}\) [mm] & \(\vec{V}_{\symup{MC}}\) [mm] & Difference [mm] \\
        \hline
        0.5 & 
        \((76.91, 123.60)\) & 
        \((76.24, 123.64)\) & 
        0.67 \\
        
        1   & 
        \((76.88, 124.70)\) & 
        \((76.22, 124.76)\) & 
        0.67 \\
        
        2   & 
        \((76.76, 126.89)\) & 
        \((76.04, 126.99)\) & 
        0.73 \\
        
        3   & 
        \((76.55, 129.09)\) & 
        \((75.89, 129.21)\) & 
        0.67 \\
        \hline
    \end{tabular}
\end{table}


\newpage
~\\

\subsection{Bone phantom}
\label{sec:bonephantom}
\subsubsection*{B = 0 T}
\label{sec:bonephantomb0t}
A next possible analysis step is to focus on the effect of the \gls{hulut} used and the proton interaction with a more complex material. For this purpose, bone material is examined. Bone structures can have different CT number values. For this work, a CT number of 1000 HU is assumed to represent bone material \cite{denotter_hounsfield_2024}.
~\\

Analogous to the calculations in water, the following \autoref{tab:MATLABTOPASboneB0} lists the results for MATLAB \gls{csda} and \gls{topas} $\symup{R_{80}}$ ranges.
\begin{table}[h!]
    \centering
    \caption{Comparison of \gls{csda} ranges calculated analytically in MATLAB ($\symup{r_{ana}}$) and $\symup{R_{80}}$ ranges extracted from a \gls{mc} simulation depth dose curve in \gls{topas} ($\symup{r_{MC}}$) for a bone phantom. Four different energy values are analyzed and the resulting differences between the two ranges can be seen.}
    \label{tab:MATLABTOPASboneB0}
    \begin{tabular}{c c c c }
        \multicolumn{4}{c}{\textbf{B = 0 T}} \\
        \hline
        E [MeV] & $\symup{r_{ana}}$ [mm] & $\symup{r_{MC}}$ [mm] & Difference [\%] \\
        \hline
        50  & 15.19 & 14.38 & 6.22 \\
        
        100 & 52.63 & 50.17 & 4.90 \\
        
        150 & 107.55 & 103.07 & 4.35 \\
        
        200 & 176.93 & 169.84 & 4.17 \\
        \hline
    \end{tabular}
\end{table}
\newpage
\subsubsection*{B > 0 T}
\label{sec:bonephantombg0t}
Relative dose values calculated with $\symup{R_{80}}$, analogous to \autoref{fig:waterb0e200mevdistance} and \autoref{tab:waterrelativedoser80} are analyzed for given bone phantom. These results are listed in table \autoref{tab:bonerelativedoser80}.
\begin{table}[h!]
    \centering
    \caption{Relative dose values for a \gls{topas} simulation in a bone phantom with B = 0 T and four different energy values calculated with prior determined $\symup{R_{80}}$ ranges and distances.}
    \label{tab:bonerelativedoser80}
    \begin{tabular}{c c}
        %\multicolumn{4}{c}{\textbf{B = 0 T}} \\
        \hline
        E [MeV] &  Rel. Dose [\%] \\
        \hline
        50  & 87.76 \\
        
        100 & 82.85 \\
        
        150 & 82.28 \\
        
        200 & 82.48 \\
        \hline
    \end{tabular}
\end{table}
~\\Furthermore, distances calculated in MATLAB and in \gls{topas} with their differences for the given bone phantom are listed in \autoref{tab:compareMATLABTOPASdistancebone}.
\begin{table}[h!]
    \centering
    \caption{Comparison of distances calculated analytically in MATLAB ($\symup{s_{ana}}$) and with a \gls{mc} simulation in \gls{topas} ($\symup{s_{MC}}$) for a bone phantom. Different energy values and magnetic field strengths analogous to \autoref{tab:compareMATLABanalytical} are analyzed with their differences listed.}
    \label{tab:compareMATLABTOPASdistancebone}
    \begin{tabular}{c c c c }
        
        \multicolumn{4}{c}{\textbf{B = 1.5 T}} \\
        \hline
        E [MeV] & $\symup{s_{ana}}$ [mm] & $\symup{s_{MC}}$ [mm] & Difference [\%] \\
        \hline
        50  & 15.19 & 14.29 & 6.29 \\
        
        100 & 52.63 & 50.04 & 5.18 \\
        
        150 & 107.55 & 103.11 & 4.31 \\
        
        200 & 176.93 & 169.82 & 4.19 \\
        \hline
    \end{tabular}
    
    \vspace{1cm}

    \begin{tabular}{c c c c}
        
        \multicolumn{4}{c}{\textbf{E = 100 MeV}} \\
        \hline
        B [T] & $\symup{s_{ana}}$ [mm] & $\symup{s_{MC}}$ [mm] & Difference [\%] \\
        \hline
        0.5 & 52.63 & 50.15 & 4.95 \\
        
        1   & 52.63 & 50.08 & 5.09 \\
        
        2   & 52.63  & 50.00  & 5.26 \\
        
        3   & 52.63  & 49.90  & 5.47 \\
        \hline
    \end{tabular}
\end{table}
\newpage
For this bone phantom the final position coordinates have been calculated in MATLAB and \gls{topas} as well and are listed in the following \autoref{tab:compareMATLABTOPAScoordinatesbone}.
\begin{table}[h!]
    \centering
    \caption{Comparison of end positions calculated analytically in MATLAB (\(\vec{V}_{\symup{ana}}\)) and in a \gls{mc} simulation using \gls{topas} (\(\vec{V}_{\symup{MC}}\)) for a bone phantom \gls{ct} dataset assuming a constant \gls{ct} number of 1000 HU with \(\vec{V} = (x, \, y)\). Different energy values and magnetic field strengths are analyzed with their differences are calculated ($\symup{Difference} = \sqrt{(x_{\symup{MC}} - x_{\symup{ana}})^2 + (y_{\symup{MC}} - y_{\symup{ana}})^2}$) and listed.}
    \label{tab:compareMATLABTOPAScoordinatesbone}
    \begin{tabular}{c c c c }
        \multicolumn{4}{c}{\textbf{B = 1.5 T}} \\
        \hline
        E [MeV] & \(\vec{V}_{\symup{ana}}\) [mm] & \(\vec{V}_{\symup{MC}}\) [mm] & Difference [mm] \\
        \hline
        50  & 
        \((15.19, 122.68)\) & 
        \((14.29, 122.67)\) & 
        0.90 \\
        
        100 & 
        \((52.60, 124.05)\) & 
        \((50.01, 123.93)\) & 
        2.60 \\
        
        150 & 
        \((107.36, 127.95)\) & 
        \((102.92, 127.38)\) & 
        4.48 \\
        
        200 & 
        \((176.27, 135.69)\) & 
        \((169.18, 133.87)\) & 
        7.27 \\
        \hline
    \end{tabular}
    
    \vspace{1cm}
    
    \begin{tabular}{c c c c}
        \multicolumn{4}{c}{\textbf{E = 100 MeV}} \\
        \hline
        B [T] & \(\vec{V}_{\symup{ana}}\) [mm] & \(\vec{V}_{\symup{MC}}\) [mm] & Difference [mm] \\
        \hline
        0.5 & 
        \((52.63, 123.02)\) & 
        \((50.14, 122.97)\) & 
        2.49 \\
        
        1   & 
        \((52.62, 123.53)\) & 
        \((50.06, 123.46)\) & 
        2.56 \\
        
        2   & 
        \((52.58, 124.56)\) & 
        \((49.94, 124.42)\) & 
        2.65 \\
        
        3   & 
        \((52.51, 125.59)\) & 
        \((49.78, 125.38)\) & 
        2.73 \\
        \hline
    \end{tabular}
\end{table}
\newpage
\subsection{Prostate patient}
\label{sec:prostate}
\subsubsection*{B = 0 T}
Finally, a complex volume of various different \gls{ct} numbers is investigated. Thus, for the given prostate patient \gls{ct} dataset, that will be further analyzed in the following sections, MATLAB \gls{csda} and \gls{topas} $\symup{R_{80}}$ ranges are listed in the following \autoref{tab:MATLABTOPASprostateB0}.
\begin{table}[h!]
    \centering
    \caption{Comparison of \gls{csda} ranges calculated analytically in MATLAB ($\symup{r_{ana}}$) and $\symup{R_{80}}$ ranges extracted from a \gls{mc} simulation depth dose curve in \gls{topas} ($\symup{r_{MC}}$) for a prostate patient \gls{ct} dataset. Four different energy values are analyzed and the resulting differences between the two ranges can be seen.}
    \label{tab:MATLABTOPASprostateB0}
    \begin{tabular}{c c c c }
        \multicolumn{4}{c}{\textbf{B = 0 T}} \\
        \hline
        E [MeV] & $\symup{r_{ana}}$ [mm] & $\symup{r_{MC}}$ [mm] & Difference [\%] \\
        \hline
        50  & 30.88 & 30.74 & 0.45 \\
        
        100 & 79.52 & 80.80 & 1.61 \\
        
        150 & 150.52 & 152.33 & 1.20 \\
        
        200 & 246.27 & 249.97 & 1.50 \\
        \hline
    \end{tabular}
\end{table}
\subsubsection*{B > 0 T}
The relative dose values calculated with $\symup{R_{80}}$ for the prostate patient are listed in \autoref{tab:prostaterelativedoser80}.
\begin{table}[h!]
    \centering
    \caption{Relative dose values for a \gls{topas} simulation in given prostate patient \gls{ct} dataset with B = 0 T and four different energy values calculated with prior determined $\symup{R_{80}}$ ranges and distances.}
    \label{tab:prostaterelativedoser80}
    \begin{tabular}{c c}
        %\multicolumn{4}{c}{\textbf{B = 0 T}} \\
        \hline
        E [MeV] &  Rel. Dose [\%] \\
        \hline
        50  & 83.33 \\
        
        100 & 83.44 \\
        
        150 & 82.95 \\
        
        200 & 79.04 \\
        \hline
    \end{tabular}
\end{table}
~\\Additionally, the distances calculated in MATLAB and \gls{topas} for the prostate patient \gls{ct} dataset are listed in table \autoref{tab:compareMATLABTOPASdistanceprostate}.
\begin{table}[h!]
    \centering
    \caption{Comparison of distances calculated analytically in MATLAB ($\symup{s_{ana}}$) and with a \gls{mc} simulation in \gls{topas} ($\symup{s_{MC}}$) for given prostate patient \gls{ct} dataset. Different energy values and magnetic field strengths analogous to \autoref{tab:compareMATLABanalytical} are analyzed with their differences listed.}
    \label{tab:compareMATLABTOPASdistanceprostate}
    \begin{tabular}{c c c c }
        
        \multicolumn{4}{c}{\textbf{B = 1.5 T}} \\
        \hline
        E [MeV] & $\symup{s_{ana}}$ [mm] & $\symup{s_{MC}}$ [mm] & Difference [\%] \\
        \hline
        50  & 30.86 & 30.41 & 1.48 \\
        
        100 & 79.25 & 80.62 & 1.73 \\
        
        150 & 149.01 & 153.60 & 3.08 \\
        
        200 & 244.63 & 256.92 & 5.02 \\
        \hline
    \end{tabular}
    
    \vspace{1cm}

    \begin{tabular}{c c c c}
        
        \multicolumn{4}{c}{\textbf{E = 100 MeV}} \\
        \hline
        B [T] & $\symup{s_{ana}}$ [mm] & $\symup{s_{MC}}$ [mm] & Difference [\%] \\
        \hline
        0.5 & 79.53 & 80.69 & 1.46 \\
        
        1   & 79.38 & 80.68 & 1.64 \\
        
        2   & 79.22  & 80.61  & 1.75 \\
        
        3   & 79.13 & 80.93  & 2.27 \\
        \hline
    \end{tabular}
\end{table}

Finally, the following \autoref{tab:compareMATLABTOPAScoordinatesprostate} lists the end position coordinates calculated in MATLAB and \gls{topas} for the given prostate patient \gls{ct} dataset.
\begin{table}[h!]
    \centering
    \caption{Comparison of end positions calculated analytically in MATLAB (\(\vec{V}_{\symup{ana}}\)) and in a \gls{mc} simulation using \gls{topas} (\(\vec{V}_{\symup{MC}}\)) for a given prostate patient \gls{ct} dataset with \(\vec{V} = (x, \, y)\). Different energy values and magnetic field strengths are analyzed. Furthermore, their differences in end positions are calculated ($\symup{Difference} = \sqrt{(x_{\symup{MC}} - x_{\symup{ana}})^2 + (y_{\symup{MC}} - y_{\symup{ana}})^2}$) and listed.}
    \label{tab:compareMATLABTOPAScoordinatesprostate}
    \begin{tabular}{c c c c }
        \multicolumn{4}{c}{\textbf{B = 1.5 T}} \\
        \hline
        E [MeV] & \(\vec{V}_{\symup{ana}}\) [mm] & \(\vec{V}_{\symup{MC}}\) [mm] & Difference [mm] \\
        \hline
        50  & 
        \((31.47, 123.25)\) & 
        \((30.39, 123.24)\) & 
        1.08 \\
        
        100 & 
        \((82.70, 126.32)\) & 
        \((80.48, 126.10)\) & 
        2.22 \\
        
        150 & 
        \((153.74, 133.71)\) & 
        \((152.73, 134.26)\) & 
        1.06 \\
        
        200 & 
        \((250.46, 149.29)\) & 
        \((254.33, 150.72)\) & 
        4.12 \\
        \hline
    \end{tabular}
    
    \vspace{1cm}
    
    \begin{tabular}{c c c c}
        \multicolumn{4}{c}{\textbf{E = 100 MeV}} \\
        \hline
        B [T] & \(\vec{V}_{\symup{ana}}\) [mm] & \(\vec{V}_{\symup{MC}}\) [mm] & Difference [mm] \\
        \hline
        0.5 & 
        \((81.01, 123.72)\) & 
        \((80.65, 123.84)\) & 
        0.36 \\
        
        1   & 
        \((81.46, 124.97)\) & 
        \((80.62, 125.00)\) & 
        0.84 \\
        
        2   & 
        \((83.93, 127.76)\) & 
        \((80.35, 127.18)\) & 
        3.64 \\
        
        3   & 
        \((83.11, 130.27)\) & 
        \((80.19, 129.01)\) & 
        3.12 \\
        \hline
    \end{tabular}
\end{table}
\section{Recalculation of spot selection for a matRad treatment plan}
\label{sec:matRadPlan}
After comparing the developed proton trajectory calculation in MATLAB with \gls{topas}, the focus is now shifted to matRad. First of all, a treatment plan is introduced with its' resulting dose distribution (see \autoref{sec:treatment}). Initial starting y-positions and energy values are exported and calculated in MATLAB for comparison of bixel end-position in the \gls{ctv} (see \autoref{sec:comparison}). Furthermore, a magnetic field is introduced and position differences are shown (see \autoref{sec:matradmagneticfield}). Finally, the gradient descent algorithm is used for optimization and recalculation of shifted coordinate positions towards the initial end positions (see \autoref{sec:matradgradientdescent}).

\subsection{Initial treatment plan}
\label{sec:treatment}
The resulting dose distribution with the objective of achieving a mean dose of $\symup{50 \, \, Gy}$ in a given \gls{ctv} is visualized in \autoref{fig:treatmentplanmatraddose}.
\begin{figure}[h!]
    \centering
    \includegraphics[width=\textwidth]{../Screenshots/matrad/treatmentmatrad.png}
    \caption{Dose distribution for given prostate \gls{ct} dataset with objective of achieving a mean dose of $\symup{50 \, \, Gy}$. A candidate \gls{ctv} (prostate) is highlighted in red, along with potential \glspl{oar} (femoral heads and rectum) in pink and dark red.}
    \label{fig:treatmentplanmatraddose}
\end{figure}

\newpage
The exported $\symup{R_{80}}$ ranges for each bixel for this treatment plan are visualized in \autoref{fig:matradr80bixels}.
\begin{figure}[h!]
    \centering
    \includegraphics[width=\textwidth]{../Screenshots/matrad/matradr80.png}
    \caption{Visualization of \gls{rspr} values representing the prostate patient \gls{ct} dataset. A simplified contour is shown representing only the body \gls{rspr} value for visualization purposes. The \gls{ctv} is highlighted (white) as well as the $\symup{R_{80}}$ ranges retrieved from the "dij" matrix after matRad irradiation (blue dots).}
    \label{fig:matradr80bixels}
\end{figure}

\newpage
\subsection{Verifying treatment plan beam positions}
\label{sec:comparison}
In order to introduce a magnetic field into matRad, the current end positions visualized in \autoref{fig:matradr80bixels} need to be reproduced in MATLAB. These initial positions and energy values are exported from the "stf" structure, imported and calculated in MATLAB. The following \autoref{fig:matradmatlabr80bixels} illustrates the end positions calculated in MATLAB and matRad.
\begin{figure}[h!]
    \centering
    \includegraphics[width=\textwidth]{../Screenshots/matrad/matradmatlabr80.png}
    \caption{Illustration of the enlarged prostate patient \gls{ct} dataset represented by \gls{rspr} values. Qualitative comparison between calculated MATLAB (orange dots) and matRad (blue dots) end positions using the same initial positions and energy values.}
    \label{fig:matradmatlabr80bixels}
\end{figure}
\newpage
A quantitative analysis is done by calculating the euclidean distances of these positions and visualizing the result in a histogram in \autoref{fig:matradmatlabr80bixelshistogram}.
\begin{figure}[h!]
    \centering
    \includegraphics[width=\textwidth]{../Screenshots/statistics/initialmatlabvsmatradhistogram.png}
    \caption{Histogram for quantitatively analyzing difference between MATLAB and matRad end positions for initial matRad treatment plan setup. Euclidean distances are calculated and the frequency can be seen. Median (red) and mean (green) values are highlighted with $\symup{med(d) = 0.49 \, \, mm}$ and $\symup{mean(d) = 0.77 \, \, mm}$.}
    \label{fig:matradmatlabr80bixelshistogram}
\end{figure}
\newpage

\subsection{Calculation of shifted spots}
\label{sec:matradmagneticfield}
A magnetic field is introduced in the proton trajectory calculation in MATLAB ($\symup{B_z = 1.5 \, \, T}$) and the qualitative results are visualized in \autoref{fig:matradmatlabr80bixelsshifted}.
\begin{figure}[h!]
    \centering
    \includegraphics[width=\textwidth]{../Screenshots/matrad/matradmatlabr80shifted.png}
    \caption{Illustration of the enlarged prostate patient \gls{ct} dataset represented by \gls{rspr} values. Qualitative comparison between calculated MATLAB (orange dots) and matRad (blue dots) end positions using the same initial positions and energy values. A magnetic field ($\symup{B_z = 1.5 \, \, T}$) is introduced in MATLAB.}
    \label{fig:matradmatlabr80bixelsshifted}
\end{figure}
\newpage
A histogram containing the euclidean distances for the shifted positions seen in \autoref{fig:matradmatlabr80bixelsshifted} is visualized in the following \autoref{fig:matradmatlabr80bixelshistogramshifted}.
\begin{figure}[h!]
    \centering
    \includegraphics[width=\textwidth]{../Screenshots/statistics/initialmatlabvsmatradhistogramshifted.png}
    \caption{Histogram for quantitatively analyzing difference between MATLAB ($\symup{B_z = 1.5 \, \, T}$) and matRad ($\symup{B_z = 0 \, \, T}$) end positions for initial matRad treatment plan setup. Euclidean distances are calculated and the frequency can be seen. Median (red) and mean (green) values are highlighted with $\symup{med(d) = 18.23 \, \, mm}$ and $\symup{mean(d) = 18.38 \, \, mm}$.}
    \label{fig:matradmatlabr80bixelshistogramshifted}
\end{figure}
\newpage
\subsection{Optimization of shifted spots}
\label{sec:matradgradientdescent}
The goal is to optimize and recalculate the spots that are shifted due to the introduction of a magnetic field, so that they converge to the initial positions. To achieve this goal, the previously explained gradient descent algorithm is applied.
~\\

Prior shown shifted positions are optimized and the result of end positions is visualized in the following \autoref{fig:gradientdescentspotsplus}.
\begin{figure}[h!]
    \centering
    \includegraphics[width=\textwidth]{../Screenshots/matrad/gradientandmatradplus.png}
    \caption{Illustration of the enlarged prostate patient \gls{ct} dataset with highlighted \gls{ctv} (white). Qualitative comparison between magnetic field influenced and optimized spots calculated in MATLAB (red dots) and matRad end positions (blue dots) for each bixel. Spots without optimization with their updated position for the first and last ray are also demonstrated (orange).}
    \label{fig:gradientdescentspotsplus}
\end{figure}
\newpage
The following \autoref{fig:matradmatlabr80bixelshistogramshiftedoptimized} represents the quantitative analysis of the optimized spots in a histogram.
\begin{figure}[h!]
    \centering
    \includegraphics[width=\textwidth]{../Screenshots/statistics/initialmatlabvsmatradhistogramshiftedoptimized.png}
    \caption{Histogram for quantitatively analyzing difference between MATLAB ($\symup{B_z = 1.5 \, \, T}$) and matRad ($\symup{B_z = 0 \, \, T}$) end positions for initial matRad treatment plan setup. Optimization has been introduced to converge towards matRad positions. Euclidean distances are calculated and the frequency can be seen. Median (red) and mean (green) values are highlighted with $\symup{med(d) = 1.06 \, \, mm}$ and $\symup{mean(d) = 1.47 \, \, mm}$.}
    \label{fig:matradmatlabr80bixelshistogramshiftedoptimized}
\end{figure}