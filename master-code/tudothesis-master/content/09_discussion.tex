\chapter{Discussion}
\label{sec:discussion}
In order to achieve a possible magnetic field influenced pencil beam spot selection for the \gls{tps} matRad, several assumptions and approximations were made. In addition, simplified settings and initializations were made. In this chapter, the results presented so far are discussed and possible limitations are mentioned.
~\\

\section{Verification of proton stopping position}
\label{sec:protontrajectorydiscussion}
\subsection{Vacuum state}
\label{sec:vacuumdiscussion}
In the analysis of a vacuum state, the calculated proton trajectory was compared to the analytical results derived using \autoref{eqn:radiusmagnet}. Systematic uncertainties may arise from inaccuracies in the variables used for calculations. For the four candidate energy values analyzed, as shown in \autoref{tab:compareMATLABanalytical}, the radii computed by the developed proton transfer algorithm were consistently slightly larger than the corresponding analytical results from \autoref{eqn:radiusmagnet}. The difference between $r_{\symup{ana}}$ and $r_{\symup{eq}}$ decreased with increasing energy values and converged to a constant relative difference as the magnetic field strength varied.
~\\

A possible explanation for this observation is that the developed proton transfer vacuum calculation assumes a small energy loss of $\symup{\frac{\mathrm{d}E}{\mathrm{d}x} = 10^{-14} \, \, \frac{\mathrm{keV}}{\mu \mathrm{m}}}$ to approximate vacuum conditions. In an ideal vacuum, no energy loss would occur. Consequently, the analytical results from \autoref{eqn:radiusmagnet} are likely closer to the true values than those obtained using the developed proton transfer algorithm. However, the magnitude of this error in the developed algorithm is minimal. Testing with $\symup{\frac{\mathrm{d}E}{\mathrm{d}x} = 10^{-9} \, \, \frac{\mathrm{keV}}{\mu \mathrm{m}}}$ yielded identical results.
~\\

Voxel-wise spatial parameterization introduces additional systematic uncertainties. While the analytical calculation assumes an infinitesimally small grid size, this thesis used a grid size of $\symup{1.09375 \, \, \mathrm{mm}}$, limiting the resolution for results below this threshold. Updating the proton position only at the start of each voxel may further contribute to the observed discrepancies. This uncertainty applies to all subsequent cases analyzed.
\subsection{Water phantom}
\label{sec:waterphantomdiscussion}
\subsubsection{B = 0 T}
\label{sec:zeromagneticfielddiscussionwater}
For this analysis, as well as the analyses of the bone phantom and prostate patient \gls{ct} dataset, the results of the developed proton transfer algorithm were compared with \gls{mc} simulations in \gls{topas}. This introduces the possibility of statistical uncertainties, particularly since all \gls{topas} simulations used a constant number of initial protons ($\symup{N = 10^5}$) (see \autoref{sec:parametersetup}). For the water phantom without a magnetic field, further simulations with $\symup{N = 10^7}$ were conducted. These showed a proton range difference of $\symup{0.01 \, \, mm}$. Thus, the error magnitude from using $\symup{N = 10^5}$ is relatively low. All subsequent statements assume that \gls{topas} serves as the reference for validating the results of the developed proton transfer algorithm. Result discrepancies may also stem from uncertain systematic inputs in \gls{topas}.
~\\

For $\symup{B = 0 \, \, \mathrm{T}}$, the assumption $\symup{CSDA} \approx R_{\symup{80}}$ was applied to compare analytical proton stopping positions with \gls{mc} based stopping positions in \gls{topas}, using the percentage depth dose curve (\autoref{eqn:csdar80}). Results in \autoref{tab:MATLABTOPASwaterB0} indicate that absolute and relative differences between the analytical and \gls{mc} proton ranges decrease with increasing energy. Notably, the relative difference converges toward $\symup{0 \, \, \%}$ at $E = \symup{200 \, \, MeV}$. Additional simulations at $E = \symup{225 \, \, MeV}$ confirmed a similar convergence near $\symup{0 \, \, \%}$, with only a $\symup{0.0088 \, \, \%}$ increase in relative difference.
~\\

Differences between proton ranges calculated using the developed proton transfer algorithm and \gls{mc} simulations may arise from systematic uncertainties due to the grid size resolution in the developed algorithm. Additionally, while the Schneider \gls{hulut} implementation in \gls{topas} is assumed to match that used in the developed algorithm, subtle differences could affect stopping power calculations and, consequently, proton stopping positions. Variations in underlying physics models also contribute. For instance, greater scattering effects modeled in \gls{topas} may shorten proton ranges. The default physics models in \gls{topas} include six modules \cite{perl_default_nodate}, which account for processes such as particle decay and elastic, inelastic, and capture interactions \cite{noauthor_decay_nodate,noauthor_qgsp_bic_nodate}. In contrast, the Bethe Bloch implementation in "libamtrack" used in this thesis focuses primarily on electromagnetic processes \cite{grzanka_libamtrack_nodate}.
~\\

The \gls{topas} beam model, briefly described in \autoref{sec:parametersetup} and detailed in appendix \ref{sec:appendixtopassetup}, introduces additional systematic uncertainties. A simplified beam model was employed to approximate the developed narrow proton transfer beam. Further analysis of different beam settings may reduce discrepancies between the \gls{mc}-simulated proton ranges in \gls{topas} and the analytically calculated ranges from the developed proton transfer algorithm.
~\\

Another source of systematic uncertainty arises from the positioning of the beam source relative to the width of the irradiated \gls{ct} dataset. For the water phantom, bone phantom, and prostate patient, 62 slices with a slice thickness of $\symup{0.2 \, \, \mathrm{cm}}$ were imported, yielding a total width of $\symup{12.4 \, \, \mathrm{cm}}$. In this thesis, it is assumed that the current beam setup fully covers the entire phantom geometry. Investigating this factor, along with alternative beam configurations, could uncover smaller absolute and relative differences between proton ranges calculated analytically using the developed proton transfer algorithm and those derived from the \gls{mc} platform \gls{topas}. These uncertainties are expected to persist across all subsequent \gls{topas} simulations.
\subsubsection{B > 0 T}
In this thesis, primarily homogeneous magnetic fields were investigated. However, in \gls{mript}, proton pencil beam delivery is also influenced by the heterogeneous fringe fields of an MR scanner. Machine-specific look-up tables could incorporate the effects of such fringe fields \cite{duetschler_fast_2023}.

~\\For both trajectory lengths, listed in \autoref{tab:compareMATLABTOPASdistancewater}, and proton stopping positions, shown in \autoref{tab:compareMATLABTOPAScoordinateswater}, the absolute and relative differences between results from the developed proton transfer algorithm and \gls{mc} simulations appear energy dependent. Trajectories and stopping positions were analyzed using a fixed magnetic field strength with varying energy values and a fixed energy value with varying magnetic field strengths.

~\\From \autoref{tab:compareMATLABTOPAScoordinateswater}, it can be observed that for $B_z = 1.5 \, \, \mathrm{T}$ and varying energy values, the Euclidean distances between positions calculated in the developed proton transfer algorithm and with \gls{mc} simulations decrease up until the analysis of $E = 200 \, \, \mathrm{MeV}$. When energy is fixed and magnetic field strength varies, no significant changes in distance values are observed, indicating an energy dependence. This trend could be attributed to systematic uncertainties in the developed proton transfer algorithm. Additionally, the proposed method for calculating proton stopping positions in magnetic fields may introduce further uncertainties. Exploring alternative methods based on \gls{mc} simulated pencil beam trajectories influenced by magnetic fields could improve accuracy.
\subsection{Bone phantom}
\label{sec:bonephantomdiscussion}
\subsubsection{B = 0 T}
\label{sec:zeromagneticfielddiscussionbone}
For the bone phantom without a magnetic field, proton ranges are listed in \autoref{tab:MATLABTOPASboneB0}. The ranges calculated using the developed proton transfer algorithm are consistently larger than those obtained from \gls{mc} simulations. Absolute differences increase from $\symup{0.81 \, \, \mathrm{mm}}$ at $E = \symup{50 \, \, \mathrm{MeV}}$ to $\symup{7.09 \, \, \mathrm{mm}}$ at $E = \symup{200 \, \, \mathrm{MeV}}$. Relative differences appear to converge toward a constant value of approximately $\symup{4.76 \, \, \%}$, indicating the presence of systematic uncertainties in proton range calculations.
~\\

A major source of these uncertainties could be differences in how \gls{ct} numbers, here $\symup{1000 \, \, \mathrm{HU}}$, are converted into \gls{rspr} values between the developed proton transfer algorithm and the \gls{mc} simulations. Further analysis of the Schneider \gls{hulut} method and its implementation may confirm or refute this hypothesis. Investigating alternative stopping power calculation methods could also be beneficial. For example, the \gls{mata} approach maps \gls{rspr} values to material properties using 40 predefined material compositions representative of human tissues, with mass density determined through linear interpolation \cite{permatasari_material_2020}.
\subsubsection{B > 0 T}
The absolute trajectory length differences for a fixed magnetic field strength and varying energy values (see \autoref{tab:compareMATLABTOPASdistancebone}) calculated using the developed proton transfer algorithm and \gls{mc} simulations increase with energy. Conversely, for a fixed energy value and varying magnetic field strength, these differences converge to a nearly constant range of $\symup{2.48 \, \, \mathrm{mm}}$ to $\symup{2.73 \, \, \mathrm{mm}}$. Relative differences similarly stabilize around a constant value, as observed in the analysis without a magnetic field. A comparable trend is evident when examining proton stopping positions instead of distances (see \autoref{tab:compareMATLABTOPAScoordinatesbone}).
~\\

In the bone phantom analysis, each voxel is assigned the same \gls{ct} number. For a fixed magnetic field strength and varying energy values, the increasing absolute differences suggest an energy dependent uncertainty. This trend is further supported by the nearly constant absolute differences observed for fixed energy values. These uncertainties appear multifactorial. Similar to the analysis of magnetic field influenced proton stopping positions in the water phantom (see \autoref{sec:waterphantomdiscussion}), the method used to calculate proton stopping positions from trajectory lengths may contribute to the observed discrepancies. Additionally, differences in \gls{rspr} value calculation and other systematic uncertainties within the developed proton transfer algorithm could further amplify the variations in proton stopping positions for fixed energy and magnetic field strength values.
\subsection{Prostate patient}
\label{sec:prostatephantomdiscussion}
\subsubsection{B = 0 T}
\label{sec:zeromagneticfielddiscussionprostate}
The imaging for the investigated prostate patient dataset was performed using \gls{ct}, which directly provides \gls{ct} numbers for conversion into \gls{rspr} values. The availability of the prostate patient dataset and the simplicity of stopping power calculation are the primary reasons for focusing on \gls{ct} slices in this thesis. Since the goal is \gls{mript}, developing a method to convert \gls{mri} datasets into \gls{ct} datasets, potentially in real-time, could represent a significant step toward the realization of \gls{mript}. In this context, synthetic \gls{ct} methods could be explored \cite{boulanger_deep_2021}.
~\\

The results in \autoref{tab:MATLABTOPASprostateB0} show that, with increasing energy values, the absolute difference between proton ranges calculated using the developed proton transfer algorithm and those obtained from \gls{mc} simulations decreases. At a specific energy value ($E = \symup{150 \, \, \mathrm{MeV}}$), the proton ranges derived from \gls{mc} simulations exceed those calculated by the developed proton transfer algorithm. This observation may stem from differences in the stopping power calculation methods employed in the proton transfer algorithm and the \gls{mc} simulations. This systematic uncertainty is particularly evident in the prostate patient dataset, where each voxel has a unique \gls{ct} number. Further investigations using more sophisticated stopping power calculation algorithms, such as the \gls{mata} approach, could help address these discrepancies. Additionally, uncertainties in proton stopping positions, both with and without a magnetic field, are amplified by statistical effects. For example, the proton transfer algorithm computes the beam trajectory using a single proton under idealized conditions, whereas \gls{mc} simulations account for the most probable stopping positions of multiple initial protons.
\newpage
\subsubsection{B > 0 T}
For the trajectory lengths listed in \autoref{tab:compareMATLABTOPASdistanceprostate}, similar observations and explanations can be drawn as for the proton ranges without a magnetic field discussed previously.
~\\

The proton stopping positions calculated using the developed proton transfer algorithm and derived from \gls{mc} simulations, as shown in \autoref{tab:compareMATLABTOPAScoordinatesprostate}, do not exhibit a consistent trend in absolute differences. For a fixed magnetic field strength and increasing energy values, the absolute differences and distances show no clear trend of increase or decrease, supporting the hypothesis of discrepancies in stopping power calculations. Similarly, for a fixed energy value and varying magnetic field strengths, a jump of approximately $\symup{3 \, \, \mathrm{mm}}$ is observed when increasing from $B_z = 1 \, \, \mathrm{T}$ to $B_z = 2 \, \, \mathrm{T}$.
~\\

This discrepancy may be attributed to differences in stopping power calculations, which result in trajectories passing through voxels with different \gls{ct} numbers. For instance, a specific proton state $i$ calculated using the developed algorithm might pass through a voxel with a \gls{ct} number representing air, while the same proton state $i$ in the \gls{mc} simulation might pass through a voxel with a \gls{ct} number representing bone. Consequently, the proton stopping position calculated using the developed algorithm would yield a larger value compared to that derived from the \gls{mc} simulation.
\section{Recalculation of spot selection for a matRad treatment field}
\label{sec:recalculaationspotdiscussion}
\subsection{Proton treatment field}
\label{sec:protontreaatmentfielddiscussion}
As shown in \autoref{fig:treatmentplanmatraddose}, a candidate proton treatment field was generated using matRad. However, this field is unsuitable as a treatment plan due to inadequate coverage of the \gls{ctv} and the potential for overdosing. Additionally, the \glspl{oar} were not considered during its creation. Further analysis is required to evaluate the use of additional beams, potentially from opposing directions, and to define dose constraints for the \glspl{oar}, ensuring clinically relevant dose coverage of the \gls{ctv}. Moreover, a constant proton \gls{rbe} value of 1.1 was assumed in the dose calculations. Recent studies have demonstrated that the \gls{rbe} of protons varies along the particle track, which could influence treatment effectiveness \cite{paganetti_relative_2002}.
\newpage
\subsection{Verifying treatment field beam positions}
The \autoref{fig:matradmatlabr80bixels} suggests a consistent trend within each row. Specifically, an approximately uniform difference is observed between the proton stopping positions calculated using the developed proton transfer algorithm and those extracted from matRad. A possible explanation for row-specific differences is the traversal of different materials in the prostate patient \gls{ct} slice. This highlights the potential for stopping power miscalculations in the developed proton transfer algorithm, discrepancies in stopping power calculation methods between the developed algorithm and matRad, or both.
~\\

Systematic uncertainties in proton stopping position calculations are further emphasized when analyzing the Euclidean distances between stopping positions calculated using the developed algorithm and matRad (see \autoref{fig:matradmatlabr80bixelshistogram}). The median distance of $\symup{0.49 \, \, \mathrm{mm}}$, approximately half the resolution size, supports the presence of systematic uncertainty. Additionally, random errors in stopping position estimation may arise within each voxel in matRad. For this thesis, it was assumed that proton stopping positions calculated using both the developed proton transfer algorithm and matRad occur at the same point within a voxel.
~\\
Further investigation into stopping power calculation methods, grid resolution, and the analysis of stopping positions within a voxel could help reduce these differences and improve the agreement between the developed proton transfer algorithm and matRad.

\subsection{Calculation of deflected spots}
For the calculation of deflected spots, only a single magnetic field strength ($B_z = \symup{1.5 \, \, \mathrm{T}}$) was analyzed. Consequently, the results concerning the order of magnitude of the distance between magnetic field influenced pencil beam spots calculated using the developed proton transfer algorithm and spots extracted from matRad without a magnetic field allow for only a single hypothesis. At the analyzed magnetic field strength, an approximate distance of $\symup{18 \, \, \mathrm{mm}}$ is observed (see \autoref{fig:matradmatlabr80bixelshistogramshifted}). This significant shift in spot positioning, also visualized in \autoref{fig:matradmatlabr80bixelsshifted}, intuitively suggests the need for optimization.
~\\

Further analysis incorporating additional magnetic field strengths and advanced statistical approaches is required to confirm or refute the hypothesis that optimization of pencil beam spot positioning is necessary to account for the effects of the magnetic field.
\newpage
\subsection{Optimization of deflected spots}
When analyzing \autoref{fig:gradientdescentspotsplus}, optimized spots calculated using the developed proton transfer algorithm in the middle row appear to converge more closely to matRad spots than those in the first and last rows. Ideally, the distances between optimized stopping positions under a magnetic field and those without a magnetic field should converge. However, \autoref{fig:matradmatlabr80bixelshistogramshiftedoptimized} shows that the median distance of optimized spots under a magnetic field is approximately double that of spots without a magnetic field. Specifically, the median value increased from $\symup{0.49 \, \, \mathrm{mm}}$ to $\symup{1.06 \, \, \mathrm{mm}}$. This discrepancy likely arises from multiple factors.
~\\

First, the introduced magnetic field deflects spots in both the $x$ and $y$ directions, adding an extra dimension of uncertainty to the pre-existing systematic uncertainties in spot positioning. Second, the gradient descent optimization method introduces its own uncertainties, as it depends on parameters such as the learning rate, convergence criteria, and maximum iterations. The settings used (see \autoref{sec:gradientdescentmaterial}) may have been insufficient for achieving optimal results due to computational constraints. The choice of the learning rate is critical. If the learning rate is too large, the algorithm may diverge, while if too small, convergence can be slow \cite{boyd_convex_2004}. Additionally, the optimization was applied to a single \gls{ct} slice, limiting its generalizability.
~\\

The observed increase in distances may result from a combination of uncertainties in the proton stopping position calculations by the developed algorithm and those introduced by suboptimal optimization of pencil beam spots.
~\\

To improve optimization under a magnetic field, the stopping position calculations in the developed algorithm should first be re-evaluated. Further analysis of the optimization algorithm and its parameters is also necessary. Expanding the analysis to multiple \gls{ct} slices could provide further validation of the current results. Additionally, exploring alternative optimization techniques could reduce computational costs while enhancing accuracy.
~\\

A potential improvement involves utilizing the same gradient descent algorithm with modifications in spot selection to reduce computational expense, as introduced in the following chapter.
