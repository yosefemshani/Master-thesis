\thispagestyle{plain}

\section*{Abstract}
Proton therapy, renowned for its precision and reduced healthy tissue damage, benefits from image guidance to further enhance treatment efficacy. However, \gls{mript} presents several challenges, one of which arises from the deflection of charged proton beams by the magnetic field of the \gls{mri} scanner. The developed proton transfer algorithm addresses this challenge by calculating proton trajectories through various materials under the influence of a magnetic field, validated against \gls{mc} simulations. The algorithm was tested across cases of increasing complexity, from vacuum to patient-specific phantoms, with differences between calculated and reference results analyzed. Notably, minimal differences were observed for water phantoms, while bone phantoms exhibited the largest discrepancies. The inclusion of optimization via a gradient descent algorithm allowed for adjustment of pencil beam spots to account for deflections. Median stopping position distances increased from $\symup{0.49 \, \, \mathrm{mm}}$ without a magnetic field to $\symup{1.06 \, \, \mathrm{mm}}$ under a $B_z = \symup{1.5 \, \, \mathrm{T}}$ field and optimization, highlighting the need for further refinement.
\section*{Kurzfassung}
\begin{foreignlanguage}{german}
Die Protonentherapie, die für ihre Präzision und geringere Schädigung des gesunden Gewebes bekannt ist, profitiert von der Bildführung, um die Wirksamkeit der Behandlung weiter zu verbessern. Die \gls{mript} stellt einige Herausforderungen dar, wovon eine die Ablenkung der geladenen Protonenstrahlen durch das Magnetfeld des Scanners ist. Der entwickelte Algorithmus für den Protonentransfer geht auf diese Herausforderung ein, indem er die Trajektorien von Protonen durch verschiedene Materialien unter dem Einfluss eines Magnetfeldes berechnet und anhand von \gls{mc} Simulationen validiert. Der Algorithmus wurde in Fällen mit zunehmender Komplexität getestet, vom Vakuum bis hin zu patientenspezifischen Phantomen, wobei die Unterschiede zwischen den berechneten und den Referenzergebnissen analysiert wurden. Insbesondere wurden minimale Unterschiede bei Wasserphantomen beobachtet, während Knochenphantome die größten Diskrepanzen aufwiesen. Die Optimierung mittels eines Gradientenabstiegsverfahren ermöglichte die Anpassung der Pencil Beam Spots, um Ablenkungen zu berücksichtigen. Die mittleren Abstände stiegen von $\symup{0.49 \, \, \mathrm{mm}}$ ohne Magnetfeld auf $\symup{1.06 \, \, \mathrm{mm}}$ unter einem $B_z = \symup{1.5 \, \, \mathrm{T}}$ Feld und Optimierung, was die Notwendigkeit einer weiteren Verfeinerung unterstreicht.
\end{foreignlanguage}
