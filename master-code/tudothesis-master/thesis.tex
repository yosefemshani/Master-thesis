%%%%%%%%%%%%%%%%%%%%%%%%%%%%%%%%%%%%%%%%%%%%%%%%%%%%%%%%%%%%%%%%%%%%%%%%%%%%%%%%
%%%%%%%%%%%%%%%%%%   Vorlage für eine Abschlussarbeit   %%%%%%%%%%%%%%%%%%%%%%%%
%%%%%%%%%%%%%%%%%%%%%%%%%%%%%%%%%%%%%%%%%%%%%%%%%%%%%%%%%%%%%%%%%%%%%%%%%%%%%%%%

% Erstellt von Maximilian Nöthe, <maximilian.noethe@tu-dortmund.de>
% ausgelegt für lualatex und Biblatex mit biber

% Kompilieren mit
% latexmk --lualatex --output-directory=build thesis.tex
% oder einfach mit:
% make

\documentclass[
  tucolor,       % remove for less green,
  BCOR=12mm,     % 12mm binding corrections, adjust to fit your binding
  parskip=half,  % new paragraphs start with half line vertical space
  open=any,      % chapters start on both odd and even pages
  cleardoublepage=plain,  % no header/footer on blank pages
]{tudothesis}


% Warning, if another latex run is needed
\usepackage[aux]{rerunfilecheck}

% just list chapters and sections in the toc, not subsections or smaller
\setcounter{tocdepth}{2} % 1 for no subsections in contents, 2 for subsections in contents

%------------------------------------------------------------------------------
%------------------------------ Fonts, Unicode, Language ----------------------
%------------------------------------------------------------------------------
\usepackage{fontspec}
\defaultfontfeatures{Ligatures=TeX}  % -- becomes en-dash etc.

% load english (for abstract) and ngerman language
% the main language has to come last
\usepackage[american, american]{babel}

% intelligent quotation marks, language and nesting sensitive
\usepackage[autostyle]{csquotes}

% microtypographical features, makes the text look nicer on the small scale
\usepackage{microtype}

%------------------------------------------------------------------------------
%------------------------ Math Packages and settings --------------------------
%------------------------------------------------------------------------------

\usepackage{amsmath}
\usepackage{amssymb}
\usepackage{mathtools}

% Enable Unicode-Math and follow the ISO-Standards for typesetting math
\usepackage[
  math-style=ISO,
  bold-style=ISO,
  sans-style=italic,
  nabla=upright,
  partial=upright,
  warnings-off={mathtools-colon,mathtools-overbracket}, % suppress some unnecessary warnings
]{unicode-math}
\setmathfont{Latin Modern Math}

% nice, small fracs for the text with \sfrac{}{}
\usepackage{xfrac}


%------------------------------------------------------------------------------
%---------------------------- Numbers and Units -------------------------------
%------------------------------------------------------------------------------

\usepackage[
  locale=US,
  separate-uncertainty=true,
  per-mode=symbol-or-fraction,
]{siunitx}

%------------------------------------------------------------------------------
%-------------------------------- tables  -------------------------------------
%------------------------------------------------------------------------------

\usepackage{booktabs}       % \toprule, \midrule, \bottomrule, etc

%------------------------------------------------------------------------------
%-------------------------------- graphics -------------------------------------
%------------------------------------------------------------------------------

\usepackage{graphicx}
% currently broken
% \usepackage{grffile}

% allow figures to be placed in the running text by default:
\usepackage{scrhack}
\usepackage{float}
\floatplacement{figure}{htbp}
\floatplacement{table}{htbp}

% keep figures and tables in the section
\usepackage[section, below]{placeins}

% allows to include PDFs as full pages
\usepackage{pdfpages}

% Set the PDF Version of this document to 1.7 (1.4 is the current default)
% This is needed so that PDFs with Version >1.5 can be included
\pdfvariable minorversion=7

%------------------------------------------------------------------------------
%---------------------- customize list environments ---------------------------
%------------------------------------------------------------------------------

\usepackage{enumitem}

%------------------------------------------------------------------------------
%------------------------------ Bibliographie ---------------------------------
%------------------------------------------------------------------------------

\usepackage[
  backend=biber,   % use modern biber backend
  autolang=hyphen,
  style = numeric,
  sorting= none,   % load hyphenation rules for if language of bibentry is not
                   % german, has to be loaded with \setotherlanguages
                   % in the references.bib use langid={en} for english sources
]{biblatex}
\addbibresource{references.bib}  % the bib file to use
\DefineBibliographyStrings{american}{andothers = {{et\,al\adddot}}}  % replace u.a. with et al.


% Last packages, do not change order or insert new packages after these ones
\usepackage[pdfusetitle, unicode, linkbordercolor=tugreen, citebordercolor=tugreen]{hyperref}
\usepackage[savewrites,nonumberlist, seeautonumberlist, nopostdot, acronym, style=super]{glossaries} %style long or default: list
\usepackage{bookmark}
\usepackage[shortcuts]{extdash}


\usepackage{pythonhighlight}
\usepackage{tikz}


%------------------------------------------------------------------------------
%-------------------------    Glossary   --------------------------------------
%------------------------------------------------------------------------------ 

\newcommand\acrfullr[2][]{\acrshort[#1]{#2}: \acrlong[#1]{#2}}
% increase space and align entrys similar
\newglossarystyle{mystyle}{%
  \glossarystyle{long}%
  %\renewcommand{\glsgroupskip}{}
  \renewenvironment{theglossary}%
     {\begin{longtable}{N p{\glsdescwidth}}}%
     {\end{longtable}}%
} 

%\setglossarystyle{mystyle} 
\makeglossaries
\renewcommand{\glsnamefont}[1]{\textbf{#1}}


% medical physics basics
\newacronym{ct}{CT}{computed tomography}
\newacronym{mri}{MRI}{magnetic resonance imaging}
\newacronym{sct}{sCT}{synthetic computed tomography}
\newacronym{mript}{MRiPT}{MR-integrated proton therapy}
%\newacronym{hu}{HU}{Hounsfield unit}
\newacronym{dicom}{DICOM}{Digital Imaging and Communications in Medicine}
\newacronym{uid}{UID}{unique identifier}
\newacronym{gui}{GUI}{graphical user interface}
% radiotherapy
\newacronym{rt}{RT}{radiotherapy}
\newacronym{tps}{TPS}{treatment planning system}
\newacronym{topas}{TOPAS}{Tool for Particle Simulation}
\newacronym{csda}{CSDA}{continuous slowing down approximation}
\newacronym[longplural={Hounsfield look-up tables}]{hulut}{HLUT}{Hounsfield look-up table}
\newacronym[longplural={Hounsfield units}]{hu}{HU}{Hounsfield unit}
\newacronym{rspr}{rSPR}{relative stopping power ratio}
\newacronym{mata}{MATA}{material assignment}
\newacronym{rbe}{RBE}{relative biological effectiveness}
\newacronym{linac}{LINAC}{linear accelerator}


\newacronym{mc}{MC}{Monte Carlo}

\newacronym{dvh}{DVH}{dose volume histogram}
\newacronym[longplural={regions of interest}]{roi}{ROI}{region of interest}
\newacronym[longplural={organs at risk}]{oar}{OAR}{organ at risk}
\newacronym{dsc}{DSC}{Dice similarity coefficient}

\newacronym{gli}{GLI}{glioblastoma}
\newacronym{gtv}{GTV}{gross tumor volume}
\newacronym{ctv}{CTV}{clinical target volume}
\newacronym{ptv}{PTV}{planning target volume}

% institutions
\newacronym{nci}{NCI}{National Cancer Institute}
\newacronym{uf}{UF}{University of Florida}
\newacronym{aapm}{AAPM}{American Association of Physicists in Medicine}
\newacronym{dkfz}{DKFZ}{German Cancer Research Center}
\newacronym{icrp}{ICRP}{International Commission on Radiological Protection}
\newacronym{icru}{ICRU}{International Commission on Radiation Units and Measurements}
\newacronym{nhanes}{NHANES}{National Health and Nutrition Examination Surveys}
\newacronym{cdc}{CDC}{Centers for Disease Control and Prevention}

% general
\newacronym{1D}{1D}{one-dimensional}
\newacronym{2D}{2D}{two-dimensional}
\newacronym{3D}{3D}{three-dimensional}
\newacronym{id}{ID}{identification number}

%------------------------------------------------------------------------------
%-------------------------    Angaben zur Arbeit   ----------------------------
%------------------------------------------------------------------------------

\author{Yosef Emshani}
\title{Development of magnetic field influenced pencil beam spot selection for proton therapy treatment planning}
\date{2024}
\birthplace{Castrop-Rauxel}
\chair{Lühr Group}
\division{Department of Physics}
\thesisclass{Master of Science}
\submissiondate{December 2, 2024}
\firstcorrector{Prof. Dr. Armin Lühr}
\secondcorrector{Prof. Dr. Kevin Kröninger}

% tu logo on top of the titlepage
\titlehead{\includegraphics[height=1.5cm]{logos/tu-logo.pdf}}

\begin{document}

\def\figureautorefname{figure}
\def\equationautorefname{equation}
\def\tableautorefname{table}
\def\sectionautorefname{section}


\frontmatter
\maketitle

% Gutachterseite
\makecorrectorpage

% hier beginnt der Vorspann, nummeriert in römischen Zahlen
\thispagestyle{plain}

\section*{Abstract}
Proton therapy, renowned for its precision and reduced healthy tissue damage, benefits from image guidance to further enhance treatment efficacy. However, \gls{mript} presents several challenges, one of which arises from the deflection of charged proton beams by the magnetic field of the \gls{mri} scanner. The developed proton transfer algorithm addresses this challenge by calculating proton trajectories through various materials under the influence of a magnetic field, validated against \gls{mc} simulations. The algorithm was tested across cases of increasing complexity, from vacuum to patient-specific phantoms, with differences between calculated and reference results analyzed. Notably, minimal differences were observed for water phantoms, while bone phantoms exhibited the largest discrepancies. The inclusion of optimization via a gradient descent algorithm allowed for adjustment of pencil beam spots to account for deflections. Median stopping position distances increased from $\symup{0.49 \, \, \mathrm{mm}}$ without a magnetic field to $\symup{1.06 \, \, \mathrm{mm}}$ under a $B_z = \symup{1.5 \, \, \mathrm{T}}$ field and optimization, highlighting the need for further refinement.
\section*{Kurzfassung}
\begin{foreignlanguage}{german}
Die Protonentherapie, die für ihre Präzision und geringere Schädigung des gesunden Gewebes bekannt ist, profitiert von der Bildführung, um die Wirksamkeit der Behandlung weiter zu verbessern. Die \gls{mript} stellt einige Herausforderungen dar, wovon eine die Ablenkung der geladenen Protonenstrahlen durch das Magnetfeld des Scanners ist. Der entwickelte Algorithmus für den Protonentransfer geht auf diese Herausforderung ein, indem er die Trajektorien von Protonen durch verschiedene Materialien unter dem Einfluss eines Magnetfeldes berechnet und anhand von \gls{mc} Simulationen validiert. Der Algorithmus wurde in Fällen mit zunehmender Komplexität getestet, vom Vakuum bis hin zu patientenspezifischen Phantomen, wobei die Unterschiede zwischen den berechneten und den Referenzergebnissen analysiert wurden. Insbesondere wurden minimale Unterschiede bei Wasserphantomen beobachtet, während Knochenphantome die größten Diskrepanzen aufwiesen. Die Optimierung mittels eines Gradientenabstiegsverfahren ermöglichte die Anpassung der Pencil Beam Spots, um Ablenkungen zu berücksichtigen. Die mittleren Abstände stiegen von $\symup{0.49 \, \, \mathrm{mm}}$ ohne Magnetfeld auf $\symup{1.06 \, \, \mathrm{mm}}$ unter einem $B_z = \symup{1.5 \, \, \mathrm{T}}$ Feld und Optimierung, was die Notwendigkeit einer weiteren Verfeinerung unterstreicht.
\end{foreignlanguage}

\tableofcontents
\newpage

%\glsaddall
\glsfindwidesttoplevelname

\printglossary[title={List of abbreviations}, type=\acronymtype]


\mainmatter
% Hier beginnt der Inhalt mit Seite 1 in arabischen Ziffern
\chapter{Introduction}
%first paragraph:
%- radiotherapy is relevant method to treat cancer
%- conventional radiotherapy uses photons
%- protons in proton therapy show greater sparing of health tissue \cite{suit_clinical_1977}
%
%second paragraph:
%- high precision of deposition in proton therapy leads to sensitivity to morphological changes 
%- thus, image guided proton therapy could maximize sparing health tissue
%- MRI good candidate imaging since it has excellent, soft-tissue constrast and no ionising dose \cite{schellhammer_integrating_2018}
%- thus growing interest to investiagte MR-integrated proton therapy (MRiPT)
%
%third paragraph (focus of this thesis):
%- basically: create ground work for MRiPT planning for a candidate treatment planning system
%for that:
%- develop an algorithm which allows calculating the proton transfer in various materials under influence of a magnetic field
%- export proton pencil beam status from example treatment plan and import in developed algorithm
%- run optimization algorithm to account for the deflection due to the magnetic field in order to roughly have same pencil beam spots as seen without a magnetic field
Cancer therapy currently comprises three primary modalities: surgery, chemotherapy, and radiotherapy \cite{zeman_basics_2020}. Among these, radiotherapy is widely applied, either as a standalone treatment or in combination with other therapeutic approaches \cite{p_world_nodate}. It is a well-established and highly effective method for treating cancer. Conventional radiotherapy predominantly uses photons to deliver the prescribed dose to tumor sites. However, proton therapy offers notable advantages, particularly in its ability to spare healthy tissue, owing to the unique dose deposition characteristics of protons \cite{suit_clinical_1977}. Consequently, the number of treatments and centers utilizing particle therapy, particularly proton therapy, has significantly increased in recent years \cite{han_current_2019}.
~\\

The high precision of energy deposition in proton therapy, while beneficial, also makes it particularly sensitive to morphological changes in patient anatomy. Consequently, image-guided proton therapy has the potential to further enhance the sparing of healthy tissue. \Gls{mri} is a promising candidate for image guidance due to its excellent soft-tissue contrast and the absence of ionizing radiation \cite{schellhammer_integrating_2018}. This has led to growing interest in exploring \gls{mript}. Although MR-guided photon therapy already has commercial systems in clinical use \cite{duetschler_fast_2023}, \gls{mript} remains a novel concept under investigation. Unlike photons, the primary proton beam is deflected by the magnetic field of an \gls{mri} system due to its charged nature. This deflection must be considered when determining proton stopping positions within the specified target in the patient.
~\\

This thesis establishes the foundation for integrating magnetic field influenced pencil beam spot selection within a candidate \gls{tps}. To achieve this, an algorithm is proposed to calculate proton trajectories through various materials under the influence of a magnetic field. The algorithm is validated using a candidate \gls{mc} simulation platform. Proton pencil beam properties, such as initial energy and starting position, are exported from a candidate treatment field and imported into the developed algorithm. The resulting pencil beam spots are compared to those initially calculated in the \gls{tps}. An optimization algorithm is then applied to account for magnetic field induced deflections, ensuring that the adjusted pencil beam spots closely align with those calculated in the absence of a magnetic field.

\chapter{Theoretical background}
\label{sec:theory}
This chapter outlines the theoretical foundations for developing a magnetic field influenced pencil beam spot algorithm. Key concepts include charged particle interactions with matter (see \autoref{sec:interactionofcharged}) and magnetic fields (see \autoref{sec:interactionofchargedmagnet}), a candidate optimization technique for shifting deflected spots (see \autoref{sec:gradientdescenttheory}), and Monte Carlo simulations (see \autoref{sec:montecarlo}). Additionally, the chapter reviews the radiotherapy workflow (see \autoref{sec:radiationtherapy}) and introduces the treatment planning system used in this thesis (see \autoref{sec:tpstheory}). Finally, the concept of a \gls{hulut} is explained (see \autoref{sec:HULUTtheory}).

\section{Interaction of charged particles with matter}
\label{sec:interactionofcharged}
In this thesis, it is assumed that only charged particles are considered. Focusing on proton therapy, protons serve as the primary charged particles under analysis. Their charge generates an electric field that increases the likelihood of interactions compared to neutral particles such as neutrons or photons. The use of charged particles leads to the emission of secondary electrons, which contributes to tissue damage. The energy loss of these particles encompasses nuclear, electronic, and radiative components \cite{krieger_wechselwirkungen_2023}.
~\\

The total stopping power $\symup{S_{tot}}$ is described with
\begin{equation}
\label{eqn:totalstoppingpower}
    S_{\symup{tot}} = \left[\left(\frac{dE}{dx}\right)_{\symup{nucl}} + \left(\frac{dE}{dx}\right)_{\symup{col}} + \left(\frac{dE}{dx}\right)_{\symup{rad}}\right] \, .
\end{equation}
At lower energy levels, nuclear energy loss becomes dominant \cite{hull_ion_2011}. However, this interaction is generally negligible in clinical contexts, as protons typically have energies in the range of \(\symup{200 \, MeV \leq E \leq 250 \, MeV}\) \cite{sengbusch_maximum_2009}.
~\\

Radiative energy loss is also negligible, as it is minimal compared to electronic energy loss \cite{hull_ion_2011}.
~\\

In the clinically relevant proton energy range, the electronic energy loss is significant \cite{krieger_wechselwirkungen_2023, hull_ion_2011} . This total energy loss results from inelastic interactions between the charged particle and the shell electrons, as well as interactions with the nucleus. Clinically, the main focus is on inelastic scattering leading to the emission of secondary electrons \cite{krieger_wechselwirkungen_2023}.
~\\

With these assumptions, the total stopping power (\autoref{eqn:totalstoppingpower}) can be approximated with
\begin{equation}
    S_{\symup{tot}} \approx \left(\frac{dE}{dx}\right)_{\symup{col}} \, .
\end{equation}
~\\
The approximate total stopping power, $S_{\symup{tot}}$, can be calculated using the Bethe Bloch formula \cite{bethe_zur_1930, bloch_bremsvermogen_1933, bloch_zur_1933}.  A simplified form of this formula is used, tailored specifically for applications in proton therapy \cite{newhauser_physics_2015}
\begin{equation}
\label{eqn:bethebloch}
    - \left(\frac{dE}{dx}\right)_{\symup{col}} \approx \Gamma \rho \frac{z_T}{A_T} \frac{Z^2}{\beta_{\symup{rel}}^2} \left[\frac{1}{2} \ln\left(\frac{2m_e c^2 \beta_{\symup{rel}}^2 W_{\symup{max}}}{I^2}\right) - \beta_{\symup{rel}}^2 - \frac{\delta}{2} - \frac{C}{z}\right] \, ,
\end{equation}
~\\where $\Gamma = 2 \pi N_A r^2_e m_e c^2$ and $W_{\symup{max}} = \frac{2 m_e c^2 \beta_{\symup{rel}}^2 \gamma^2}{1 + \frac{2 \gamma m_e}{M_0} + \left(\frac{m_e}{M_0}\right)}$. Here $\Gamma$ is a constant that includes the Avogadro constant $N_A$, the electron radius $r_e$, the electron mass $m_e$, and the speed of light $c$. The term $W_{\max }$ represents the maximum energy transfer to an electron. The properties of the target material are defined by the atomic number $z_T$, the mass number $A_T$, the density $\rho$, and the mean excitation potential $I$. This mean excitation potential $I$ is discussed further in the following chapters (see \autoref{sec:iterative}). The projectile properties include the atomic number $Z$, the mass $M_0$, the Lorentz factor $\gamma$, and the relativistic velocity $\beta_{\symup{rel}} = \frac{v}{c}$, where $v$ is the projectile velocity. The density correction term $\delta$ accounts for reduced energy loss at higher energies due to changes in the electric field and its interaction with shell electrons \cite{newhauser_physics_2015}. Finally the shell correction term $C$ becomes significant at lower energies \cite{newhauser_physics_2015}.

\newpage

\section{Interaction of charged particles with magnetic fields}
\label{sec:interactionofchargedmagnet}
Given a static, uniform electric field $\vec{E}$ and magnetic field $\vec{B}$, the trajectory of a charged particle in a vacuum is influenced by the Lorentz force $\vec{F_{\symup{L}}}$ \cite{hoffmann_proton_2015}. In the context of relativistic dynamics, the Lorentz force law for a charge $q$ can be expressed as
\begin{equation}
    \vec{F_{\symup{L}}} = \frac{\symup{d}\vec{p}}{\symup{d}t} = q \left(\vec{E} + \vec{v} \times \vec{B}\right) \, \, \, ,
\end{equation}
where the relativistic momentum is described by
\begin{equation}
    \vec{p} = m \vec{v} = \frac{m_{\symup{0}} \vec{v}}{\sqrt{1 - \frac{v^2}{c^2}}} \, \, \, ,
\end{equation}
with the speed of light $c$. For relativistic dynamics, the mass $m$ is expanded by the Lorentz factor
\begin{equation}
    \gamma = \left(1 - \frac{v^2}{c^2}\right)^{-\frac{1}{2}} \, \, \, ,
\end{equation}
relative to the rest mass $m_{\symup{0}}$. Assuming $\vec{E} = \symup{0}$ leads to a constant velocity $v$. Since $m$ and $\gamma$ are constants as well, the motion of a particle in a static uniform magnetic field can be treated as if it were non-relativistic
\begin{equation}
    \vec{F_{\symup{L}}} = m \frac{\symup{d}\vec{v}}{\symup{d}t} = q \left(\vec{v} \times \vec{B}\right) \, \, \, ,
\end{equation}
except that the particle's mass is greater than its rest mass by a factor of $\gamma$.
~\\

To derive the radius $r$ of the circular component of a charged particle's helical trajectory in a magnetic field, the balance between the centripetal force $\vec{F}_{\symup{C}}$ and the Lorentz force $\vec{F}_{\symup{L}}$ is considered. Setting $\vec{F}_{\symup{C}} = \vec{F}_{\symup{L}}$, leads to
\begin{equation}
    \frac{m v^2}{r} = q v B \, \, \, ,
\end{equation}
where $v$ is the particles' velocity relative to the magnetic field $B$. Solving for $r$ yields
\begin{equation}
\label{eqn:radiusmagnet}
    r = \frac{m v}{q B} \, \, \, .
\end{equation}
Thus, the particle follows a helical path with a radius $r$.
\newpage
The trajectory of a positively charged particle is generally demonstrated in \autoref{fig:helixtrajectory}.
\begin{figure}[!h]
    \centering
    \includegraphics[height=9cm]{../Screenshots/helix.png}
    \caption{Helical trajectory of a positively charged particle in a magnetic field with marked radius \cite{hoffmann_proton_2015}.}
    \label{fig:helixtrajectory}
\end{figure}

\section{Gradient descent optimization method}
\label{sec:gradientdescenttheory}
The gradient descent algorithm is an iterative optimization method used to minimize a differentiable function $f(\vec{x})$ \cite{boyd_convex_2004}. Starting from an initial position $\vec{x}_0$, the algorithm updates $\vec{x}$ in the direction opposite to the gradient $\nabla f(\vec{x})$, which points toward the steepest ascent. By moving against the gradient, a local minimum is approached. The update rule is given by
\begin{equation}
    \vec{x}_{i+1} = \vec{x}_i - \eta \nabla f(\vec{x}_i) \, \, \, ,
\end{equation}
where $\eta$ is the learning rate, a parameter that determines the step size. This iterative process continues until $\nabla f(\vec{x}) \approx 0$, indicating that a local minimum is reached. %The choice of \( \eta \) is critical; if too large, the algorithm may diverge, while if too small, convergence can be slow.
\newpage

\section{Monte Carlo simulations}
\label{sec:montecarlo}
\gls{mc} simulations are a computational technique used to model complex systems by simulating the random processes within them. This method leverages repeated random sampling to approximate solutions to problems that are analytically intractable, making it valuable for applications requiring high precision. In radiotherapy, \gls{mc} simulations are particularly beneficial as they enable detailed modeling of particle interactions within human tissue, accounting for the stochastic nature of radiation transport and dose deposition \cite{andreo_monte_2018}.
~\\

\gls{topas} is a \gls{mc} simulation platform designed for particle transport simulations, especially in medical physics. Built on the Geant4 toolkit, \gls{topas} enables highly detailed simulations of particle interactions with various materials, including tissue-equivalent substances used in radiotherapy and dosimetry \cite{perl_topas_2012} \cite{faddegon_topas_2020}. Configured through a set of editable parameter files, \gls{topas} supports a wide range of particle types. In this thesis, \gls{topas} serves as the ground truth for estimating proton stopping positions.
~\\

A \gls{topas} simulation is set up by a set of parameter files. These files can be edited using a plain text editor. The syntax of a parameter is defined as:
~\\
\begin{python}
	Parameter_Type : Parameter_Name = Parameter_Value # Optional comment
\end{python}
The parameter type indicates the type of value being used, such as an integer "i", a string "s", or a decimal "d" with an associated unit. Parameter names follow an object-oriented, hierarchical structure and always begin with a prefix assigned by \gls{topas} that indicates the parameter type. For example, "Ge" stands for geometric parameters, while "So" refers to particle source-related parameters. The prefix is followed by the object name for which the parameter is defined. Finally, the value of the parameter is specified, which can be either a string or a number with a unit, depending on its type. This structure provides clarity in the parameter file, making it immediately clear what type of parameter is being specified.

\newpage
\section{Radiation therapy}
\label{sec:radiationtherapy}
Radiotherapy is a medical treatment that uses high-energy radiation to destroy cancer cells, aiming to reduce or eliminate tumors while minimizing damage to surrounding healthy tissue. The radiotherapy workflow is visualized in \autoref{fig:radiotherapy}. The first step, \textbf{Consultation}, includes an initial patient evaluation to determine suitability for radiotherapy. During \textbf{Simulation}, \gls{ct} imaging is performed with the patient in the treatment position to guide subsequent planning. In the \textbf{Contouring} stage, target volumes and critical organs are delineated to define areas requiring treatment and those to be protected. The \textbf{Planning} phase involves calculating an optimal dose distribution to maximize tumor control while sparing healthy tissue. This thesis focuses on this step of radiation therapy. A candidate research \gls{tps} and further introductions into treatment planning will be discussed in the following \autoref{sec:tpstheory}. For conventional radiotherapy, \textbf{Delivery} is the administration of radiation using a \gls{linac} according to the treatment plan. Finally, \textbf{Follow-up} includes monitoring and assessing patient response to treatment to evaluate its effectiveness.
\begin{figure}[!h]
    \centering
    \includegraphics[width=\textwidth]{../Screenshots/radiotherapy.png}
    \caption{Illustration of radiotherapy workflow consisting of consultation, simulation, contouring, planning, delivery and follow-up stages \cite{marvaso_virtual_2022}.}
    \label{fig:radiotherapy}
\end{figure}
~\\When analyzing treatment planning in proton therapy, the proton range is of critical importance, as it determines the depth at which the maximum dose is delivered within tissue. Unlike conventional X-ray radiotherapy, protons deposit most of their energy at the end of their path, forming a characteristic Bragg peak in the percentage depth dose profile.
\newpage
The proton range can be approximated using the \gls{csda}, which calculates the average path length a proton travels before coming to rest. Clinically, the proton range is often defined by the $\symup{80\, \,\%}$ distal fall-off point of the Bragg peak, \( R_{\symup{80}} \), where the dose drops sharply. This is illustrated in \autoref{fig:protonddc}. It has been shown that
\begin{equation}
\label{eqn:csdar80}
    \text{CSDA} \approx R_{\text{80}} \, \, \, ,
\end{equation}
making it a useful approximation for clinical applications \cite{paganetti_proton_2018}.
\begin{figure}[!h]
    \centering
    \includegraphics[width=\textwidth]{../Screenshots/protonddc.png}
    \caption{Visualization of proton percentage depth dose curve example with marked characteristic Bragg peak as well as highlighted proton range, $R_{\symup{80}}$, at $\symup{80 \, \,\%}$ distal fall-off of the Bragg peak \cite{park_variation_2011}.}
    \label{fig:protonddc}
\end{figure}
~\\Accurate determination of the proton range is essential in proton therapy to ensure precise dose delivery to the tumor while minimizing exposure to surrounding healthy tissue.




\newpage
\section{matRad treatment planning system}
\label{sec:tpstheory}
The \gls{tps} analyzed in this thesis is the open source software matRad. This \gls{tps} supports intensity modulated photon, proton, and carbon ion therapy. It is developed for educational and research purposes only and is not intended for clinical use. The software environment is completely coded in MATLAB \cite{wieser_development_2017}. The matRad workflow is visually illustrated in \autoref{fig:matradworkflow}.
%\subsection{Workflow}
%\label{sec:matRadworkflow}
\begin{figure}[h!]
    \centering
    \includegraphics[width=\textwidth]{matradworkflow.pdf}
    \caption{Illustration of the matRad workflow consisting of six different steps.}
    \label{fig:matradworkflow}
\end{figure}
\subsubsection*{Imaging}
The first step is to import an imaging dataset, e.g. \gls{mri} or \gls{ct}. matRad allows the transformation of common imaging formats, such as \gls{dicom}, into ".mat" files for compatibility reasons. This transformation can be done by using the \gls{gui} provided by matRad. All of the other functions in the following steps can be analyzed using the matRad source code.

\subsubsection*{Objectives}
For the imported image and contours of \glspl{oar} and \gls{ctv}, objectives and constraints must be set, such as limiting the maximum dose or achieving a mean dose. In matRad, users can also define the goal's priority and set a penalty for unmet targets, with higher penalties increasing computational cost.

\subsubsection*{Treatment plan setup}
The treatment plan setup sets field values, such as the gantry and couch angle, the radiation mode, e.g. photons, carbons or protons and the \gls{rbe} calculation method. The isocenter and number of beams are also set. Furthermore, the bixel width can be set. This bixel concept is explained further in the next step.

\subsubsection*{Steering information}
All geometric information about the irradiation is stored in the "stf" structure, which uses prior steps to calculate steering information. The overall geometry follows a ray and bixel concept, illustrated in \autoref{fig:raybixelconcept}. In the steering information the end position coordinates are calculated, which is the focus of this thesis.
\begin{figure}[h!]
    \centering
    \includegraphics[height=4cm]{../Screenshots/matrad/raybixelconcept.png}
    \caption{A virtual radiation source (yellow) emits equidistant rays (solid black) to cover the target volume (red) within the patient (green). In the isocenter plane (not shown), the distance between beams equals the bixel width, representing discrete fluence elements (dashed black). The depth of the target is determined along each ray, and spots (black dots) are placed accordingly \cite{bangert_dose_nodate}.}
    \label{fig:raybixelconcept}
\end{figure}
\subsubsection*{Dose influence matrix}
A conventional pencil beam model is used for dose calculation \cite{bangert_dose_nodate}. The result of the dose calculation is a dose influence matrix, which is stored in a "dij" structure. In general, the "dij" entries are calculated row by row. This matrix has the following syntax:
\begin{align*}
        D_{ij} = \begin{pmatrix}
                    D_{11} & \cdots & D_{1j} \\
                    \vdots & \ddots & \vdots \\
                    D_{i1} & \cdots & D_{ij} \\
                 \end{pmatrix}
\end{align*}
\newpage
\subsubsection*{Optimization}
The goal of optimization is to determine bixel and spot weights that achieve the best possible dose distribution based on the clinical objectives and constraints of the radiation treatment. With the interior optimization algorithm the dose can be calculated with 
\begin{equation}
    d_i = \sum_j = D_{ij} \cdot w_j \, \, \, \, ,
\end{equation} 
where $D_{ij}$ is the dose influence matrix and $w_j$ is the weighting factor for each bixel $j$ resulting from the optimization algorithm with consideration of specified objectives and constraints. This step is particularly important for creating treatment plans that provide approximate clinically relevant \gls{dvh} results. Finally, matRad offers a visualization of the resulting treatment plan with its dose distribution, as well as a printout of a \gls{dvh}.

\section{Hounsfield look-up table}
\label{sec:HULUTtheory}
A \gls{hulut} is used to convert \gls{ct} numbers into corresponding \gls{rspr} values. Each \gls{ct} number reflects the attenuation properties of a specific tissue relative to water, allowing differentiation between various tissue types. A $n \times m$ \gls{rspr} map is generated from an $n \times m$ \gls{ct} number map, representing a \gls{ct} slice. This approach approximates the stopping power of different tissues by relating it to the stopping power in water through the \gls{rspr} map. With the help of linear interpolation, the material assignment based on these conversions is crucial in simulations and treatment planning, especially in radiotherapy, where precise tissue characterization directly impacts dose calculations and treatment accuracy. A candidate \gls{hulut} conversion curve is illustrated in \autoref{fig:hlutplot}.
\begin{figure}[h!]
    \centering
    \includegraphics[width=10cm]{../Screenshots/hlut.jpg}
    \caption{Visualization of candidate \gls{hulut} conversion. Modified from \cite{peters_consensus_2023}.}
    \label{fig:hlutplot}
\end{figure}
\chapter{Materials and methods}
\label{sec:materials}
To integrate a magnetic field influenced pencil beam spot selection algorithm for a given patient dataset (see \autoref{sec:patientdataset}) and treatment plan setup (see \autoref{sec:treatmentplansetup}), analytical calculations of the proton stopping position are discussed (see \autoref{sec:analyticalmethod}). To validate these calculations, a comparison is made with proton stopping positions determined in a \gls{mc} environment (\gls{topas}) (see \autoref{sec:TOPAS}). Finally, an optimization algorithm for magnetic field influenced pencil beam spot selection is introduced (see \autoref{sec:gradientdescentmaterial}).
%To achieve the goal of integrating a magnetic field influenced pencil beam spot selection algorithm for a candidate \gls{tps}, the developed proton trajectory calculation is discussed (see \autoref{sec:iterative}). Its general workflow and settings are explained (see \autoref{sec:MATLABworkflow}). To verify the results of this proton trajectory calculation, a candidate \gls{mc} environment (\gls{topas}) is presented for comparison. The settings used for the \gls{mc} simulation in \gls{topas} are demonstrated (see \autoref{sec:TOPAS}). In addition, two candidate methods for verifying the proton trajectory calculation with \gls{topas} are described (see \autoref{sec:topasB0T} and \autoref{sec:topasBGT}). Finally, matRad, which is the \gls{tps} analyzed in this work, is demonstrated (see \autoref{sec:matRad}). The general workflow (see \autoref{sec:matRadworkflow}) and the settings used (see \autoref{sec:treatmentplansetup}) are explained. To compare the proton trajectory calculation with matRad, the method of exporting matRad variables is also studied (see \autoref{sec:exportingmatradvariables}). Finally, for the introduction of a magnetic field, a possible optimization of the matRad pencil beam spot selection is introduced (see \autoref{sec:gradientdescentmaterial}).
%This chapter focuses on explaining how the proton beam trajectory is calculated (see \autoref{sec:iterative}), for which a schematic workflow is given (see \autoref{sec:MATLABworkflow}). Then, the implementation of the gradient descent algorithm is discussed (see \autoref{sec:gradientdescentmaterial}). Furthermore, the open source software matRad will be generally demonstrated (see \autoref{sec:matRad}) with the help of a simplified workflow scheme (see \autoref{sec:matRadworkflow}) and a basic explanation of its' treatment plan setup (see \autoref{sec:treatmentplansetup}). Finally, a candidate method of a magnetic field introduction for matRad will be shown (see \autoref{sec:exportingbeamvariables}).
\section{Patient dataset}
\label{sec:patientdataset}
The patient dataset used in this thesis is publicly available as part of the Gold Atlas project \cite{nyholm_mr_2018}. This project includes data from 19 patients, each with defined anatomical structures and delineation details. The delineations were independently performed in RayStation v4.7.2 via remote connections by four experienced radiation oncologists and one radiologist. Each patient’s imaging data is stored in \gls{dicom} format.
~\\

For this thesis, the dataset labeled "2\_04\_P" (patient ID 9) is used as an example. This dataset includes five structures: the urinary bladder, rectum, both femoral heads, and prostate. The \gls{ct} images were acquired using a Toshiba Aquilion \gls{ct} scanner.
~\\

The voxel dimensions are \((1.09375 \times 1.03975 \times 2) \, \symup{mm^3}\), with an overall resolution of \(341 \times 225 \times 62\). Of the 62 slices available, the slice labeled "IMG0036" and its pixel data is presented for explanation in this thesis.
~\\

Throughout this thesis, this analyzed dataset will be referred to as the prostate patient dataset.

\newpage
\section{Treatment plan setup}
\label{sec:treatmentplansetup}
%\label{sec:matRad}
%matRad is an open source radiation treatment planning software supporting intensity modulated photon, proton, and carbon ion therapy. It is developed for educational and research purposes only and is not intended for clinical use. The software environment is completely coded in MATLAB \cite{wieser_development_2017}. In this work, the feature of proton therapy is focused on.
%~\\
%
%This section deals with the general matRad workflow, which consists of six different steps. These are explain in \autoref{sec:matRadworkflow}. After that, the settings used for these steps are described, which can be seen in \autoref{sec:treatmentplansetup}.
%\subsection{Workflow}
%\label{sec:matRadworkflow}
%The matRad workflow is visually illustrated in \autoref{fig:matradworkflow}.
%\begin{figure}[h!]
%    \centering
%    \includegraphics[width=\textwidth]{matradworkflow.pdf}
%    \caption{Illustration of the matRad workflow consisting of six different steps.}
%    \label{fig:matradworkflow}
%\end{figure}
%\subsubsection*{Imaging}
%The first step is to import an imaging dataset, e.g. \gls{mri} or \gls{ct}. matRad allows the transformation of common imaging formats, such as \gls{dicom}, into ".mat" files for compatibility reasons. This transformation can be done by using the \gls{gui} provided by matRad. All of the other functions in the following steps can be used from the source code.
%
%\subsubsection*{Objectives}
%For the imported image and contours of \glspl{oar} and \gls{ctv}, objectives and constraints must be set, such as limiting the maximum dose or achieving a mean dose. In matRad, users can also define the goal's priority and set a penalty for unmet targets, with higher penalties increasing computational cost.
%
%\subsubsection*{Treatment plan setup}
%The treatment plan setup sets field values, such as the gantry and couch angle, the radiation mode, e.g. photons, carbons or protons and the \gls{rbe} calculation method. The isocenter and number of beams are also set. Furthermore, the bixel width will be set. This bixel concept is explained further in the next step.
%
%\subsubsection*{Steering information}
%All geometric information about the irradiation is stored in the "stf" structure, which uses prior steps to calculate steering information. The overall geometry follows a ray and bixel concept, illustrated in \autoref{fig:raybixelconcept}. In the steering information the end position coordinates are calculated, which is the focus of this work.
%\begin{figure}[h!]
%    \centering
%    \includegraphics[width=\textwidth]{../Screenshots/matrad/raybixelconcept.png}
%    \caption{A virtual radiation source (yellow) emits equidistant rays (solid black) to cover the target volume (red) within the patient (green). In the isocenter plane (not shown), the distance between beams equals the bixel width, representing discrete fluence elements (dashed black). The depth of the target is determined along each ray, and spots (black dots) are placed accordingly \cite{bangert_dose_nodate}.}
%    \label{fig:raybixelconcept}
%\end{figure}
%
%\subsubsection*{Dose influence matrix}
%A conventional pencil beam model is used for dose calculation. A detailed explanation can be found in the literature \cite{bangert_dose_nodate}. The result of the dose calculation is a dose influence matrix, which is stored in a "dij" structure. In general, the "dij" entries are calculated row by row. This matrix has the following syntax:
%\begin{align*}
%        D_{ij} = \begin{pmatrix}
%                    D_{11} & \cdots & D_{1j} \\
%                    \vdots & \ddots & \vdots \\
%                    D_{i1} & \cdots & D_{ij} \\
%                 \end{pmatrix}
%\end{align*}
%
%\subsubsection*{Optimization}
%The goal of optimization is to determine bixel and spot weights that achieve the best possible dose distribution based on the clinical objectives and constraints of the radiation treatment. This step is particularly important for creating treatment plans that provide approximate clinically relevant \gls{dvh} results.
%~\\
%
%Finally, matRad offers a visualization of the resulting treatment plan with its' dose distribution, as well as printing out a \gls{dvh}.
%\newpage
A treatment plan is made for the given prostate patient dataset consisting of three sample slices with the same pixel data, i.e. equal CT numbers, for which one slice is visualized in the following \autoref{fig:patient}.
\begin{figure}[h!]
    \centering
    \includegraphics[width=\textwidth]{../Screenshots/matrad/patient.png}
    \caption{Given prostate patient \gls{ct} dataset imported into matRad and visualized using the matRad \gls{gui}. A candidate \gls{ctv}, representing the prostate, is highlighted (red) as well as possible \glspl{oar}, representing both femoral heads and the rectum (pink and dark red).}
    \label{fig:patient}
\end{figure}
~\\As for objectives, the prostate is set as the only target with the goal of achieving a mean dose of $\symup{50 \, \, Gy}$ and a default priority and penalty of 1. Further objectives and constraints for the \glspl{oar} are not set, since this work does not focus on a clinically relevant result, but on the investigation of the pencil beam spot selection.
~\\

Radiation is delivered from a couch angle of 0° and a gantry angle of 270°. Protons are used, assuming a constant RBE of 1.1. The default pencil beam width of $\symup{5 \, \, mm}$ is used. For patient materials, the matRad \gls{hulut} approach is used.  The dose grid resolution is set equal to the \gls{ct} grid resolution of $\symup{(1.09375, 1.09375) \, \, mm^2}$ for x and y. The slices show a spacing of $\symup{z = 2 \, \, mm}$.
~\\
%When generating the steering information in "matRad\_generateStf.m", the variable containing information about the targets' position "voiTarget" is exported for further pencil beam spot selection analysis.
%~\\

The analytical absorbed dose is calculated using the matRad default pencil beam model. Finally, the default fluence optimization algorithm is applied.
~\\

This matRad treatment plan simulation is documented and explained in more detail in the appendix (\autoref{sec:matradtreatmentplanappendix}).

%In the step of dose influence matrix calculation, linear interpolation is used to export $\symup{R_{80}}$ ranges of each beam as well as the initial starting position and energy.

\section{Analytical calculation of proton stopping position}
\label{sec:analyticalmethod}
\subsection{Developed proton transfer algorithm}
\label{sec:iterative}
%WRITE THIS IN DISCUSSION MAYBE? The goal is to optimize a magnetic field influenced pencil beam spot selection in the candidate treatment planning system matRad. For this purpose, the development of a proton trajectory calculation in MATLAB is advantageous, since this is the environment in which matRad was developed.% The general workflow of the developed algorithm is discussed in the following \autoref{sec:MATLABworkflow}.
%\subsection{Workflow}
%\label{sec:MATLABworkflow}
In this section, an analytical method is proposed to provide a rough estimation of the expected proton stopping position in the presence of a magnetic field.
~\\

The CT slice is assumed to be perpendicular to the magnetic field ($\vec{B} = B_z$). Since the goal is to estimate the approximate stopping location of the proton, random effects such as particle collisions and scattering are not calculated in detail. Instead, the simulation focuses on the general effects of energy loss and deflection due to the magnetic field.
~\\

The simulation of the expected proton trajectory is calculated in a step-wise manner, with two steps per voxel. In each step, the state of a given incident proton is described by its position $\vec{r}$, velocity $\vec{v}$, magnetic field $\vec{B}$, and material-specific stopping power (\gls{rspr}), determined by the CT number.
%The implementation of energy loss calculation using \gls{csda} is inspired by "libamtrack" \cite{grzanka_libamtrack_nodate}. This implementation allows to calculate the energy loss up to $E\symup{ = 0.49 \, \, MeV}$. When this energy value is reached, the calculation stops. In Grzanka et al. the mean excitation value, for the Bethe Bloch equation explained in \autoref{eqn:bethebloch}, is $I\symup{ = 75 \, \, eV}$. This was updated with the \gls{icru} report 90 to $I\symup{ = 78 \, \, eV}$ \cite{czarnecki_impact_2018}, which is also used in \gls{topas} \cite{perl_30_nodate}. Thus, the developed calculation for the proton stopping position in this thesis will use the same mean excitation value $I\symup{ = 78 \, \, eV}$.
~\\

The initialization of a given simulation is conducted by:
%The first code performs the calculation of force, acceleration, velocity, position and energy loss. A step-wise approach is used to iteratively compute all of the above with a convergence check as the last iteration step. The step size of is set to $\symup{1.09375 \, \, mm}$, which corresponds to the voxel step size of the \gls{ct} datasets analyzed in the following chapters. These aspects are illustrated in \autoref{fig:matlabworkfloww}.
%~\\
%\begin{figure}[h!]
%    \centering
%    \includegraphics[width=\textwidth]{figures.pdf}
%    \caption{(A) Demonstration of iterative workflow for MATLAB proton trajectory calculation. (B) MATLAB voxel is graphically illustrated with its' size labeled. For each voxel the calculation steps are calculated twice (blue and green line).}
%    \label{fig:matlabworkfloww}
%\end{figure}
%~\\
%The second code references and initializes the first. The purpose of this second code is to start a simplified simulation with only the input of varying parameters, e.g. initial energy, initial position, \gls{rspr} map and magnetic field strength. When comparing these results with \gls{topas}, the Schneider \gls{hulut} is used, while for the comparison with matRad, the matRad implemented \gls{hulut} is used.
%
%\newpage
%The calculations for the implemented algorithm are realized by first initializing the first step of velocity and acceleration with
\begin{align*}
    \gamma_0       =& \, \, 1 + \frac{E_0}{m_u} \\
    \beta_0        =& \, \, \sqrt{1 - \frac{1}{\gamma^2_0}} \\
    \vec{v}_{0}    =& \, \, \beta_0 \cdot c \\
    \vec{F}_{0}    =& - q \cdot \left(\vec{v}_{0} \times \vec{B_0}\right) \\
    \vec{a}_{0}    =& \, \frac{\vec{F}_{0}}{m_{\symup{p}}} \, \, \, \, \, ,
\end{align*}
~\\where $\gamma_0$ is the Lorentz factor, $E_0$ is the initial energy in MeV, $m_u$ is the atomic mass unit, $\beta_0$ is the relativistic velocity, $c$ is the light of speed and $\vec{v}_0$ is the initial velocity of the proton. Then, the initial Lorentz force $\vec{F}_0$ is computed with the charge of the proton and magnetic field strength $\vec{B}_0$. Finally, the initial acceleration of the proton $\vec{a}_0$ can be described with the initial lorentz force and mass of the proton $m_p$.
~\\
Given the status of the proton at step $i-1$ ($\vec{r_{i-1}}, \vec{v_{i-1}}$, $\vec{B_i}$, \gls{rspr}$_i$), the proton status at step $i$ can be calculated iteratively with
\begin{align*}
    %\symup{dt}_{i} =& \, \frac{\symup{\increment x}}{\symup{\lVert \vec{v}_{i-1}\rVert}} \\
    \vec{F}_{i}    =&  - q \cdot \left(\vec{v}_{i-1} \times \vec{B_i}\right) \\
    \vec{a}_{i}    =& \, \frac{\vec{F}_{i}}{m_{\symup{p}}} \\
    \vec{v}_{i}    =& \, \, \vec{v}_{i-1} + \vec{a}_{i} \cdot \symup{dt}_{i} \\
    \vec{r}_{i}    =& \, \, \vec{r}_{i-1} + \vec{v}_{i} \cdot \symup{dt}_{i} \, \, \, \, \, ,
\end{align*}
where the deflection, due to a magnetic field $\vec{B}_i$, is calculating using the Lorentz force $\vec{F}_i$ with the charge of the proton $q$, the norm of the prior velocity status $\vec{v}_{i-1}$. With the Lorentz force and the mass of the proton $m_{\symup{p}}$, the updated deflection $\vec{a}_i$ can be computed. The next step is calculating the next velocity step $\vec{v}_i$ using the updated acceleration, time step and prior velocity step $\vec{v}_{i-1}$. Finally, the updated proton position $\vec{r}_i$ is computed using the prior position step $\vec{r}_{i-1}$, updated velocity and time step. The time step itself is calculated by
\begin{align*}
    \symup{dt}_{i} =& \, \frac{\symup{\increment x}}{\symup{\lVert \vec{v}_{i-1,x}\rVert}} \, \, \, \, \, ,
\end{align*}
where $\symup{dt}_{i}$ is the time step, $\symup{\increment x}$ is half of the grid step size and $\|\vec{v}_{i-1,x}\|$ is the norm of the prior calculated velocity in x-direction.
~\\

The update of the velocity magnitude depends on the energy of the given proton status. Energy loss is calculated using the \gls{csda}. The implementation of energy loss calculation using \gls{csda} is inspired by "libamtrack" \cite{grzanka_libamtrack_nodate}. This implementation allows to calculate the energy loss up to $E\symup{ = 0.49 \, \, MeV}$. Lower energy values are neglected. Thus, this energy is set as the energy threshold of the proton stopping position calculation in this thesis. The initial mean excitation value for Bethe Bloch (\autoref{eqn:bethebloch}) is $I\symup{ = 75 \, \, eV}$. This was updated with the \gls{icru} report 90 to $I\symup{ = 78 \, \, eV}$ \cite{czarnecki_impact_2018}, which is also used in \gls{topas} \cite{perl_30_nodate}. Thus, the developed proton transfer algorithm is set to have an equal mean excitation value of $I\symup{ = 78 \, \, eV}$.
~\\

In order to calculate proton energy loss, a \gls{rspr} map is imported. With the help of
\begin{align*}
    \increment E_{i,\symup{SPR}} =& \, \, \increment E_{i} \cdot \symup{SPR}_i \\
    E_{i}                        =& \, \, \increment E_{i,\symup{SPR}} \cdot \increment x \, \, \, \, \, ,
\end{align*}
where $\increment E_{i,\symup{SPR}}$ is the energy loss with \gls{rspr} map value applied relative to the current proton position, $\increment E_{i}$ is the energy loss calculated using the "libamtrack" implementation \cite{grzanka_libamtrack_nodate} and $\symup{SPR}_i$ is the \gls{rspr} map value relative to the current proton position. Finally, the energy is updated using the updated energy loss and step size $\increment x$.
~\\
A detailed documentation of this developed algorithm can be found in the appendix (\autoref{sec:protontrajectoryappendix}).
~\\

To verify the developed algorithm of proton transfer, cases of increasing geometric complexity are analyzed. First, the helical radius (\autoref{eqn:radiusmagnet}) is verified via a simulation in a vacuum. For this simulation an energy loss of $\symup{\frac{\mathrm{d}E}{\mathrm{d}x} = 10^{-14} \, \, \frac{\mathrm{keV}}{\mu \mathrm{m}}}$ is assumed. Next, simulations are conducted in water (CT number: $\symup{0 \, \, HU}$ \cite{denotter_hounsfield_2024}), bone (CT number: $\symup{1000 \, \, HU}$ \cite{denotter_hounsfield_2024}), and prostate patient phantoms (CT number: on patient CT) with and without a magnetic field. The voxel dimensions and overall resolutions of all phantoms are set equal to the patient dataset (see \autoref{sec:patientdataset}). Each of these cases is compared to \gls{mc} based calculations of proton stopping positions under matching conditions, including initial energy, beam source position, and the same \gls{hulut}. The specific \gls{hulut} used in verification is the "Schneider" \gls{hulut} \cite{schneider_correlation_2000}.

\subsection{matRad}
\label{sec:matRad}
The dose influence matrix $D_{ij}$, generated during matRad treatment planning according to given objectives and constraints, is analyzed to obtain the planned proton pencil beam stopping position, which is determined by the distal fall-off at $\symup{80 \, \,\%}$ of the relative maximum dose $R_{\symup{80}}$ on the laterally integrated dose distribution. With the stopping positions of all pencil beams used in the initial treatment plan, a spot grid, that is supposed to fill a given \gls{ctv}, can be achieved. The \gls{ctv} is visualized by exporting the variable "voiTarget" inside the steering information in $\symup{"matRad\_generateStf.m"}$ is exported. This analysis is documented and can be seen in the appendix (\autoref{sec:exporting_matRad_R80_positionsappendix}).
~\\
To further analyze the developed algorithm, initial positions and energy values for each bixel in the treatment field are exported and input into the algorithm code with the matRad \gls{hulut}. The resulting spots are then compared. Finally, a magnetic field is introduced to assess resulting differences, and deflected spots are optimized.
\newpage
\section{Monte Carlo based calculation of proton stopping position}
\label{sec:TOPAS}
The parameter setup used in this thesis and two candidate methods for verifying the analytical calculation of the proton stopping position for both $\symup{B = 0 \, \, T}$ and $\symup{B > 0 \, \, T}$ are described in \autoref{sec:parametersetup}, \autoref{sec:topasB0T} and \autoref{sec:topasBGT} respectively.
\subsection{Simulation setup}
\label{sec:parametersetup}
The default physics settings are applied for all \gls{topas} simulations in this thesis.

~\\Additionally, the beam model is configured with an energy spread of 0, a "Flat" position distribution, and an "Ellipse" cutoff shape. The cutoffs in both the x- and y-directions are set to \(\symup{0.5 \, cm}\), and the position spread is set to \(\symup{0.1 \, mm}\). No angular distribution is applied. All simulations use the default random seed and a constant initial proton count of \(\symup{N = 10^5}\). The beam is positioned at the left, center-aligned starting voxel of the \gls{ct} dataset, as illustrated in \autoref{fig:parametersetup}. Detailed simulation settings and magnetic field configurations are provided in \autoref{sec:appendixtopassetup}.
~\\
~\\\begin{figure}[h!]
    \centering
    \includegraphics[height=9cm]{../Screenshots/examples/parametersetup.png}
    \caption{Visualization of the beam start position in \gls{topas}. In this example, a prostate patient \gls{ct} dataset is irradiated with protons (E = 100 MeV, N$ \,\, = \symup{10^3}$). Primary particles, i.e. protons are marked (blue lines), as well as secondary particles, i.e. electrons (red lines) and gamma rays (green lines).}
    \label{fig:parametersetup}
\end{figure}
~\\
Intuitively, each voxel in a \gls{ct} slice has a different \gls{ct} number. For all simulations, the "Schneider" conversion method is used to assign \gls{ct} numbers to tissue parameters. The "Schneider" conversion method is implemented with the following lines:
~\\
\begin{python}
    includeFile = HUtoMaterialSchneider.txt
    s:Ge/Patient/ImagingtoMaterialConverter = "Schneider"
\end{python}
For absorbed dose measurements, a volume scorer is used. This scorer divides a geometric volume into symmetrical bins. Imported \gls{ct} datasets for different materials are divided into multiple voxels. Each voxel contains the absorbed dose value for the corresponding bin. The bin size for all dimensions is set equivalent to the default \gls{ct} grid size.
\newpage
\subsection{Estimation of proton stopping position for B = 0 T}
\label{sec:topasB0T}
In the absence of a magnetic field, no deflection occurs in proton trajectories. The $\symup{80 \, \,\%}$ distal fall-off of the laterally integrated proton dose distribution is taken as the stopping position, consistent with the proton stopping calculation used in matRad (see \autoref{sec:matRad}).
\subsection{Estimation of proton stopping position for B > 0 T}
\label{sec:topasBGT}
To estimate the proton stopping position in a magnetic field, the following method is proposed. First, the \gls{3D} dose distribution \( D(i,j,k) \) is first simplified to a simple proton trajectory curve, \( y(x) \), by applying a Gaussian fit to the \gls{2D} dose cross section at each depth \( x \), i.e.,
\begin{equation}
    y(x) = \symup{max}\Biggl(\symup{Gaussian}\Bigl(D\left(x, :, :\right)\Bigr)\Biggr) \, \, \, ,
\end{equation}
where $y(x)$ is regarded as the expected trajectory of the corresponding magnetic field influenced deflected proton pencil beam (see \autoref{fig:plottogether}). 
\\\begin{figure}[h!]
    \centering
    \includegraphics[width=\textwidth]{../Screenshots/plottogether.png}
    \caption{Visualization of magnetic field influenced proton trajectory simulated in \gls{topas} (left). A candidate value of $x = 40 \, \, \symup{mm}$ is marked and the \gls{1D} dose distribution along $y$ with Gaussian fit for specified $x$ value is demonstrated (right).}
    \label{fig:plottogether}
\end{figure}
~\\Once the proton trajectory is determined, the proton stopping position is calculated using the following method. It is assumed that the path length a proton travels is primarily governed by stopping power, which is minimally influenced by the deflection induced by a magnetic field. Therefore, the trajectory lengths of protons with and without a magnetic field are considered approximately equivalent. The proton stopping position is determined by equating the path length of a proton traveling along the deflected trajectory in a magnetic field to the path length of the same proton pencil beam in the absence of a magnetic field. The stopping position is then calculated as follows: First, the trajectory length of \( P_2 \), denoted \( s_{\symup{L2}} \), is determined based on the 80 \% distal fall-off of the corresponding laterally integrated dose distribution. Subsequently, the trajectory length of \( P_1 \), denoted \( s_{\symup{L1}} \), is calculated by
\begin{align}
\label{eqn:sdistance}
    s_{\symup{L1}} = \sum_i \sqrt{\increment x_i^2 + \increment y_i^2} \, \, \, \, \text{with} \, \, \, \, \increment x_i = x_i - x_{i-1} \, \land \, \increment y_i = y_i - x_{i-1} \, \, \, ,
\end{align}
until step $i$ for which $s_{\symup{L1}} = s_{\symup{L2}}$. Finally, $(x_i,\, y_i)$ is regarded as the proton stopping position of $P_2$. This approximation is visualized in \autoref{fig:sdistance}. A detailed code implementation can be found in appendix (\autoref{sec:topasb0tappendix}).
~\\

The results of the \gls{mc} based calculation of proton stopping position are assumed to be the ground truth for verifying the analytically calculated proton stopping position.
\begin{figure}[h!]
    \centering
    \includegraphics[width=13cm]{sdistance.pdf}
    \caption{Illustration of magnetic field influenced proton beam trajectory approximation. $P_1$ refers to the magnetic field influenced pencil beam.}
    \label{fig:sdistance}
\end{figure}
\section{Optimization for magnetic field introduction}
\label{sec:gradientdescentmaterial}
The purpose of this section is to present a method for modifying and adapting the pencil beam spots of the initial treatment field considering the impact of the influence of an additional given magnetic field, ensuring that the modified spots provide approximately the same coverage of the \gls{ctv} as the original spots without a magnetic field.
~\\

First, for a given pencil beam $i$ in the initial treatment field with initial position $\vec{r}_i$ and energy $E_i$, the proton stopping position without a magnetic field, $\vec{V}_{i}$, is calculated using the developed algorithm (see \autoref{sec:iterative}). Next, for the same pencil beam, the proton stopping position in the presence of the given magnetic field, $\vec{V}_{{\symup{M}},i}$, is determined. Finally, the updated new position $\vec{r}_{\symup{op,i}}$ and energy $E_{\symup{op,i}}$ is obtained via gradient descent optimization (see \autoref{sec:gradientdescenttheory}) by minimizing the difference between $\vec{V}_{{\symup{M}},i}$ and $\vec{V}_{i}$. Thus, $\vec{r}_{\symup{op},i} \approx \vec{V}_{i}$ is assumed. This is visualized in \autoref{fig:gradientplot}.

\newpage
The gradient descent parameters used are:
%The \gls{rspr} map for given \gls{ct} datasets is imported and the proton trajectory calculation (see \autoref{sec:iterative}) is used to compute each gradient descent step. The gradient descent starts at the position shifted by the magnetic field, aiming to minimize the trajectory to converge towards the position without a magnetic field, using the same initial position and energy. The gradient descent parameters are:
\begin{python}
    learning_rate_E = 0.0005; % Step size for energy
    learning_rate_y = 0.0005; % Step size for position (if y is adjusted)
    tolerance = 1e-4;         % Convergence tolerance
    max_steps = 200;          % Maximum iterations
\end{python}
A detailed documentation of the computational gradient descent implementation can be seen in the appendix (\autoref{sec:gradientdescentoptimization}).
\begin{figure}[h!]
    \centering
    \includegraphics[height=12.8cm]{gradientplot.pdf}
    \caption{Illustration of gradient descent optimization. P1 marks the magnetic field influenced pencil beam with proton stopping position $\vec{V}_{\symup{M},i}$. P2 refers to the pencil beam without a magnetic field with initial position $\vec{r}_i$ and proton stopping position $\vec{V}_i$. After gradient descent optimization, the pencil beam $P_{\symup{op},1}$ stopping position $\vec{r}_{\symup{op},i}$ is marked.}
    \label{fig:gradientplot}
\end{figure}
\chapter{Results}
\label{sec:Results}
This chapter presents the results of the developed algorithm across cases of increasing complexity, verified against ground truth references and previously discussed methods (see \autoref{sec:VerifyBeam}). The analysis then extends to the initial results of a candidate treatment field using the \gls{tps}, comparing it to the developed algorithm, and exploring pencil beam spot optimization under the influence of a magnetic field (see \autoref{sec:matRadPlan}).
\section{Verification of proton stopping position}
\label{sec:VerifyBeam}
\subsection{Vacuum state}
\label{sec:vacuum}
%Before verifying the energy loss calculations for materials of varying complexity, the influence of the magnetic field is examined. For this investigation, a fixed energy value is analyzed under four different magnetic field strengths, and vice versa. An illustration of an example proton beam trajectory calculated under influence of a magnetic field using the developed algorithm is visualized in \autoref{fig:vacuumexample}. For simplicity, this developed algorithm will be referred to as the MATLAB code for the remainder of this thesis.
%\\
%
The comparison between radii calculated using the developed proton transfer algorithm ($r_{\symup{ana}}$) and analytical calculation using \autoref{eqn:radiusmagnet} ($r_{\symup{eq}}$) is listed in table \ref{tab:compareMATLABanalytical}.
~\\

In a vacuum, with a fixed magnetic field strength of \(\symup{1.5 \, \, T}\), relative differences remain positive and converge around an average of $\symup{0.3075 \, \,\%}$. Additionally, relative differences decrease as energy increases. For a set energy of \(\symup{100 \, \, MeV}\), analysis across four different magnetic field strengths shows that positive relative differences continue, converging around $\symup{0.3275 \, \, \%}$. Varying magnetic field strength thus appears to produce a stable relative difference. Absolute differences slightly increase with higher energy values but decrease with increasing magnetic field strength.
%\begin{figure}[h]
%    \begin{minipage}{0.5\textwidth}
%        \centering
%        \includegraphics[width=8cm]{../Screenshots/vacuum/VacuumB15E50MeV.png}
%        \label{fig:VacuumB15E50MeV}
%    \end{minipage}
%    \hfill
%    \begin{minipage}[c]{0.5\textwidth}
%        \centering
%        \includegraphics[width=8cm]{../Screenshots/vacuum/VacuumB3E100MeV.png}
%        \label{fig:VacuumB3E100MeV}
%    \end{minipage}
%    \caption{The left figure is ight figure.}
%    \label{fig:VacuumExample}
%\end{figure}
\begin{figure}[h!]
    \centering
    \includegraphics[height=7cm]{../Screenshots/vacuum/VacuumB15E50MeV.png}
    \caption{Example proton trajectory in a vacuum under influence of a magnetic field with $E = \symup{50 \, \, MeV}$ and $B_z = \symup{1.5 \, \,T}$ resulting from MATLAB simulation. Highlighted position (red dot) marks the diameter of the circular trajectory.}
    \label{fig:vacuumexample}
\end{figure}
\begin{table}[h!]
    \centering
    \caption{Comparison of resulting radii calculated using the developed proton transfer algorithm ($r_{\symup{ana}}$) and analytical equation for the helical radius ($r_{\symup{eq}}$) under influence of a magnetic field ($B_z$) in a vacuum state. The first table has a fixed magnetic field strength with four different energy values, while the second table shows the results for a fixed energy value and four different magnetic field strengths. Absolute and relative differences between those radii are also listed.}
    \label{tab:compareMATLABanalytical}
    \begin{tabular}{c c c c c}
        
        \multicolumn{5}{c}{\textbf{B = 1.5 T}} \\
        \hline
        E [MeV] & $r_{\symup{ana}}$ [cm] & $r_{\symup{eq}}$ [cm] & Abs. Difference [cm] & Rel. Difference [\%] \\
        \hline
        50  & 65.76 & 65.53 & 0.23 & 0.35 \\
        
        100 & 89.63 & 89.35 & 0.28 & 0.31 \\
        
        150 & 106.03 & 105.71 & 0.32 & 0.30 \\
        
        200 & 118.46 & 118.14 & 0.32 & 0.27 \\
        \hline
    \end{tabular}
    
    \vspace{1cm}

    \begin{tabular}{c c c c c}
        
        \multicolumn{5}{c}{\textbf{E = 100 MeV}} \\
        \hline
        B [T] & $r_{\symup{ana}}$ [cm] & $r_{\symup{eq}}$ [cm] & Abs. Difference [cm] & Rel. Difference [\%] \\
        \hline
        0.5 & 268.89 & 268.04 & 0.85 &  0.32 \\
        
        1   & 134.45 & 134.02 & 0.43 & 0.32 \\
        
        2   & 67.23  & 67.01  & 0.22 & 0.33 \\
        
        3   & 44.82  & 44.67  & 0.15 & 0.34 \\
        \hline
    \end{tabular}
\end{table}
\newpage

\subsection{Water phantom}
\label{sec:waterphantom}
\subsubsection{B = 0 T}
%~\\As explained in section \ref{sec:TOPAS}, \gls{topas} will be used as a reference for given MATLAB results. Both simulations are intuitively carried out under the same conditions. These conditions include, for example, the inital energy, the starting position of our beam source and the same \gls{hulut}.
An example of proton beam trajectories calculated using the developed proton transfer algorithm and \gls{mc} simulations is visualized in \autoref{fig:waterb0e200mev}. For the \gls{mc} simulated pencil beam, the laterally integrated dose and corresponding proton range, \( R_{\symup{80}} \), are shown in \autoref{fig:waterb0e200mevddc}. A comparison of proton ranges calculated using the developed proton transfer algorithm ($r_{\symup{ana}}$) and those extracted from \gls{mc} simulations ($r_{\symup{MC}}$) for proton beams with different energies in a water phantom without a magnetic field is listed in \autoref{tab:MATLABTOPASwaterB0}.
\begin{figure}[h!]
    \centering
    \includegraphics[height=9cm]{../Screenshots/water/Water_B0_E200MeV.png}
    \caption{Resulting proton beam trajectories from the developed algorithm and \gls{mc} simulations (\gls{topas}) in water for $E = \symup{200 \, \,MeV}$ and $B_z = \symup{0 \, \,T}$. The white line shows the trajectory calculated in the developed algorithm, while the heatmap dose distribution refers to the \gls{mc} simulated proton beam trajectory.}
    \label{fig:waterb0e200mev}
\end{figure}
~\\
Qualitatively, the proton trajectory from a \gls{mc} simulation shows a dimensional spread, contrasting with the simpler trajectory from the developed proton transfer algorithm. Furthermore, the stopping position calculated using the developed algorithm occurs shortly after the maximum relative dose observed in the \gls{mc} based calculation.
\begin{figure}[!h]
    \centering
    \includegraphics[height=9.5cm]{../Screenshots/water/Water_B0_E200MeV_ddc.png}
    \caption{Resulting percentage depth dose curve with marked proton range, $\symup{R_{80}}$, from a \gls{mc} simulation (\gls{topas}) in water for $E = 200\symup{\, \, MeV}$ and $B_z = 0\symup{\, \,T}$ with $N = 10^5$ protons.}
    \label{fig:waterb0e200mevddc}
\end{figure}
%Results for MATLAB \gls{csda} ranges ($r_{\symup{ana}}$) and \gls{topas} $\symup{R_{80}}$ ranges ($r_{\symup{MC}}$), with their differences, for four different energy values are listed in the following table \ref{tab:MATLABTOPASwaterB0}.
\begin{table}[h!]
    \centering
    \caption{Comparison of proton ranges calculated analytically using the developed proton transfer algorithm ($r_{\symup{ana}}$) and proton ranges ranges extracted from \gls{mc} simulation (\gls{topas}) depth dose curves ($r_{\symup{MC}}$) for a water phantom without a magnetic field. Four different energy values are analyzed and the resulting relative and absolute differences between the two ranges can be seen.}
    \label{tab:MATLABTOPASwaterB0}
    \begin{tabular}{c c c c c}
        
        \multicolumn{5}{c}{\textbf{B = 0 T}} \\
        \hline
        E [MeV] & $r_{\symup{ana}}$ [mm] & $r_{\symup{MC}}$ [mm] & Abs. Difference [mm] & Rel. Difference [\%] \\
        \hline
        50  & 22.19 & 21.26 & 0.93 & 4.37 \\
        
        100 & 76.92 & 76.28 & 0.64 & 0.84 \\
        
        150 & 157.21 & 156.87 & 0.34 & 0.22 \\
        
        200 & 258.63 & 258.64 & -0.01 & -0.0039 \\
        \hline
    \end{tabular}
\end{table}
~\\Without a magnetic field and for a water phantom, relative differences decrease significantly with increasing energy. In all cases, the range calculated analytically using the developed algorithm is higher, except for \( \symup{E = 200 \, \, MeV} \). Absolute differences also decrease, with a maximum of \( \symup{0.93 \, \, mm} \).
\newpage
The relative dose value required to achieve \(s_{\symup{L1}} = s_{\symup{L2}}\), for further magnetic field introductions, is visualized in \autoref{fig:waterb0e200mevdistance}. This example corresponds to the candidate energy value previously examined in the \gls{mc} simulation in water.
\begin{figure}[!h]
    \centering
    \includegraphics[width=\textwidth]{../Screenshots/water/Water_B0_200MeV_distance.png}
    \caption{Percentage depth dose curve and marked $\symup{R_{80}}$ range with the corresponding relative absorbed dose value from a \gls{mc} simulation (\gls{topas}) in water with $E = \symup{200 \, \, MeV}$ and $B_z = \symup{0 \, \, T}$ with $N = 10^5$ protons. $s_{\symup{L1}}$ refers to the proton trajectory length in a magnetic field and $s_{\symup{L2}}$ refers to the proton trajectory length without a magnetic field.}
    \label{fig:waterb0e200mevdistance}
\end{figure}
~\\Multiple relative absorbed dose values for four different energy values with the calculation of $R_{\symup{80}}$ for $s = s_{\symup{L1}} = s_{\symup{L2}}$ for the analyzed water phantom are listed in \autoref{tab:waterrelativedoser80}.
\begin{table}[h!]
    \centering
    \caption{Relative dose values calculated using \gls{mc} simulations (\gls{topas}) in a water phantom with $B_z = \symup{0 \, \, T}$ and four different energy values with prior determined proton ranges and trajectory lengths.}
    \label{tab:waterrelativedoser80}
    \begin{tabular}{c c}
        %\multicolumn{4}{c}{\textbf{B = 0 T}} \\
        \hline
        E [MeV] &  Rel. Dose [\%] \\
        \hline
        50  & 79.90 \\
        
        100 & 81.60 \\
        
        150 & 82.60 \\
        
        200 & 83.69 \\
        \hline
    \end{tabular}
\end{table}

\subsubsection{B > 0 T}
The relative dose value for the same example previously shown without a magnetic field ($\symup{83.69 \,\, \%}$) and its corresponding trajectory length $s_{\symup{L1}}$ for $B_z = \symup{1.5 \, \, T}$ is analyzed and visualized in \autoref{fig:waterb15e200mevdistance}.
\begin{figure}[!h]
    \centering
    \includegraphics[width=\textwidth]{../Screenshots/water/Water_B15_200MeV_distance.png}
    \caption{Percentage depth dose curve and marked $s = s_{\symup{L1}} = s_{\symup{L2}}$ value with the corresponding relative absorbed dose value from a \gls{mc} simulation (\gls{topas}) in water with $E = \symup{200 \, \, MeV}$ and $B_z = \symup{1.5 \, \, T}$ with $N = 10^5$ protons. $s_{\symup{L1}}$ refers to the proton trajectory length in a magnetic field and $s_{\symup{L2}}$ refers to the proton trajectory length without a magnetic field.}
    \label{fig:waterb15e200mevdistance}
\end{figure}
~\\
The trajectory lengths $s$ calculated analytically in the developed proton transfer algorithm ($s_{\symup{ana}}$) and with \gls{mc} simulations ($s_{\symup{MC}}$) for the analyzed water phantom are listed in table \autoref{tab:compareMATLABTOPASdistancewater}.
\begin{table}[h!]
    \centering
    \caption{Comparison of trajectory lengths calculated analytically in the developed proton transfer algorithm ($s_{\symup{ana}}$) and with \gls{mc} simulations ($s_{\symup{MC}}$) for a water phantom. Different energy values and magnetic field strengths are analyzed with their absolute and relative differences listed.}
    \label{tab:compareMATLABTOPASdistancewater}
    \begin{tabular}{c c c c c}
        
        \multicolumn{5}{c}{\textbf{B = 1.5 T}} \\
        \hline
        E [MeV] & $s_{\symup{ana}}$ [mm] & $s_{\symup{MC}}$ [mm] & Abs. Difference [mm] & Rel. Difference [\%] \\
        \hline
        50  & 22.19 & 21.26 & 0.93 & 4.37 \\
        
        100 & 76.92 & 76.22 & 0.70 & 0.92 \\
        
        150 & 157.21 & 157.01 & 0.20 & 0.13 \\
        
        200 & 258.63 & 258.29 & 0.34 & 0.13 \\
        \hline
    \end{tabular}
    
    \vspace{1cm}

    \begin{tabular}{c c c c c}
        
        \multicolumn{5}{c}{\textbf{E = 100 MeV}} \\
        \hline
        B [T] & $s_{\symup{ana}}$ [mm] & $s_{\symup{MC}}$ [mm] & Abs. Difference [mm] & Rel. Difference [\%] \\
        \hline
        0.5 & 76.92 & 76.26 & 0.66 & 0.87 \\
        
        1   & 76.92 & 76.27 & 0.65 & 0.85 \\
        
        2   & 76.92  & 76.24 & 0.68 & 0.89 \\
        
        3   & 76.92  & 76.32 & 0.60 & 0.79 \\
        \hline
    \end{tabular}
\end{table}
~\\

In addition, \autoref{tab:compareMATLABTOPAScoordinateswater} lists the coordinates for the proton stopping positions calculated using the developed proton transfer algorithm ($\vec{V}_{\symup{ana}}$) and using \gls{mc} simulations ($\vec{V}_{\symup{MC}}$) for the analyzed water phantom.
%\begin{table}[h!]
%    \centering
%    \caption{Comparison of \(\vec{X}_{\symup{MATLAB}}\) and \(\vec{X}_{\symup{TOPAS}}\) for a water phantom with \(\vec{X} = (x \, \, \, y)^{\symup{T}}\). Different energy values and magnetic field strengths are analyzed with their differences listed.}
%    \label{tab:compareMATLABTOPAScoordinateswater}
%    \begin{tabular}{c c c c }
%        \multicolumn{4}{c}{\textbf{B = 1.5 T}} \\
%        \hline
%        E [MeV] & \(\vec{X}_{\symup{MATLAB}}\) [mm] & \(\vec{X}_{\symup{TOPAS}}\) [mm] & Difference [\%] \\
%        \hline
%        50  & 
%        \(\begin{pmatrix} 22.19 \\ 122.87 \end{pmatrix}\) & 
%        \(\begin{pmatrix} 21.26 \\ 122.89 \end{pmatrix}\) & 
%        \(\begin{pmatrix} 4.37 \\ 0.016 \end{pmatrix} \) \\
%        \vspace{0.005cm} \\
%        
%        100 & 
%        \(\begin{pmatrix} 76.83 \\ 125.79 \end{pmatrix}\) & 
%        \(\begin{pmatrix} 76.11 \\ 125.87 \end{pmatrix}\) & 
%        \(\begin{pmatrix} 0.94 \\ 0.064 \end{pmatrix} \) \\
%        \vspace{0.005cm} \\
%        
%        150 & 
%        \(\begin{pmatrix} 156.63 \\ 134.13 \end{pmatrix}\) & 
%        \(\begin{pmatrix} 156.38 \\ 133.97 \end{pmatrix}\) & 
%        \(\begin{pmatrix} 0.16 \\ 0.12 \end{pmatrix} \) \\
%        \vspace{0.005cm} \\
%        
%        200 & 
%        \(\begin{pmatrix} 256.58 \\ 150.62 \end{pmatrix}\) & 
%        \(\begin{pmatrix} 256.21 \\ 149.24 \end{pmatrix}\) & 
%        \(\begin{pmatrix} 0.14 \\ 0.92 \end{pmatrix} \) \\
%        \hline
%    \end{tabular}
%    
%    \vspace{1cm}
%    
%    \begin{tabular}{c c c c}
%        \multicolumn{4}{c}{\textbf{E = 100 MeV}} \\
%        \hline
%        B [T] & \(\vec{X}_{\symup{MATLAB}}\) [mm] & \(\vec{X}_{\symup{TOPAS}}\) [mm] & Difference [\%] \\
%        \hline
%        0.5 & 
%        \(\begin{pmatrix} 76.91 \\ 123.60 \end{pmatrix}\) & 
%        \(\begin{pmatrix} 76.24 \\ 123.64 \end{pmatrix}\) & 
%        \(\begin{pmatrix} 0.87 \\ 0.032 \end{pmatrix} \) \\
%        \vspace{0.005cm} \\
%        
%        1   & 
%        \(\begin{pmatrix} 76.88 \\ 124.70 \end{pmatrix}\) & 
%        \(\begin{pmatrix} 76.22 \\ 124.76 \end{pmatrix}\) & 
%        \(\begin{pmatrix} 0.86 \\ 0.048 \end{pmatrix} \) \\
%        \vspace{0.005cm} \\
%        
%        2   & 
%        \(\begin{pmatrix} 76.76 \\ 126.89 \end{pmatrix}\) & 
%        \(\begin{pmatrix} 76.04 \\ 126.99 \end{pmatrix}\) & 
%        \(\begin{pmatrix} 0.94 \\ 0.079 \end{pmatrix} \) \\
%        \vspace{0.005cm} \\
%        
%        3   & 
%        \(\begin{pmatrix} 76.55 \\ 129.09 \end{pmatrix}\) & 
%        \(\begin{pmatrix} 75.89 \\ 129.21 \end{pmatrix}\) & 
%        \(\begin{pmatrix} 0.86 \\ 0.093 \end{pmatrix} \) \\
%        \hline
%    \end{tabular}
%\end{table}
\begin{table}[h!]
    \centering
    \caption{Comparison of end positions calculated analytically using the developed proton transfer algorithm (\(\vec{V}_{\symup{ana}}\)) and using \gls{mc} simulations (\(\vec{V}_{\symup{MC}}\)) for a water phantom with \(\vec{V} = (x, \, y)\). Different energy values and magnetic field strengths are analyzed and their vectorial absolute differences ($\increment \vec{V}$) are listed. Additionally, the Euclidean distances ($| \vec{V} | = \sqrt{(x_{\symup{MC}} - x_{\symup{ana}})^2 + (y_{\symup{MC}} - y_{\symup{ana}})^2}$) are shown.}
    \label{tab:compareMATLABTOPAScoordinateswater}
    \begin{tabular}{c c c c c}
        \multicolumn{5}{c}{\textbf{B = 1.5 T}} \\
        \hline
        E [MeV] & \(\vec{V}_{\symup{ana}}\) [$\symup{mm}$] & \(\vec{V}_{\symup{MC}}\) [$\symup{mm}$] & $\increment \vec{V}$ [$\symup{mm}$] & $| \vec{V} |$ [mm] \\
        \hline
        50  & 
        \((22.19, 122.87)\) & 
        \((21.26, 122.89)\) &
        \((0.93, - 0.02)\) &
        0.93 \\
        
        100 & 
        \((76.83, 125.79)\) & 
        \((76.11, 125.87)\) &
        \((0.72, - 0.08)\) &
        0.72 \\
        
        150 & 
        \((156.63, 134.13)\) & 
        \((156.38, 133.97)\) &
        \((0.25, 0.16)\) &
        0.30 \\
        
        200 & 
        \((256.58, 150.62)\) & 
        \((256.21, 149.24)\) &
        \((0.37, 1.38)\) &
        1.43 \\
        \hline
    \end{tabular}
    
    \vspace{1cm}
    
    \begin{tabular}{c c c c c}
        \multicolumn{5}{c}{\textbf{E = 100 MeV}} \\
        \hline
        B [T] & \(\vec{V}_{\symup{ana}}\) [$\symup{mm}$] & \(\vec{V}_{\symup{MC}}\) [$\symup{mm}$] & $\increment \vec{V}$ [$\symup{mm}$] & $| \vec{V} |$ [mm] \\
        \hline
        0.5 & 
        \((76.91, 123.60)\) & 
        \((76.24, 123.64)\) &
        \((0.67, - 0.04)\) &
        0.67 \\
        
        1   & 
        \((76.88, 124.70)\) & 
        \((76.22, 124.76)\) &
        \((0.66, - 0.06)\) &
        0.66 \\
        
        2   & 
        \((76.76, 126.89)\) & 
        \((76.04, 126.99)\) &
        \((0.72, - 0.10)\) &
        0.73 \\
        
        3   & 
        \((76.55, 129.09)\) & 
        \((75.89, 129.21)\) &
        \((0.66, -0.12)\) & 
        0.67 \\
        \hline
    \end{tabular}
\end{table}
~\\
~\\
~\\
~\\
For trajectory lengths calculated and listed in \autoref{tab:compareMATLABTOPASdistancewater}, absolute differences vary from \( \symup{0.34 \, \, mm} \) to \( \symup{0.93 \, \, mm} \). Generally, relative differences decrease with increasing energy, while they remain nearly constant across varying magnetic field strengths. In \autoref{tab:compareMATLABTOPAScoordinateswater}, absolute differences for increasing energy values appear to decrease until \( \symup{200 \, \, MeV} \), where a slight increase in \( x \) and a larger increase in \( y \) can be observed. Additionally, distances initially decrease until reaching \( \symup{200 \, \, MeV} \), at which point an increase in distance is seen. In contrast, for a fixed energy value and varying magnetic field strengths, dimensional absolute differences and distances appear to converge to an average value.
\subsection{Bone phantom}
\label{sec:bonephantom}
\subsubsection*{B = 0 T}
\label{sec:bonephantomb0t}
%A next possible analysis step is to focus on the effect of the \gls{hulut} used and the proton interaction with a more complex material. For this purpose, bone material is examined. Bone structures can have different CT number values. For this work, a CT number of 1000 HU is assumed to represent bone material \cite{denotter_hounsfield_2024}.
%~\\
A comparison of proton ranges calculated using the developed proton transfer algorithm ($r_{\symup{ana}}$) and those extracted from \gls{mc} simulations ($r_{\symup{MC}}$) for proton beams with different energies in a bone phantom without a magnetic field is listed in \autoref{tab:MATLABTOPASboneB0}.\begin{table}[h!]
    \centering
    \caption{Comparison of proton ranges calculated analytically using the developed proton transfer algorithm ($r_{\symup{ana}}$) and proton ranges ranges extracted from \gls{mc} simulation (\gls{topas}) depth dose curves ($r_{\symup{MC}}$) for a bone phantom without a magnetic field. Four different energy values are analyzed and the resulting relative and absolute differences between the two ranges can be seen.}    \label{tab:MATLABTOPASboneB0}
    \begin{tabular}{c c c c c}
        
        \multicolumn{5}{c}{\textbf{B = 0 T}} \\
        \hline
        E [MeV] & $r_{\symup{ana}}$ [mm] & $r_{\symup{MC}}$ [mm] & Abs. Difference [mm] & Rel. Difference [\%] \\
        \hline
        50  & 15.19 & 14.38 & 0.81 & 5.63 \\
        
        100 & 52.63 & 50.17 & 2.46 & 4.90 \\
        
        150 & 107.55 & 103.07 & 4.48 & 4.35 \\
        
        200 & 176.93 & 169.84 & 7.09 & 4.17 \\
        \hline
    \end{tabular}
\end{table}
~\\In the bone phantom, with increasing energy values and without a magnetic field, both absolute and relative differences are larger than the results for the water phantom (cf. \autoref{sec:waterphantom}). Generally, all analytically calculated proton stopping positions (\( r_{\symup{ana}} \)) are higher than the \gls{mc} based calculated proton stopping positions (\( r_{\symup{MC}} \)). With increasing energy, absolute differences rise from an initial value of \( \symup{0.81 \, mm} \) up to \( \symup{7.09 \, mm} \).
~\\

Relative dose values calculated with $R_{\symup{80}}$ needed for $s = s_{\symup{L1}} = s_{\symup{L2}}$ are analyzed for the bone phantom. These results are listed in table \autoref{tab:bonerelativedoser80}.
\begin{table}[h!]
    \centering
    \caption{Relative dose values calculated using \gls{mc} simulations (\gls{topas}) in a bone phantom with $B_z = \symup{0 \, \, T}$ and four different energy values with prior determined proton ranges and trajectory lengths.}
    \label{tab:bonerelativedoser80}
    \begin{tabular}{c c}
        %\multicolumn{4}{c}{\textbf{B = 0 T}} \\
        \hline
        E [MeV] &  Rel. Dose [\%] \\
        \hline
        50  & 87.76 \\
        
        100 & 82.85 \\
        
        150 & 82.28 \\
        
        200 & 82.48 \\
        \hline
    \end{tabular}
\end{table}
\subsubsection*{B > 0 T}
\label{sec:bonephantombg0t}
The trajectory lengths $s$ calculated analytically in the developed proton transfer algorithm ($s_{\symup{ana}}$) and with \gls{mc} simulations ($s_{\symup{MC}}$) for the analyzed bone phantom are listed in table \autoref{tab:compareMATLABTOPASdistancebone}.
\begin{table}[h!]
    \centering
    \caption{Comparison of trajectory lengths calculated analytically in the developed proton transfer algorithm ($s_{\symup{ana}}$) and with \gls{mc} simulations ($s_{\symup{MC}}$) for a bone phantom. Different energy values and magnetic field strengths are analyzed with their absolute and relative differences listed.}
    \label{tab:compareMATLABTOPASdistancebone}
    \begin{tabular}{c c c c c}
        
        \multicolumn{5}{c}{\textbf{B = 1.5 T}} \\
        \hline
        E [MeV] & $s_{\symup{ana}}$ [mm] & $s_{\symup{MC}}$ [mm] & Abs. Difference [mm] & Rel. Difference [\%] \\
        \hline
        50  & 15.19 & 14.29 & 0.9 & 6.29 \\
        
        100 & 52.63 & 50.04 & 2.59 & 5.18 \\
        
        150 & 107.55 & 103.11 & 4.44 & 4.31 \\
        
        200 & 176.93 & 169.82 & 7.11 & 4.19 \\
        \hline
    \end{tabular}
    
    \vspace{1cm}

    \begin{tabular}{c c c c c}
        
        \multicolumn{5}{c}{\textbf{E = 100 MeV}} \\
        \hline
        B [T] & $s_{\symup{ana}}$ [mm] & $s_{\symup{MC}}$ [mm] & Abs. Difference [mm] & Rel. Difference [\%] \\
        \hline
        0.5 & 52.63 & 50.15 & 2.48 & 4.95 \\
        
        1   & 52.63 & 50.08 & 2.55 & 5.09 \\
        
        2   & 52.63  & 50.00 & 2.63 & 5.26\\
        
        3   & 52.63  & 49.90 & 2.73 & 5.47 \\
        \hline
    \end{tabular}
\end{table}
~\\Relative differences decrease with increasing energy, though less pronounced than in the water phantom analysis (cf. \autoref{tab:compareMATLABTOPASdistancewater}). Absolute differences, however, increase with energy, reaching a maximum of \( \symup{7.11 \, \, \mathrm{mm}} \). For varying magnetic field strengths, both absolute and relative differences converge to nearly constant values, mirroring observations in the water phantom analysis.
\newpage
For this bone phantom, \autoref{tab:compareMATLABTOPAScoordinatesbone} lists the coordinates for the proton stopping positions calculated using the developed proton transfer algorithm ($\vec{V}_{\symup{ana}}$) and using \gls{mc} simulations ($\vec{V}_{\symup{MC}}$).
\begin{table}[h!]
    \centering
    \caption{Comparison of end positions calculated analytically using the developed proton transfer algorithm (\(\vec{V}_{\symup{ana}}\)) and using \gls{mc} simulations (\(\vec{V}_{\symup{MC}}\)) for a bone phantom with \(\vec{V} = (x, \, y)\) assuming a constant \gls{ct} number of $\symup{1000 \, \, HU}$. Different energy values and magnetic field strengths are analyzed and their vectorial absolute differences ($\increment \vec{V}$) are listed. Additionally, the Euclidean distances ($| \vec{V} | = \sqrt{(x_{\symup{MC}} - x_{\symup{ana}})^2 + (y_{\symup{MC}} - y_{\symup{ana}})^2}$) are shown.}
    \label{tab:compareMATLABTOPAScoordinatesbone}
    \begin{tabular}{c c c c c}
        \multicolumn{5}{c}{\textbf{B = 1.5 T}} \\
        \hline
        E [MeV] & \(\vec{V}_{\symup{ana}}\) [$\symup{mm}$] & \(\vec{V}_{\symup{MC}}\) [$\symup{mm}$] & $\increment \vec{V}$ [$\symup{mm}$] & $| \vec{V} |$ [mm] \\
        \hline
        50  & 
        \((15.19, 122.68)\) & 
        \((14.29, 122.67)\) &
        \((0.9, 0.01)\) &
        0.90 \\
        
        100 & 
        \((52.60, 124.05)\) & 
        \((50.01, 123.93)\) &
        \((2.59, 0.12)\) &
        2.59 \\
        
        150 & 
        \((107.36, 127.95)\) & 
        \((102.92, 127.38)\) &
        \((4.44, 0.57)\) &
        4.48 \\
        
        200 & 
        \((176.27, 135.69)\) & 
        \((169.18, 133.87)\) &
        \((7.09, 1.82)\) &
        7.32 \\
        \hline
    \end{tabular}
    
    \vspace{1cm}
    
    \begin{tabular}{c c c c c}
        \multicolumn{5}{c}{\textbf{E = 100 MeV}} \\
        \hline
        B [T] & \(\vec{V}_{\symup{ana}}\) [$\symup{mm}$] & \(\vec{V}_{\symup{MC}}\) [$\symup{mm}$] & $\increment \vec{V}$ [$\symup{mm}$] & $| \vec{V} |$ [mm] \\
        \hline
        0.5 & 
        \((52.63, 123.02)\) & 
        \((50.14, 122.97)\) &
        \((2.49, 0.06)\) &
        2.49 \\
        
        1   & 
        \((52.62, 123.53)\) & 
        \((50.06, 123.46)\) &
        \((2.56, 0.07)\) &
        2.56 \\
        
        2   & 
        \((52.58, 124.56)\) & 
        \((49.94, 124.42)\) &
        \((2.64, 0.14)\) &
        2.64 \\
        
        3   & 
        \((52.51, 125.59)\) & 
        \((49.78, 125.38)\) &
        \((2.73, 0.21)\) & 
        2.73 \\
        \hline
    \end{tabular}
\end{table}
~\\Evaluating proton stopping positions for the bone phantom shows that, with increasing energy, the difference between \( \vec{V}_{\symup{ana}} \) and \( \vec{V}_{\symup{MC}} \) increases significantly along \( x \), while the increase along \( y \) is present but less pronounced. Additionally, the analytically developed proton transfer algorithm consistently yields higher proton stopping positions than those calculated using \gls{mc} simulations. For fixed energy and varying magnetic field strengths, differences along \( x \) are more prominent than along \( y \).
\newpage
\subsection{Prostate patient}
\label{sec:prostate}
\subsubsection*{B = 0 T}
Finally, for the given prostate patient \gls{ct} dataset, a comparison of proton ranges calculated using the developed proton transfer algorithm ($r_{\symup{ana}}$) and those extracted from \gls{mc} simulations ($r_{\symup{MC}}$) for proton beams with different energies without a magnetic field is listed in \autoref{tab:MATLABTOPASboneB0}.
\begin{table}[h!]
    \centering
    \caption{Comparison of proton ranges calculated analytically using the developed proton transfer algorithm ($r_{\symup{ana}}$) and proton ranges ranges extracted from \gls{mc} simulation (\gls{topas}) depth dose curves ($r_{\symup{MC}}$) for given prostate patient \gls{ct} dataset without a magnetic field. Four different energy values are analyzed and the resulting relative and absolute differences between the two ranges can be seen.}
    \label{tab:MATLABTOPASprostateB0}
    \begin{tabular}{c c c c c}
        
        \multicolumn{5}{c}{\textbf{B = 0 T}} \\
        \hline
        E [MeV] & $r_{\symup{ana}}$ [mm] & $r_{\symup{MC}}$ [mm] & Abs. Difference [mm] & Rel. Difference [\%] \\
        \hline
        50  & 31.48 & 30.74 & 0.74 & 2.41 \\
        
        100 & 81.03 & 80.80 & 0.23 & 0.28 \\
        
        150 & 150.66 & 152.33 & -1.67 & -1.11 \\
        
        200 & 244.97 & 249.97 & -5.00 & -2.04 \\
        \hline
    \end{tabular}
\end{table}
~\\Without a magnetic field, absolute differences between \( r_{\symup{ana}} \) and \( r_{\symup{MC}} \) decrease, starting from \( \symup{0.74 \, \, mm} \) and reaching \( \symup{-5.00 \, \, mm} \). Up to \( \symup{100 \, \, MeV} \), \( r_{\symup{ana}} > r_{\symup{MC}} \) is given, which is consistent with previous results. However, at higher energies, \( r_{\symup{ana}} < r_{\symup{MC}} \) can be observed. Relative differences follow a similar decreasing trend.
~\\

Multiple relative absorbed dose values for four different energy values with the calculation of $R_{\symup{80}}$ for $s = s_{\symup{L1}} = s_{\symup{L2}}$ for the prostate patient dataset are listed in \autoref{tab:waterrelativedoser80}.
\begin{table}[h!]
    \centering
    \caption{Relative dose values calculated using \gls{mc} simulations (\gls{topas}) in given prostate patient \gls{ct} dataset with $B_z = \symup{0 \, \, T}$ and four different energy values with prior determined proton ranges and trajectory lengths.}
    \label{tab:prostaterelativedoser80}
    \begin{tabular}{c c}
        %\multicolumn{4}{c}{\textbf{B = 0 T}} \\
        \hline
        E [MeV] &  Rel. Dose [\%] \\
        \hline
        50  & 83.33 \\
        
        100 & 83.44 \\
        
        150 & 82.95 \\
        
        200 & 79.04 \\
        \hline
    \end{tabular}
\end{table}
\newpage
\subsubsection*{B > 0 T}
Additionally, the trajectory lengths $s$ calculated analytically in the developed proton transfer algorithm ($s_{\symup{ana}}$) and with \gls{mc} simulations ($s_{\symup{MC}}$) for the given prostate patient \gls{ct} dataset are listed in table \autoref{tab:compareMATLABTOPASdistanceprostate}.
\begin{table}[h!]
    \centering
    \caption{Comparison of trajectory lengths calculated analytically in the developed proton transfer algorithm ($s_{\symup{ana}}$) and with \gls{mc} simulations ($s_{\symup{MC}}$) for given prostate patient \gls{ct} dataset. Different energy values and magnetic field strengths are analyzed with their absolute and relative differences listed.}
    \label{tab:compareMATLABTOPASdistanceprostate}
    \begin{tabular}{c c c c c}
        
        \multicolumn{5}{c}{\textbf{B = 1.5 T}} \\
        \hline
        E [MeV] & $s_{\symup{ana}}$ [mm] & $s_{\symup{MC}}$ [mm] & Abs. Difference [mm] & Rel. Difference [\%] \\
        \hline
        50  & 31.49 & 30.41 & 1.08 & 3.55 \\
        
        100 & 82.94 & 80.62 & 2.32 & 2.75 \\
        
        150 & 154.96 & 153.60 & 1.36 & 0.89 \\
        
        200 & 254.35 & 256.92 & -2.57 & -1.01 \\
        \hline
    \end{tabular}
    
    \vspace{1cm}

    \begin{tabular}{c c c c c}
        
        \multicolumn{5}{c}{\textbf{E = 100 MeV}} \\
        \hline
        B [T] & $s_{\symup{ana}}$ [mm] & $s_{\symup{MC}}$ [mm] & Abs. Difference [mm] & Rel. Difference [\%] \\
        \hline
        0.5 & 81.04 & 80.69 & 0.35 & 0.43 \\
        
        1   & 81.55 & 80.68 & 0.87 & 1.08 \\
        
        2   & 84.36  & 80.61 & 3.75 & 4.65 \\
        
        3   & 84.09  & 80.93 & 3.16 & 3.90 \\
        \hline
    \end{tabular}
\end{table}
~\\Relative differences between \( s_{\symup{ana}} \) and \( s_{\symup{MC}} \) appear to decrease with increasing energy. A trend of \( r_{\symup{ana}} > r_{\symup{MC}} \) is observed up to the final energy value of \( \symup{E = 200 \, MeV} \), where \( r_{\symup{ana}} < r_{\symup{MC}} \) occurs. For a fixed energy and increasing magnetic field strength, a notable jump in absolute and relative difference values is observed from \( \symup{1 \, T} \) to \( \symup{2 \, T} \).


\newpage
The \autoref{tab:compareMATLABTOPAScoordinatesprostate} lists the coordinates for the proton stopping positions calculated using the developed proton transfer algorithm ($\vec{V}_{\symup{ana}}$) and using \gls{mc} simulations ($\vec{V}_{\symup{MC}}$) for the given prostate patient \gls{ct} dataset.
\begin{table}[h!]
    \centering
    \caption{Comparison of end positions calculated analytically using the developed proton transfer algorithm (\(\vec{V}_{\symup{ana}}\)) and using \gls{mc} simulations (\(\vec{V}_{\symup{MC}}\)) for given prostate patient \gls{ct} dataset with \(\vec{V} = (x, \, y)\). Different energy values and magnetic field strengths are analyzed and their vectorial absolute differences ($\increment \vec{V}$) are listed. Additionally, the Euclidean distances ($| \vec{V} | = \sqrt{(x_{\symup{MC}} - x_{\symup{ana}})^2 + (y_{\symup{MC}} - y_{\symup{ana}})^2}$) are shown.}
    \label{tab:compareMATLABTOPAScoordinatesprostate}
    \begin{tabular}{c c c c c}
        \multicolumn{5}{c}{\textbf{B = 1.5 T}} \\
        \hline
        E [MeV] & \(\vec{V}_{\symup{ana}}\) [$\symup{mm}$] & \(\vec{V}_{\symup{MC}}\) [$\symup{mm}$] & $\increment \vec{V}$ [$\symup{mm}$] & $| \vec{V} |$ [mm] \\
        \hline
        50  & 
        \((31.49, 123.25)\) & 
        \((30.39, 123.24)\) &
        \((1.10, 0.01)\) &
        1.10 \\
        
        100 & 
        \((82.82, 126.33)\) & 
        \((80.48, 126.10)\) &
        \((2.34, 0.23)\) &
        2.35 \\
        
        150 & 
        \((154.41, 133.80)\) & 
        \((152.73, 134.26)\) &
        \((1.68, -0.46)\) &
        1.74 \\
        
        200 & 
        \((252.41, 149.70)\) & 
        \((254.33, 150.72)\) &
        \((-1.92, -1.02)\) &
        2.17 \\
        \hline
    \end{tabular}
    
    \vspace{1cm}
    
    \begin{tabular}{c c c c c}
        \multicolumn{5}{c}{\textbf{E = 100 MeV}} \\
        \hline
        B [T] & \(\vec{V}_{\symup{ana}}\) [$\symup{mm}$] & \(\vec{V}_{\symup{MC}}\) [$\symup{mm}$] & $\increment \vec{V}$ [$\symup{mm}$] & $| \vec{V} |$ [mm] \\
        \hline
        0.5 & 
        \((81.03, 123.72)\) & 
        \((80.65, 123.84)\) &
        \((0.38, -0.12)\) &
        0.40 \\
        
        1   & 
        \((81.49, 124.72)\) & 
        \((80.62, 125.00)\) &
        \((0.87, -0.28)\) &
        0.91 \\
        
        2   & 
        \((84.14, 127.79)\) & 
        \((80.35, 127.18)\) &
        \((3.79, 0.61)\) &
        3.84 \\
        
        3   & 
        \((83.61, 130.34)\) & 
        \((80.19, 129.01)\) &
        \((3.42, 1.33)\) & 
        3.67 \\
        \hline
    \end{tabular}
\end{table}
~\\For the complex case of a prostate patient, no consistent trend of increase or decrease in absolute differences is observed, contrasting with previous analyses (cf. \autoref{tab:compareMATLABTOPAScoordinateswater}, \autoref{tab:compareMATLABTOPAScoordinatesbone}). Increasing energy values lead to both increases and decreases in absolute differences. Distances show similar results, ranging from \( \symup{1.10 \, \, mm} \) to \( \symup{2.35 \, \, mm} \). However, for a fixed energy value and increasing magnetic field strengths, a shift in absolute differences and distances is observed when increasing the field from \( \symup{1 \, \, T} \) to \( \symup{2 \, \, T} \). Across varying energy values and magnetic field strengths, no clear trend or correlation between \( \vec{V}_{\symup{ana}} \) and \( \vec{V}_{\symup{MC}} \) is apparent.
\newpage
An example of two magnetic field proton beam trajectories for analyzed prostate patient \gls{ct} dataset is visualized in \autoref{fig:prostatetrajectories}.
\begin{figure}[h!]
    \centering
    \includegraphics[width=\textwidth]{../Screenshots/prostate/prostate.png}
    \caption{Visualization of proton beam trajectories from \gls{mc} simulations (\gls{topas}) with an introduced magnetic field ($B_z = \symup{1.5 \, \, T}$) and two candidate energy values ($E = \symup{150 \, \, MeV}$ and $E = \symup{200 \, \, MeV}$) for the prostate patient \gls{ct} dataset. The proton stopping positions are highlighted for $E = \symup{150 \, \, MeV}$ at $\vec{V}_{\symup{150 \, MeV}} = \left(152.73, 134.26\right) \, \symup{mm}$ (light green dot) and for $E = \symup{200 \, \, MeV}$ at $\vec{V}_{\symup{200 \, MeV}} = \left(254.33, 150.72\right) \, \symup{mm}$ (cyan dot), where $\vec{V} = (x, \, y)$.}
    \label{fig:prostatetrajectories}
\end{figure}

\newpage
\section{Recalculation of spot selection for a matRad treatment field}
\label{sec:matRadPlan}
After comparing the results from the developed proton transfer algorithm with those obtained from \gls{mc} simulations, the focus shifts to the \gls{tps} matRad. Initially, a treatment field is introduced along with its resulting dose distribution (see \autoref{sec:treatment}). The initial starting positions and energy values are exported for comparison of pencil beam end-positions within the \gls{ctv} (see \autoref{sec:comparison}). Furthermore, a magnetic field is introduced, and position differences are analyzed (see \autoref{sec:matradmagneticfield}). Finally, the gradient descent algorithm is applied to optimize and recalculate deflected coordinate positions, shifting them towards the initial end positions (see \autoref{sec:matradgradientdescent}).

\subsection{Proton treatment field}
\label{sec:treatment}
The prostate patient \gls{ct} is overlaid with the combined dose distribution of 88 proton pencil beams, generated by the \gls{tps} to cover the \gls{ctv}, as visualized in \autoref{fig:treatmentplanmatraddose}.
\begin{figure}[h!]
    \centering
    \includegraphics[width=12cm]{../Screenshots/matrad/treatmentmatrad.png}
    \caption{Dose distribution for given prostate \gls{ct} dataset with objective of achieving a mean dose of $\symup{50 \, \, Gy}$. A candidate \gls{ctv} (prostate) is highlighted in red, along with potential \glspl{oar} (femoral heads and rectum) in pink and dark red.}
    \label{fig:treatmentplanmatraddose}
\end{figure}
\newpage
The exported stopping positions for each pencil beam for this treatment field are visualized in \autoref{fig:matradr80bixels}.
\begin{figure}[h!]
    \centering
    \includegraphics[width=\textwidth]{../Screenshots/matrad/matradr80.png}
    \caption{Visualization of a simplified contour representing only the body contour of given prostate patient \gls{ct} dataset. The \gls{ctv} is highlighted (white) as well as the proton stopping positions for each pencil beam retrieved from the "dij" matrix after matRad irradiation (blue dots).}
    \label{fig:matradr80bixels}
\end{figure}
~\\For the following analysis, a section of the body contour around the \gls{ctv} is focused on to facilitate visualization and to better differentiate pencil beam spots.

\newpage

\subsection{Verifying treatment field beam positions}
\label{sec:comparison}
%In order to introduce a magnetic field into matRad, the current end positions visualized in \autoref{fig:matradr80bixels} need to be reproduced in MATLAB. These initial positions and energy values are exported from the "stf" structure, imported and calculated in MATLAB.
The following \autoref{fig:matradmatlabr80bixels} visualizes the proton stopping positions calculated using the developed proton transfer algorithm and compares them with previously mentioned matRad extracted proton stopping positions.
\begin{figure}[h!]
    \centering
    \includegraphics[width=\textwidth]{../Screenshots/matrad/matradmatlabr80.png}
    \caption{Visualization of the enlarged prostate patient \gls{ct} dataset with highlighted \gls{ctv} (white). Qualitative comparison between proton stopping positions for each pencil beam calculated via the developed proton transfer algorithm (orange dots) and matRad (blue dots) using the same initial positions and energy values.}
    \label{fig:matradmatlabr80bixels}
\end{figure}
~\\Using identical initial starting positions and energy values, a trend is observed across the rows. For instance, in the first row, the proton stopping positions are closer to the matRad stopping positions compared to those in the second row. However, within each row, a consistent difference is maintained between the proton stopping positions calculated using the developed proton transfer algorithm and the matRad stopping positions.
\newpage
A quantitative analysis is done by calculating the Euclidean distances of matRad and proton stopping positions calculated via the developed proton transfer algorithm. This is visualized as a histogram in \autoref{fig:matradmatlabr80bixelshistogram}.
\begin{figure}[h!]
    \centering
    \includegraphics[width=\textwidth]{../Screenshots/statistics/initialmatlabvsmatradhistogram.png}
    \caption{Histogram for quantitatively analyzing difference between stopping positions calculated via the developed proton transfer algorithm ($B_z = \symup{0 \, \, T}$) and matRad extracted stopping positions using initial matRad starting positions and energy values. Median (red) and mean (green) values are highlighted.}
    \label{fig:matradmatlabr80bixelshistogram}
\end{figure}
~\\A general trend is observed, with most pencil beams showing distances below \( \symup{1.0 \, \, mm} \). The median distance is \( \symup{0.49 \, \, mm} \), and the mean distance is \( \symup{0.77 \, \, mm} \).
\newpage
\subsection{Calculation of deflected spots}
\label{sec:matradmagneticfield}
The initial proton stopping positions extracted from matRad are visually compared to the proton stopping positions influenced by a magnetic field ($B_z = \symup{1.5 \, \, \mathrm{T}}$) in \autoref{fig:matradmatlabr80bixelsshifted}. The magnetic field influenced stopping positions are calculated using the developed proton transfer algorithm.
\begin{figure}[h!]
    \centering
    \includegraphics[width=\textwidth]{../Screenshots/matrad/matradmatlabr80shifted.png}
    \caption{Visualization of the enlarged prostate patient \gls{ct} dataset with highlighted \gls{ctv} (white). Qualitative comparison between stopping positions calculated using the developed proton transfer algorithm (orange dots) and extracted from matRad (blue dots) using the same initial positions and energy values. A magnetic field ($B_z = \symup{1.5 \, \, T}$) is introduced in the developed proton transfer algorithm.}
    \label{fig:matradmatlabr80bixelsshifted}
\end{figure}
~\\Without further optimization of the original pencil beams in terms of energy and position, the \gls{ctv} would no longer be fully covered under the influence of a magnetic field.
\newpage
A histogram visualizing the Euclidean distances between stopping positions calculated using the developed proton transfer algorithm and extracted from matRad is visualized in \autoref{fig:matradmatlabr80bixelshistogramshifted}.
\begin{figure}[h!]
    \centering
    \includegraphics[width=\textwidth]{../Screenshots/statistics/initialmatlabvsmatradhistogramshifted.png}
    \caption{Histogram for quantitatively analyzing difference between stopping positions calculated using the developed proton transfer algorithm and stopping positions extracted from matRad using initial matRad starting positions and energy values. A magnetic field ($B_z = \symup{1.5 \, \, \mathrm{T}}$) is introduced in the developed proton transfer algorithm. Median (red) and mean (green) values are highlighted.}
    \label{fig:matradmatlabr80bixelshistogramshifted}
\end{figure}
~\\The introduction of a magnetic field results in larger distances between the proton stopping positions calculated using the developed proton transfer algorithm and those extracted from matRad. The median distance is \( \symup{18.23 \, \, \mathrm{mm}} \), while the mean distance is \( \symup{18.38 \, \, \mathrm{mm}} \).
\newpage
\subsection{Optimization of deflected spots}
\label{sec:matradgradientdescent}
%The goal is to optimize and recalculate the spots that are shifted due to the introduction of a magnetic field, so that they converge to the initial positions. To achieve this goal, the previously explained gradient descent algorithm is applied.
%~\\
%
Optimized proton stopping positions, calculated using the developed proton transfer algorithm with the introduction of a magnetic field (\( B_z = \symup{1.5 \, \, T} \)), are compared with the initial matRad stopping positions without a magnetic field, as visualized in \autoref{fig:gradientdescentspotsplus}.
\begin{figure}[h!]
    \centering
    \includegraphics[width=\textwidth]{../Screenshots/matrad/gradientandmatradplus.png}
    \caption{Visualization of the enlarged prostate patient \gls{ct} dataset with highlighted \gls{ctv} (white). Qualitative comparison between magnetic field influenced and optimized spots calculated using the developed proton transfer algorithm (red dots) and matRad extracted proton stopping positions (blue dots) for each pencil beam. A candidate spot without optimization (orange dot) with its updated position is shown (orange arrow).}
    \label{fig:gradientdescentspotsplus}
\end{figure}
~\\Qualitatively, the optimization shows that the deflected proton stopping positions calculated using the developed proton transfer algorithm are shifted around a position approximately matching the initial matRad proton stopping positions.
\newpage
The following \autoref{fig:matradmatlabr80bixelshistogramshiftedoptimized} presents the quantitative analysis of the optimized spots in a histogram, comparing this dataset to the previously calculated stopping positions using the developed proton transfer algorithm without a magnetic field (see \autoref{sec:comparison}).
\begin{figure}[h!]
    \centering
    \includegraphics[width=\textwidth]{../Screenshots/statistics/initialmatlabvsmatradhistogramshiftedoptimized.png}
    \caption{Histogram for quantitatively analyzing the differences between optimized stopping positions calculated using the developed proton transfer algorithm with an introduced magnetic field (\(B_z = \symup{1.5 \, \, \mathrm{T}} \)) and proton stopping positions extracted from matRad (blue bars). For comparison, initial stopping positions calculated using the developed proton transfer algorithm without a magnetic field are shown (gray bars). Euclidean distances are used, with their frequency distribution displayed. Median (red) and mean (green) values for both datasets are highlighted.}
    \label{fig:matradmatlabr80bixelshistogramshiftedoptimized}
\end{figure}
~\\After optimization, the majority of distances between the optimized stopping positions calculated using the developed proton transfer algorithm and proton stopping positions extracted from matRad are below \( \symup{2 \, \, mm} \), as indicated by the blue bars. In contrast, the initial distances between stopping positions calculated using the developed proton transfer algorithm and those extracted from matRad, without the introduction of a magnetic field, were predominantly below \( \symup{1 \, \, mm} \) (cf. \autoref{fig:matradmatlabr80bixelshistogram}). After introducing a magnetic field and performing optimization, the mean distance increased to \( \symup{1.47 \, \, mm} \), with the median distance at \( \symup{1.06 \, \, mm} \).
\chapter{Discussion}
\label{sec:discussion}
In order to achieve a possible magnetic field influenced pencil beam spot selection for the \gls{tps} matRad, several assumptions and approximations were made. In addition, simplified settings and initializations were made. In this chapter, the results presented so far are discussed and possible limitations are mentioned.
~\\

\section{Verification of proton stopping position}
\label{sec:protontrajectorydiscussion}
\subsection{Vacuum state}
\label{sec:vacuumdiscussion}
In the analysis of a vacuum state, the calculated proton trajectory was compared to the analytical results derived using \autoref{eqn:radiusmagnet}. Systematic uncertainties may arise from inaccuracies in the variables used for calculations. For the four candidate energy values analyzed, as shown in \autoref{tab:compareMATLABanalytical}, the radii computed by the developed proton transfer algorithm were consistently slightly larger than the corresponding analytical results from \autoref{eqn:radiusmagnet}. The difference between $r_{\symup{ana}}$ and $r_{\symup{eq}}$ decreased with increasing energy values and converged to a constant relative difference as the magnetic field strength varied.
~\\

A possible explanation for this observation is that the developed proton transfer vacuum calculation assumes a small energy loss of $\symup{\frac{\mathrm{d}E}{\mathrm{d}x} = 10^{-14} \, \, \frac{\mathrm{keV}}{\mu \mathrm{m}}}$ to approximate vacuum conditions. In an ideal vacuum, no energy loss would occur. Consequently, the analytical results from \autoref{eqn:radiusmagnet} are likely closer to the true values than those obtained using the developed proton transfer algorithm. However, the magnitude of this error in the developed algorithm is minimal. Testing with $\symup{\frac{\mathrm{d}E}{\mathrm{d}x} = 10^{-9} \, \, \frac{\mathrm{keV}}{\mu \mathrm{m}}}$ yielded identical results.
~\\

Voxel-wise spatial parameterization introduces additional systematic uncertainties. While the analytical calculation assumes an infinitesimally small grid size, this thesis used a grid size of $\symup{1.09375 \, \, \mathrm{mm}}$, limiting the resolution for results below this threshold. Updating the proton position only at the start of each voxel may further contribute to the observed discrepancies. This uncertainty applies to all subsequent cases analyzed.
\subsection{Water phantom}
\label{sec:waterphantomdiscussion}
\subsubsection{B = 0 T}
\label{sec:zeromagneticfielddiscussionwater}
For this analysis, as well as the analyses of the bone phantom and prostate patient \gls{ct} dataset, the results of the developed proton transfer algorithm were compared with \gls{mc} simulations in \gls{topas}. This introduces the possibility of statistical uncertainties, particularly since all \gls{topas} simulations used a constant number of initial protons ($\symup{N = 10^5}$) (see \autoref{sec:parametersetup}). For the water phantom without a magnetic field, further simulations with $\symup{N = 10^7}$ were conducted. These showed a proton range difference of $\symup{0.01 \, \, mm}$. Thus, the error magnitude from using $\symup{N = 10^5}$ is relatively low. All subsequent statements assume that \gls{topas} serves as the reference for validating the results of the developed proton transfer algorithm. Result discrepancies may also stem from uncertain systematic inputs in \gls{topas}.
~\\

For $\symup{B = 0 \, \, \mathrm{T}}$, the assumption $\symup{CSDA} \approx R_{\symup{80}}$ was applied to compare analytical proton stopping positions with \gls{mc} based stopping positions in \gls{topas}, using the percentage depth dose curve (\autoref{eqn:csdar80}). Results in \autoref{tab:MATLABTOPASwaterB0} indicate that absolute and relative differences between the analytical and \gls{mc} proton ranges decrease with increasing energy. Notably, the relative difference converges toward $\symup{0 \, \, \%}$ at $E = \symup{200 \, \, MeV}$. Additional simulations at $E = \symup{225 \, \, MeV}$ confirmed a similar convergence near $\symup{0 \, \, \%}$, with only a $\symup{0.0088 \, \, \%}$ increase in relative difference.
~\\

Differences between proton ranges calculated using the developed proton transfer algorithm and \gls{mc} simulations may arise from systematic uncertainties due to the grid size resolution in the developed algorithm. Additionally, while the Schneider \gls{hulut} implementation in \gls{topas} is assumed to match that used in the developed algorithm, subtle differences could affect stopping power calculations and, consequently, proton stopping positions. Variations in underlying physics models also contribute. For instance, greater scattering effects modeled in \gls{topas} may shorten proton ranges. The default physics models in \gls{topas} include six modules \cite{perl_default_nodate}, which account for processes such as particle decay and elastic, inelastic, and capture interactions \cite{noauthor_decay_nodate,noauthor_qgsp_bic_nodate}. In contrast, the Bethe Bloch implementation in "libamtrack" used in this thesis focuses primarily on electromagnetic processes \cite{grzanka_libamtrack_nodate}.
~\\

The \gls{topas} beam model, briefly described in \autoref{sec:parametersetup} and detailed in appendix \ref{sec:appendixtopassetup}, introduces additional systematic uncertainties. A simplified beam model was employed to approximate the developed narrow proton transfer beam. Further analysis of different beam settings may reduce discrepancies between the \gls{mc}-simulated proton ranges in \gls{topas} and the analytically calculated ranges from the developed proton transfer algorithm.
~\\

Another source of systematic uncertainty arises from the positioning of the beam source relative to the width of the irradiated \gls{ct} dataset. For the water phantom, bone phantom, and prostate patient, 62 slices with a slice thickness of $\symup{0.2 \, \, \mathrm{cm}}$ were imported, yielding a total width of $\symup{12.4 \, \, \mathrm{cm}}$. In this thesis, it is assumed that the current beam setup fully covers the entire phantom geometry. Investigating this factor, along with alternative beam configurations, could uncover smaller absolute and relative differences between proton ranges calculated analytically using the developed proton transfer algorithm and those derived from the \gls{mc} platform \gls{topas}. These uncertainties are expected to persist across all subsequent \gls{topas} simulations.
\subsubsection{B > 0 T}
In this thesis, primarily homogeneous magnetic fields were investigated. However, in \gls{mript}, proton pencil beam delivery is also influenced by the heterogeneous fringe fields of an MR scanner. Machine-specific look-up tables could incorporate the effects of such fringe fields \cite{duetschler_fast_2023}.

~\\For both trajectory lengths, listed in \autoref{tab:compareMATLABTOPASdistancewater}, and proton stopping positions, shown in \autoref{tab:compareMATLABTOPAScoordinateswater}, the absolute and relative differences between results from the developed proton transfer algorithm and \gls{mc} simulations appear energy dependent. Trajectories and stopping positions were analyzed using a fixed magnetic field strength with varying energy values and a fixed energy value with varying magnetic field strengths.

~\\From \autoref{tab:compareMATLABTOPAScoordinateswater}, it can be observed that for $B_z = 1.5 \, \, \mathrm{T}$ and varying energy values, the Euclidean distances between positions calculated in the developed proton transfer algorithm and with \gls{mc} simulations decrease up until the analysis of $E = 200 \, \, \mathrm{MeV}$. When energy is fixed and magnetic field strength varies, no significant changes in distance values are observed, indicating an energy dependence. This trend could be attributed to systematic uncertainties in the developed proton transfer algorithm. Additionally, the proposed method for calculating proton stopping positions in magnetic fields may introduce further uncertainties. Exploring alternative methods based on \gls{mc} simulated pencil beam trajectories influenced by magnetic fields could improve accuracy.
\subsection{Bone phantom}
\label{sec:bonephantomdiscussion}
\subsubsection{B = 0 T}
\label{sec:zeromagneticfielddiscussionbone}
For the bone phantom without a magnetic field, proton ranges are listed in \autoref{tab:MATLABTOPASboneB0}. The ranges calculated using the developed proton transfer algorithm are consistently larger than those obtained from \gls{mc} simulations. Absolute differences increase from $\symup{0.81 \, \, \mathrm{mm}}$ at $E = \symup{50 \, \, \mathrm{MeV}}$ to $\symup{7.09 \, \, \mathrm{mm}}$ at $E = \symup{200 \, \, \mathrm{MeV}}$. Relative differences appear to converge toward a constant value of approximately $\symup{4.76 \, \, \%}$, indicating the presence of systematic uncertainties in proton range calculations.
~\\

A major source of these uncertainties could be differences in how \gls{ct} numbers, here $\symup{1000 \, \, \mathrm{HU}}$, are converted into \gls{rspr} values between the developed proton transfer algorithm and the \gls{mc} simulations. Further analysis of the Schneider \gls{hulut} method and its implementation may confirm or refute this hypothesis. Investigating alternative stopping power calculation methods could also be beneficial. For example, the \gls{mata} approach maps \gls{rspr} values to material properties using 40 predefined material compositions representative of human tissues, with mass density determined through linear interpolation \cite{permatasari_material_2020}.
\subsubsection{B > 0 T}
The absolute trajectory length differences for a fixed magnetic field strength and varying energy values (see \autoref{tab:compareMATLABTOPASdistancebone}) calculated using the developed proton transfer algorithm and \gls{mc} simulations increase with energy. Conversely, for a fixed energy value and varying magnetic field strength, these differences converge to a nearly constant range of $\symup{2.48 \, \, \mathrm{mm}}$ to $\symup{2.73 \, \, \mathrm{mm}}$. Relative differences similarly stabilize around a constant value, as observed in the analysis without a magnetic field. A comparable trend is evident when examining proton stopping positions instead of distances (see \autoref{tab:compareMATLABTOPAScoordinatesbone}).
~\\

In the bone phantom analysis, each voxel is assigned the same \gls{ct} number. For a fixed magnetic field strength and varying energy values, the increasing absolute differences suggest an energy dependent uncertainty. This trend is further supported by the nearly constant absolute differences observed for fixed energy values. These uncertainties appear multifactorial. Similar to the analysis of magnetic field influenced proton stopping positions in the water phantom (see \autoref{sec:waterphantomdiscussion}), the method used to calculate proton stopping positions from trajectory lengths may contribute to the observed discrepancies. Additionally, differences in \gls{rspr} value calculation and other systematic uncertainties within the developed proton transfer algorithm could further amplify the variations in proton stopping positions for fixed energy and magnetic field strength values.
\subsection{Prostate patient}
\label{sec:prostatephantomdiscussion}
\subsubsection{B = 0 T}
\label{sec:zeromagneticfielddiscussionprostate}
The imaging for the investigated prostate patient dataset was performed using \gls{ct}, which directly provides \gls{ct} numbers for conversion into \gls{rspr} values. The availability of the prostate patient dataset and the simplicity of stopping power calculation are the primary reasons for focusing on \gls{ct} slices in this thesis. Since the goal is \gls{mript}, developing a method to convert \gls{mri} datasets into \gls{ct} datasets, potentially in real-time, could represent a significant step toward the realization of \gls{mript}. In this context, synthetic \gls{ct} methods could be explored \cite{boulanger_deep_2021}.
~\\

The results in \autoref{tab:MATLABTOPASprostateB0} show that, with increasing energy values, the absolute difference between proton ranges calculated using the developed proton transfer algorithm and those obtained from \gls{mc} simulations decreases. At a specific energy value ($E = \symup{150 \, \, \mathrm{MeV}}$), the proton ranges derived from \gls{mc} simulations exceed those calculated by the developed proton transfer algorithm. This observation may stem from differences in the stopping power calculation methods employed in the proton transfer algorithm and the \gls{mc} simulations. This systematic uncertainty is particularly evident in the prostate patient dataset, where each voxel has a unique \gls{ct} number. Further investigations using more sophisticated stopping power calculation algorithms, such as the \gls{mata} approach, could help address these discrepancies. Additionally, uncertainties in proton stopping positions, both with and without a magnetic field, are amplified by statistical effects. For example, the proton transfer algorithm computes the beam trajectory using a single proton under idealized conditions, whereas \gls{mc} simulations account for the most probable stopping positions of multiple initial protons.
\newpage
\subsubsection{B > 0 T}
For the trajectory lengths listed in \autoref{tab:compareMATLABTOPASdistanceprostate}, similar observations and explanations can be drawn as for the proton ranges without a magnetic field discussed previously.
~\\

The proton stopping positions calculated using the developed proton transfer algorithm and derived from \gls{mc} simulations, as shown in \autoref{tab:compareMATLABTOPAScoordinatesprostate}, do not exhibit a consistent trend in absolute differences. For a fixed magnetic field strength and increasing energy values, the absolute differences and distances show no clear trend of increase or decrease, supporting the hypothesis of discrepancies in stopping power calculations. Similarly, for a fixed energy value and varying magnetic field strengths, a jump of approximately $\symup{3 \, \, \mathrm{mm}}$ is observed when increasing from $B_z = 1 \, \, \mathrm{T}$ to $B_z = 2 \, \, \mathrm{T}$.
~\\

This discrepancy may be attributed to differences in stopping power calculations, which result in trajectories passing through voxels with different \gls{ct} numbers. For instance, a specific proton state $i$ calculated using the developed algorithm might pass through a voxel with a \gls{ct} number representing air, while the same proton state $i$ in the \gls{mc} simulation might pass through a voxel with a \gls{ct} number representing bone. Consequently, the proton stopping position calculated using the developed algorithm would yield a larger value compared to that derived from the \gls{mc} simulation.
\section{Recalculation of spot selection for a matRad treatment field}
\label{sec:recalculaationspotdiscussion}
\subsection{Proton treatment field}
\label{sec:protontreaatmentfielddiscussion}
As shown in \autoref{fig:treatmentplanmatraddose}, a candidate proton treatment field was generated using matRad. However, this field is unsuitable as a treatment plan due to inadequate coverage of the \gls{ctv} and the potential for overdosing. Additionally, the \glspl{oar} were not considered during its creation. Further analysis is required to evaluate the use of additional beams, potentially from opposing directions, and to define dose constraints for the \glspl{oar}, ensuring clinically relevant dose coverage of the \gls{ctv}. Moreover, a constant proton \gls{rbe} value of 1.1 was assumed in the dose calculations. Recent studies have demonstrated that the \gls{rbe} of protons varies along the particle track, which could influence treatment effectiveness \cite{paganetti_relative_2002}.
\newpage
\subsection{Verifying treatment field beam positions}
The \autoref{fig:matradmatlabr80bixels} suggests a consistent trend within each row. Specifically, an approximately uniform difference is observed between the proton stopping positions calculated using the developed proton transfer algorithm and those extracted from matRad. A possible explanation for row-specific differences is the traversal of different materials in the prostate patient \gls{ct} slice. This highlights the potential for stopping power miscalculations in the developed proton transfer algorithm, discrepancies in stopping power calculation methods between the developed algorithm and matRad, or both.
~\\

Systematic uncertainties in proton stopping position calculations are further emphasized when analyzing the Euclidean distances between stopping positions calculated using the developed algorithm and matRad (see \autoref{fig:matradmatlabr80bixelshistogram}). The median distance of $\symup{0.49 \, \, \mathrm{mm}}$, approximately half the resolution size, supports the presence of systematic uncertainty. Additionally, random errors in stopping position estimation may arise within each voxel in matRad. For this thesis, it was assumed that proton stopping positions calculated using both the developed proton transfer algorithm and matRad occur at the same point within a voxel.
~\\
Further investigation into stopping power calculation methods, grid resolution, and the analysis of stopping positions within a voxel could help reduce these differences and improve the agreement between the developed proton transfer algorithm and matRad.

\subsection{Calculation of deflected spots}
For the calculation of deflected spots, only a single magnetic field strength ($B_z = \symup{1.5 \, \, \mathrm{T}}$) was analyzed. Consequently, the results concerning the order of magnitude of the distance between magnetic field influenced pencil beam spots calculated using the developed proton transfer algorithm and spots extracted from matRad without a magnetic field allow for only a single hypothesis. At the analyzed magnetic field strength, an approximate distance of $\symup{18 \, \, \mathrm{mm}}$ is observed (see \autoref{fig:matradmatlabr80bixelshistogramshifted}). This significant shift in spot positioning, also visualized in \autoref{fig:matradmatlabr80bixelsshifted}, intuitively suggests the need for optimization.
~\\

Further analysis incorporating additional magnetic field strengths and advanced statistical approaches is required to confirm or refute the hypothesis that optimization of pencil beam spot positioning is necessary to account for the effects of the magnetic field.
\newpage
\subsection{Optimization of deflected spots}
When analyzing \autoref{fig:gradientdescentspotsplus}, optimized spots calculated using the developed proton transfer algorithm in the middle row appear to converge more closely to matRad spots than those in the first and last rows. Ideally, the distances between optimized stopping positions under a magnetic field and those without a magnetic field should converge. However, \autoref{fig:matradmatlabr80bixelshistogramshiftedoptimized} shows that the median distance of optimized spots under a magnetic field is approximately double that of spots without a magnetic field. Specifically, the median value increased from $\symup{0.49 \, \, \mathrm{mm}}$ to $\symup{1.06 \, \, \mathrm{mm}}$. This discrepancy likely arises from multiple factors.
~\\

First, the introduced magnetic field deflects spots in both the $x$ and $y$ directions, adding an extra dimension of uncertainty to the pre-existing systematic uncertainties in spot positioning. Second, the gradient descent optimization method introduces its own uncertainties, as it depends on parameters such as the learning rate, convergence criteria, and maximum iterations. The settings used (see \autoref{sec:gradientdescentmaterial}) may have been insufficient for achieving optimal results due to computational constraints. The choice of the learning rate is critical. If the learning rate is too large, the algorithm may diverge, while if too small, convergence can be slow \cite{boyd_convex_2004}. Additionally, the optimization was applied to a single \gls{ct} slice, limiting its generalizability.
~\\

The observed increase in distances may result from a combination of uncertainties in the proton stopping position calculations by the developed algorithm and those introduced by suboptimal optimization of pencil beam spots.
~\\

To improve optimization under a magnetic field, the stopping position calculations in the developed algorithm should first be re-evaluated. Further analysis of the optimization algorithm and its parameters is also necessary. Expanding the analysis to multiple \gls{ct} slices could provide further validation of the current results. Additionally, exploring alternative optimization techniques could reduce computational costs while enhancing accuracy.
~\\

A potential improvement involves utilizing the same gradient descent algorithm with modifications in spot selection to reduce computational expense, as introduced in the following chapter.

\chapter{Conclusion and outlook}
In this thesis, a method for analytically calculating magnetic field influenced pencil beam spots was successfully proposed and validated as a foundation for further integration within a research \gls{tps}. 
~\\

The vacuum state analysis demonstrated converging constant relative differences between the analytical calculations and the developed proton transfer algorithm. Among the phantoms analyzed, the water phantom exhibited the lowest differences between proton ranges calculated using the developed algorithm and \gls{mc} simulations. Conversely, the bone phantom showed the largest discrepancies, while the prostate patient phantom revealed no consistent trend in differences between stopping positions calculated with the developed algorithm and \gls{mc} simulations for magnetic field influenced pencil beams. 
~\\

The analysis identified several sources of uncertainty in the developed proton transfer algorithm, with stopping power calculations emerging as the most significant. To enhance stopping position accuracy, implementing the \gls{mata} method and recalculating all simulations is recommended. Further investigation into the resolution of the developed algorithm also appears promising. Additionally, the precise determination of stopping positions within a voxel warrants further study to ensure maximal accuracy.
~\\

The method of estimating proton stopping positions under magnetic fields based on trajectory lengths, assumed to be accurate in this thesis, requires further validation. All calculations performed in this study were limited to a single \gls{ct} slice. Expanding the analysis to volumetric data would generalize the algorithm for \gls{3D} systems, which is a crucial step for its integration into a clinical \gls{tps}. Moreover, the analysis was restricted to a single irradiation starting position. Evaluating multiple and varied starting positions would provide a more comprehensive validation of the algorithm. 
~\\Despite the identified systematic uncertainties and the need for further refinement, the current method provides an approximate estimation of proton stopping positions. For the prostate patient analysis, the method demonstrated an approximate error of 2 to 3 mm, which underscores its potential for application with further optimization.
\newpage
Furthermore, for a candidate proton treatment field in the analyzed \gls{tps} matRad, proton stopping positions, along with the required variables for their calculation, were successfully exported and integrated into the developed proton transfer algorithm.
~\\

The analysis of initial proton stopping positions, without optimization or an introduced magnetic field, yielded a median distance of $\symup{0.49 \, \, \mathrm{mm}}$. After optimization and introducing a magnetic field ($B_z = \symup{1.5 \, \, \mathrm{T}}$), the median distance increased to $\symup{1.06 \, \, \mathrm{mm}}$, approximately double the value compared to the case without a magnetic field or optimization.
~\\

The incorporation of inhomogeneous magnetic fields is crucial to improve the realism of calculations and further advance the feasibility of \gls{mript}.
~\\

For further analysis of the \gls{tps} matRad, a clinically relevant treatment plan needs to be explored. Practical objectives and constraints must be established for the \gls{ctv} and the \glspl{oar}, while incorporating multiple beam sources and varied starting positions. Additionally, developing methods to convert \gls{mri} datasets into \gls{ct} datasets is essential if the current \gls{ct} slice analysis is pursued. In this context, synthetic \gls{ct} methods, as previously mentioned, could provide significant advantages.
~\\

Optimization was carried out using a gradient descent algorithm. A critical next step is to investigate the optimization parameters, such as the learning rate, convergence criteria, and maximum iterations, to improve its effectiveness. Moreover, extending the optimization to account for different magnetic field strengths, as well as adapting the method to a \gls{3D} system with multiple beams and \gls{ct} slices, should be prioritized.
~\\

To reduce computational costs, an alternative method for spot selection is recommended. Currently, stopping positions for each pencil beam are optimized individually. By focusing the optimization on an outer layer of pencil beam spots for a given \gls{ctv}, an optimized outer grid could be created for the target. The required dose for the inner layer of the target could then be delivered using standardized pencil beam scanning techniques.
~\\

Finally, after maximizing proton stopping position accuracy and optimizing magnetic field influenced pencil beam spots, the developed proton transfer algorithm must be integrated into the \gls{tps} for clinical application.


\appendix
% Hier beginnt der Anhang, nummeriert in lateinischen Buchstaben
\chapter{Appendix}
\section{Developed proton transfer algorithm}
\label{sec:protontrajectoryappendix}
\subsection{Proton transfer algorithm}
\begin{python}
    classdef ProtonSimulation_half
    properties
        % Constant variables
        mp_MeV = 938.272013;
        mp = 1.67262158e-27; % [kg]
        c = 299792458 * 100; % [mm/s]
        atomic_mass_unit_MeV_c2 = 931.494; % MeV/c^2
        q = 1.602e-19; % Elementary charge

        % Variables one might change
        E0_MeV; % Initial proton energy
        B; % Magnetic field [T]
        initial_position; % Initial position [mm]
        initial_velocity; % Initial velocity [mm/s]
        
        % Arrays to store results
        positions;
        velocities;
        accelerations;
        times;
        energy_losses_adjusted;
        energies; % To store energy at each step
        total_trajectory_length;
        
        % Initial conditions
        grid_step; % [m]
        total_distance; % [mm]
        num_steps;
        SPR_values;
        E_MeV;
        x_distance = 0;
    end
    
    methods
        function obj = ProtonSimulation_half(SPR_values, E0_MeV, B, grid_step, total_distance, initial_position, initial_velocity)
            % Constructor to initialize properties and load SPR values
            obj.SPR_values = SPR_values;
            obj.E0_MeV = E0_MeV;
            obj.B = B;
            obj.grid_step = grid_step;
            obj.total_distance = total_distance;
            obj.initial_position = initial_position;
            obj.initial_velocity = initial_velocity;
            steps = round(obj.total_distance / obj.grid_step);
            obj.num_steps = steps;
            obj.positions = zeros(3, steps);
            obj.velocities = zeros(3, steps);
            obj.accelerations = zeros(3, steps);
            obj.times = zeros(1, steps);
            obj.energy_losses_adjusted = zeros(1, steps);
            obj.energies = zeros(1, steps); % Initialize energies array
            obj.E_MeV = obj.E0_MeV;
            obj.total_trajectory_length = 0; % Initialize total trajectory length
        end
        
        function obj = initializeStep(obj)
            % Initialize first step of velocity and acceleration
            gamma0 = 1 + obj.E0_MeV / obj.atomic_mass_unit_MeV_c2;
            beta0 = sqrt(1 - 1/gamma0^2);
            v0 = beta0 * obj.c;
            v = obj.initial_velocity; % Use the initial velocity input
            F = - obj.q * cross(v, obj.B);
            a = F / obj.mp;
            obj.positions(:, 1) = obj.initial_position; % Use the initial position input
            obj.velocities(:, 1) = v;
            obj.energies(1) = obj.E0_MeV; % Store initial energy
        end
        
        function obj = simulate(obj)
            [num_rows, num_cols] = size(obj.SPR_values);
            
            for step = 2:obj.num_steps
                % Time step calculation based on half the grid step
                half_grid_step = obj.grid_step / 2;
                half_time_step = half_grid_step / norm(obj.velocities(1, step - 1));

                % First half-step to mid-point of the voxel
                F_mid = - obj.q * cross(obj.velocities(:, step - 1), obj.B);
                a_mid = F_mid / obj.mp;

                mid_velocities = obj.velocities(:, step - 1) + a_mid * half_time_step;
                mid_positions = obj.positions(:, step - 1) + obj.velocities(:, step - 1) * half_time_step;

                % Calculate energy loss at the mid-point
                energy_loss_keV_um = obj.calculate_energy_loss(obj.E_MeV);

                if obj.E_MeV < 0.49
                    energy_loss_keV_um = 0;
                    obj.E_MeV = 0;
                    %disp(['Energy dropped below Bethe lower limit, E=0 MeV']);
                    break;
                else
                    x_pos = mid_positions(1) * 10; % Convert to mm
                    y_pos = mid_positions(2) * 10; % Convert to mm
                    x_index = min(max(round(x_pos), 1), num_cols);
                    y_index = min(max(round(y_pos), 1), num_rows);
                    SPR_value = obj.SPR_values(y_index, x_index);
                    energy_loss_keV_um_adjusted = energy_loss_keV_um * SPR_value;
                    obj.energy_losses_adjusted(step) =
                    energy_loss_keV_um_adjusted;
                    obj.E_MeV = obj.E_MeV - energy_loss_keV_um_adjusted * obj.grid_step;
                    obj.energies(step) = obj.E_MeV; % Store energy at current step
                end

                % Update gamma and beta
                gamma = 1 + obj.E_MeV / obj.atomic_mass_unit_MeV_c2;
                beta = sqrt(1 - 1/gamma^2);
                if gamma < 1
                    %disp(['Gamma dropped below 1 at step ', num2str(step)]);
                    break;
                end

                % Complete the second half-step to the end of the voxel
                F_end = - obj.q * cross(mid_velocities, obj.B);
                a_end = F_end / obj.mp;

                obj.velocities(:, step) = mid_velocities + a_end * half_time_step;
                obj.positions(:, step) = mid_positions + mid_velocities * half_time_step;
                obj.accelerations(:, step) = a_end;
                obj.times(step) = obj.times(step - 1) + 2 * half_time_step;

                % Calculate trajectory length increment for this step
                delta_position = obj.positions(:, step) - obj.positions(:, step - 1);
                delta_s = norm(delta_position); % Calculate delta_s using sqrt(Δx^2 + Δy^2 + Δz^2)
                obj.total_trajectory_length = obj.total_trajectory_length + delta_s; % Accumulate trajectory length
            end
        end
        
     function energy_loss_MeV_cm = calculate_energy_loss(obj, E_MeV)
    % constant variables
    me_MeV = 0.510998918; %(see https://github.com/libamtrack/library/blob/master/include/AT_Constants.h#L38 l. 38)
    mp_MeV = 938.272013; %(see https://github.com/libamtrack/library/blob/master/include/AT_Constants.h#L35 l. 35)
    m_MeV = 1.0079 * mp_MeV; %(see https://github.com/libamtrack/library/blob/master/src/AT_PhysicsRoutines.c#L278 l. 280)
    c = 299792458; % Speed of light in m/s
    re = 2.8179403262 * 10^(-15); % electron radius in m
    NA = 6.02214086 * 10^23; % 1/mol
    atomic_mass_unit_MeV_c2 = 931.494; %MeV/c^2
    Dirac_constant_J_s = 1.054571628e-34;
    BETHE_LOWER_LIMIT_E_MEV_U = 0.49;
    phase_undefined = 0;
    phase_condensed = 1;
    phase_gas = 2;
    Z = 1; %for protons

    % variables one might change!
    E_restricted_keV = 1000;
    I_eV = 78; % 75 eV for liquid water / 78 in TOPAS. for others,
    %see https://github.com/libamtrack/library/blob/
    %44dc48cfa977c05008ad12646d798d6c4b6ea504/include/AT_DataMaterial.h l. 197-203
    AT = 18.0006; % mass number for H2O
    ZT = 10; % ordinary number for target: water (H2O: 2 * 1 + 1 * 8 = 10)
    phase = phase_condensed; %for liquid water! for others, see https://github.com/libamtrack/library/blob/master/include/AT_DataMaterial.h#L246
                                                               %lines 240-253 for different materials!

    % simple calculations and conversion factors
    gamma = 1 + E_MeV / atomic_mass_unit_MeV_c2; %(see https://github.com/libamtrack/library/blob/master/src/AT_PhysicsRoutines.c#L45 lines 65-82)
    assert(gamma >= 1.0);

    beta = sqrt(1 - 1/gamma^2); %(see https://github.com/libamtrack/library/blob/master/src/AT_PhysicsRoutines.c#L45 line 75)
    mass_correction_term = 1 + (2 * (me_MeV / m_MeV) / gamma) + (me_MeV / m_MeV)^2; %(see https://github.com/libamtrack/library/blob/master/src/AT_PhysicsRoutines.c#L278 l. 281)
    Wm_MeV = 2 * me_MeV * beta^2 / (1 - beta^2) / mass_correction_term; %(see https://github.com/libamtrack/library/blob/master/src/AT_PhysicsRoutines.c#L285 l.288)
    beta2 = beta^2;
    I_MeV = I_eV * 1e-6;
    m_to_cm = 100;
    MeV_to_J = 1.60217646e-13;
    I_J = I_MeV * MeV_to_J;

    % Stopping power

    % Restricted stopping number requested?
    restricted = false;
    if E_restricted_keV > 0.0 && (E_restricted_keV / 1000.0) < Wm_MeV
        restricted = true;
    end

    % First part of stopping number
    SN11 = (2.0 * me_MeV * beta2) / (1.0 - beta2);
    assert(I_MeV > 0);
    SN11 = SN11 / I_MeV;

    if restricted
        Wm_MeV = E_restricted_keV * 1e-3;
    end
    SN12 = Wm_MeV / I_MeV;

    % Second part of stopping number
    SN2 = beta2;
    if restricted
        SN2 = SN2 / 2;
        SN2 = (SN2 + (1.0 - beta2) * Wm_MeV) / (4.0 * me_MeV);
    end

    % Third part of stopping number (density correction following Sternheimer, 1971)
    delta = 0.0;
    if phase ~= phase_undefined
        gamma_single =  1.0 + E_MeV / atomic_mass_unit_MeV_c2;
        assert(gamma_single >= 1.0);

        assert(E_MeV >= 0);
        beta_single = sqrt(1 - (1/gamma_single^2));

        assert(beta_single * gamma_single > 0);
        kinetic_variable_single = log10(beta_single*gamma_single);
        kinetic_variable = kinetic_variable_single;

        rho_gcm3 = 1;
        numbers_of_atoms_per_g = NA / AT;
        numbers_of_electrons_per_g = numbers_of_atoms_per_g * ZT;
        electron_density_per_cm3 = numbers_of_electrons_per_g * rho_gcm3;
        electron_density_m3_single = electron_density_per_cm3 * m_to_cm * m_to_cm * m_to_cm;
        electron_density_m3 = electron_density_m3_single;

        %https://github.com/libamtrack/library/blob/master/src/AT_DataMaterial.c#L294
        %line 307
        plasma_energy_J = sqrt((4 * pi * electron_density_m3 * re) * Dirac_constant_J_s * c);

        C = 1.0 + 2.0 * log(I_J / plasma_energy_J);

        % Find x_0 and x_1 dependent on phase, I-value, and C
        if phase == phase_condensed
            if I_eV < 100
                x_1 = 2.0;
                if C <= 3.681
                    x_0 = 0.2;
                else
                    x_0 = 0.326 * C - 1.0;
                end
            else % I_eV >= 100
                x_1 = 3.0;
                if C <= 5.215
                    x_0 = 0.2;
                else
                    x_0 = 0.326 * C - 1.5;
                end
            end
        else % gaseous material
            x_0 = 0.326 * C - 2.5;
            x_1 = 5.0;
            if C < 10.0
                x_0 = 1.6;
                x_1 = 4.0;
            end
            if C >= 10.0 && C < 10.5
                x_0 = 1.7;
                x_1 = 4.0;
            end
            if C >= 10.5 && C < 11.0
                x_0 = 1.8;
                x_1 = 4.0;
            end
            if C >= 11.0 && C < 11.5
                x_0 = 1.9;
                x_1 = 4.0;
            end
            if C >= 11.5 && C < 12.25
                x_0 = 2.0;
                x_1 = 4.0;
            end
            if C >= 12.25 && C < 13.804
                x_0 = 2.0;
                x_1 = 5.0;
            end
        end

        x_a = C / 4.606;
        m = 3.0;
        a = 4.606 * (x_a - x_0) / ((x_1 - x_0)^m);

        if kinetic_variable >= x_0 && kinetic_variable <= x_1
            delta = 4.606 * kinetic_variable - C + a * (x_1 - kinetic_variable)^m;
        end
        if kinetic_variable > x_1
            delta = 4.606 * kinetic_variable - C;
        end
    end
    SN3 = delta;

    % Forth part of stopping number (shell correction) TODO: implement

    %assert(SN11 > 0);
    %assert(SN12 > 0);

    stopping_number = 0.5 * log(SN11 * SN12) - SN2 - SN3;

    % Leading energy loss term
    %(we assume that effective charge is used! otherwise, 
    %change code with the help of
    %https://github.com/libamtrack/library/blob/
    %44dc48cfa977c05008ad12646d798d6c4b6ea504/src/AT_StoppingPowerDataBethe.c#L49 l. 62-65)

    assert(AT > 0);
    assert(beta2 > 0);

    % calculation of effective charge according to Barkas-Bethe approximation
    if Z ~= 1
        effective_charge = Z * (1.0 - exp(-125.0 * beta / (Z^(2.0/3.0))));
    else
        effective_charge = 1.0 - exp(-125.0 * beta);
    end

    %since we only use effective charge we can say z = Z;
    z = Z;

    % ICRU49, p.6, after Cohen and Taylor (1986), k_MeV_cm2_g = 0.307075
    energy_loss_leading_term_MeV_cm2_g = 0.307075 * (ZT / AT) * (z^2) / beta2;

    
    % Energy loss
    % Compute only above 1.0 MeV, otherwise theory is too wrong below return zero
    % TODO: Find smarter criterion because this may cause problems in the code (as it did
    % TODO: with the inappropriately set lower limit for CSDA range integration (was 0, now 1.0 MeV)
    
    if E_MeV >= BETHE_LOWER_LIMIT_E_MEV_U
        energy_loss_MeV_cm = energy_loss_leading_term_MeV_cm2_g * stopping_number;
    else
       energy_loss_MeV_cm = 0;
    end

    % unit conversion
    energy_loss_keV_um = energy_loss_MeV_cm / 10; % [keV/μm] AND [MeV/mm]
    energy_loss_MeV_m = energy_loss_MeV_cm / 100; % [MeV/m]
end
        % Function for saving results
        function saveResults(obj)
            non_zero_mask = any(obj.positions ~= 0, 1);
            filtered_positions = obj.positions(:, non_zero_mask);
            filtered_energyloss = obj.energy_losses_adjusted(:, non_zero_mask);
            csvwrite('trajectory.csv', filtered_positions');
            disp(['Total trajectory length s: ', num2str(obj.total_trajectory_length), ' mm']);
        end
        
        function energy = getEnergyAtStep(obj, step)
            % Method to retrieve energy at a given step
            energy = obj.energies(step);
        end

        % Function for storing and saving step range
        function num_non_zero_steps = displayStepRange(obj)
            non_zero_mask = any(obj.positions ~= 0, 1);
            num_non_zero_steps = sum(non_zero_mask);
        end
    end
end
\end{python}

\subsection{Initialization of proton transfer algorithm}
\begin{python}
y_position = 11.2; % Example initial y-position
E0_MeV = 100;  % Example initial energy in MeV
magnetic_field = [0; 0; 3]; % Example magnetic field in T
SPR = readmatrix('SPR_values.csv'); 
% Load SPR values SPR_values(_slice).csv watertestSPR.csv bone_SPR.csv heavybone_SPR.csv

final_pos = analyzeProtonTrajectory(y_position, E0_MeV, magnetic_field, SPR);
% Display last x-position
disp([num2str(final_pos(1)')]);

% New Function to Analyze Proton Trajectory
function final_position = analyzeProtonTrajectory(y, E, B, SPR)
    % Constants (needed for velocity calculation for example)
    atomic_mass_unit_MeV_c2 = 931.494; % MeV/c^2
    c = 299792458 * 100; % [mm/s]
    
    grid_step = 0.00109375; % [m] This is the CT grid step /"PixelSpacing"  one can aquire by looking at metadata using 3D Slicer for example
    total_distance = 83.91796875; % [mm] CT rows * CT columns * CT grid step
    
    % Pre-calculations for initial velocity
    gamma0 = 1 + E / atomic_mass_unit_MeV_c2;
    beta0 = sqrt(1 - 1/gamma0^2);
    v0 = beta0 * c;
    
    % Set initial conditions based on input parameters
    initial_position = [0; y; 0]; % Initial y-position
    initial_velocity = [v0; 0; 0]; % Initial velocity
    
    % Create an instance of ProtonSimulation
    protonSim = ProtonSimulation_half(SPR, E, B, grid_step, total_distance, initial_position, initial_velocity);
    
    % Initialize and run the simulation
    protonSim = protonSim.initializeStep();
    protonSim = protonSim.simulate();
    protonSim.saveResults();
    

    % Display the number of steps with non-zero values
    % Get the number of steps with non-zero values
    num_non_zero_steps = protonSim.displayStepRange();
    
    % Get the final position using the number of non-zero steps
    if num_non_zero_steps > 0
        final_position = protonSim.positions(:, num_non_zero_steps); % Extract the last valid position
        %disp(['Final valid position (x, y, z): ', num2str(final_position')]);
    else
        %disp('No valid positions found.');
    end
end
\end{python}

\section{TOPAS}
\subsection{Simulation setup}
\label{sec:appendixtopassetup}
\begin{python}
# General settings

i:Ts/NumberOfThreads = 0 # Max CPU threads
includeFile = HUtoMaterialSchneider.txt # Import Schneider HULUT

# Proton source positioning
s:Ge/BeamPosition/Parent="World"
s:Ge/BeamPosition/Type="Group"
d:Ge/BeamPosition/TransX= -18.6484375 cm
d:Ge/BeamPosition/TransY= 0. cm
d:Ge/BeamPosition/TransZ= 0. cm
d:Ge/BeamPosition/RotX=90. deg
d:Ge/BeamPosition/RotY=270. deg
d:Ge/BeamPosition/RotZ=0. deg

# Proton beam model
s:So/Demo/Type = "Beam"
s:So/Demo/Component = "BeamPosition"
s:So/Demo/BeamParticle = "proton"
d:So/Demo/BeamEnergy = X. MeV # set X to desired energy value
u:So/Demo/BeamEnergySpread = 0
s:So/Demo/BeamPositionDistribution = "Flat"
s:So/Demo/BeamPositionCutoffShape = "Ellipse"
d:So/Demo/BeamPositionCutoffX = 0.5 cm
d:So/Demo/BeamPositionCutoffY = 0.5 cm
d:So/Demo/BeamPositionSpreadX = 0.1 mm
d:So/Demo/BeamPositionSpreadY = 0.1 mm
s:So/Demo/BeamAngularDistribution = "None"
d:So/Demo/BeamAngularCutoffX = 90. deg
d:So/Demo/BeamAngularCutoffY = 90. deg
d:So/Demo/BeamAngularSpreadX = 0.000000032 rad
d:So/Demo/BeamAngularSpreadY = 0.000000032 rad
i:So/Demo/NumberOfHistoriesInRun = 100000

# CT import

s:Ge/Patient/Type                       = "TsDicomPatient"
s:Ge/Patient/Parent                     = "World"
s:Ge/Patient/ImagingtoMaterialConverter = "Schneider"
d:Ge/Patient/RotX   = 0. deg
d:Ge/Patient/RotY   = 180. deg
d:Ge/Patient/RotZ   = 180. deg
s:Ge/Patient/Field 		    	= "DipoleMagnet" 
u:Ge/Patient/MagneticFieldDirectionX	= 0.0
u:Ge/Patient/MagneticFieldDirectionY 	= 0.0
u:Ge/Patient/MagneticFieldDirectionZ 	= 1.0 # for Bz
s:Ge/Patient/MagneticField3DTable 		= "PurgMag3D.TABLE"
d:Ge/Patient/MagneticFieldStrength   	= X.X tesla # set X.X to desired magnetic field strength

# Dose scorer

s:Sc/MyScorer/Quantity                  = "DoseToMedium"
s:Sc/MyScorer/Component                 = "Patient"
s:Sc/MyScorer/OutputFile                = "TOPAS_Beam"
s:Sc/MyScorer/OutputType                = "csv"
s:Sc/MyScorer/IfOutputFileAlreadyExists = "Overwrite"
\end{python}


\subsection{Estimation of proton stopping positions for B > 0 T}
\label{sec:topasb0tappendix}
\begin{python}
import pandas as pd
import numpy as np
from matplotlib import pyplot as plt
from scipy.optimize import curve_fit
from scipy.interpolate import interp1d

# Define Gaussian function for fitting
def gaussian(y, a, mean, sigma):
    return a * np.exp(-0.5 * ((y - mean) / sigma) ** 2)

# Read the csv-file
outputfile_topas = '../Data/TOPAS_prostate_B3_100MeV.csv'
#outputfile_topas = '../TOPAS/TOPAS_Beam.csv'
df = pd.read_csv(outputfile_topas, comment='#', header=None)

# Convert the dataframe to a numpy array
topas_datamatrix = np.array(df)

# Extract depth (x), height (y), and dose values
depth = topas_datamatrix[:, 0]  # depth (x)
y_positions = topas_datamatrix[:, 1]  # y positions (height)
dose = topas_datamatrix[:, 3]  # dose values

# Reshape the dose data into a 2D grid based on 341 x bins, 225 y bins, and 7 z bins
n_x_bins = 341
n_y_bins = 225
n_z_bins = 62

dose_grid = dose.reshape(n_x_bins, n_y_bins, n_z_bins)

# Define x axis for the plot (convert from cm to mm)
x_bin_size = 0.109375 * 10  # mm per bin in x
x_axis = np.arange(n_x_bins) * x_bin_size  # depth in mm

# Sum dose values across all z-bins (integrating across slices)
integrated_dose_z = np.sum(dose_grid, axis=2)

# Integrate the dose along the y-axis (collapse into 1D dose profile along x)
integrated_dose = np.sum(integrated_dose_z, axis=1)

# Normalize the integrated dose by its maximum value
integrated_dose_norm = (integrated_dose / np.max(integrated_dose)) * 100

# Find the peak dose index (where max dose occurs)
peak_idx = np.argmax(integrated_dose_norm)

# Search for the position after the peak where dose crosses below 80% of the peak dose
r80_value = 80  # 80% of the maximum dose
post_peak_dose = integrated_dose_norm[peak_idx:]
crossing_idx = np.where(post_peak_dose <= r80_value)[0][0] + peak_idx

# Get the values at the two points surrounding the R80 crossing
x1, x2 = x_axis[crossing_idx - 1], x_axis[crossing_idx]
y1, y2 = integrated_dose_norm[crossing_idx - 1], integrated_dose_norm[crossing_idx]

# Perform linear interpolation to find the exact x position for R80
r80_x = x1 + (r80_value - y1) * (x2 - x1) / (y2 - y1)

print(f"R80 value at depth: {r80_x:.2f} mm")

# Define y axis (height) in mm for the plot
y_bin_size = 0.109375 * 10  # mm per bin in y
y_axis = np.arange(n_y_bins) * y_bin_size  # height in mm

# Initialize a list to store the (x, y, dose) values
yshift_array = []

# Boolean flag to track when the first Gaussian fit fails
fitting_failed = False

# Iterate over all x_bins
for x_bin in range(n_x_bins):
    if fitting_failed:
        x_value = x_bin * x_bin_size  # Convert x_bin to depth in mm
        yshift_array.append([x_value, 0.0, 0.0])
        continue

    # Extract dose values along the y-axis for the current x_bin (summed over z)
    y_dose_distribution = np.sum(dose_grid[x_bin, :, :], axis=1)  # Sum over all z-bins

    # Check if the dose is all zeros or close to zero
    if np.max(y_dose_distribution) <= 0:
        x_value = x_bin * x_bin_size  # Convert x_bin to depth in mm
        yshift_array.append([x_value, 0.0, 0.0])
        continue

    # Normalize the dose distribution
    y_dose_distribution_norm = (y_dose_distribution / np.max(y_dose_distribution)) * 100

    if np.isnan(y_dose_distribution_norm).any() or np.isinf(y_dose_distribution_norm).any():
        x_value = x_bin * x_bin_size  # Convert x_bin to depth in mm
        yshift_array.append([x_value, 0.0, 0.0])
        continue

    try:
        # Fit a Gaussian function to the normalized y-dose distribution
        popt, _ = curve_fit(gaussian, y_axis, y_dose_distribution_norm, 
                            p0=[np.max(y_dose_distribution_norm), np.mean(y_axis), np.std(y_axis)], 
                            maxfev=4000)

        # Extract the mean (which corresponds to the y position of the Gaussian peak)
        mean_fit = popt[1]

        # Get the absolute dose value at the position where the Gaussian peak occurs
        closest_y_idx = np.argmin(np.abs(y_axis - mean_fit))
        absolute_dose = y_dose_distribution[closest_y_idx]

        # Append the current x_bin (converted to depth), y value (mean_fit), and absolute dose to the array
        x_value = x_bin * x_bin_size  # Convert x_bin to depth in mm
        yshift_array.append([x_value, mean_fit, absolute_dose])

    except RuntimeError:
        # If curve fitting fails, store 0.0 values for this x_bin
        fitting_failed = True
        x_value = x_bin * x_bin_size  # Convert x_bin to depth in mm
        yshift_array.append([x_value, 0.0, 0.0])

# Convert the list to a DataFrame
yshift_df = pd.DataFrame(yshift_array, columns=['x (mm)', 'y (mm)', 'dose (absolute)'])

# Filter out rows where both y and dose are zero
yshift_df_filtered = yshift_df[(yshift_df['y (mm)'] > 0) & (yshift_df['dose (absolute)'] > 0)].copy()

# Calculate cumulative distance 's'
s_values = [0.0]  # Start with s = 0 for the first point
for i in range(1, len(yshift_df_filtered)):
    delta_x = yshift_df_filtered['x (mm)'].iloc[i] - yshift_df_filtered['x (mm)'].iloc[i - 1]
    delta_y = yshift_df_filtered['y (mm)'].iloc[i] - yshift_df_filtered['y (mm)'].iloc[i - 1]
    delta_s = np.sqrt(delta_x**2 + delta_y**2)
    s_values.append(s_values[-1] + delta_s)

yshift_df_filtered.loc[:, 's (mm)'] = s_values

# Normalize the dose
yshift_df_filtered.loc[:, 'dose_norm'] = (yshift_df_filtered['dose (absolute)'] / np.max(yshift_df_filtered['dose (absolute)'])) * 100

# Apply a threshold to remove very small dose values (e.g., less than 1%)
dose_threshold = 1.0
threshold_idx = np.where(yshift_df_filtered['dose_norm'] < dose_threshold)[0]

# Keep the next 4 points after the dose drops below the threshold
if len(threshold_idx) > 0:
    first_threshold_idx = threshold_idx[0]
    end_idx = first_threshold_idx + 4  # Keep 4 more points
    yshift_df_filtered = yshift_df_filtered.iloc[:min(end_idx, len(yshift_df_filtered))]


# Toggle to show Gaussian fit for a specific bin
show_gaussian_fit = True  # Set to False to disable

# Specify the bin for Gaussian example (this is where the Gaussian will be plotted)
example_bin = 20  # You can change this bin as needed

# Plot the Gaussian fit for a specific bin if requested
if show_gaussian_fit:
    # Extract dose values along the y-axis for the chosen x_bin (summed over z)
    y_dose_distribution_example = np.sum(dose_grid[example_bin, :, :], axis=1)  # Sum over z-bins
    
    # Normalize the dose distribution
    y_dose_distribution_example_norm = (y_dose_distribution_example / np.max(y_dose_distribution_example)) * 100
    
    # Fit a Gaussian function
    popt, _ = curve_fit(gaussian, y_axis, y_dose_distribution_example_norm, 
                        p0=[np.max(y_dose_distribution_example_norm), np.mean(y_axis), np.std(y_axis)], 
                        maxfev=4000)
    
    # Extract fitted Gaussian
    gaussian_fit_example = gaussian(y_axis, *popt)

    # Plot Gaussian fit
    plt.figure(figsize=(12, 8))
    plt.plot(y_axis, y_dose_distribution_example_norm, label='Normalized Dose Distribution', color='blue')
    plt.plot(y_axis, gaussian_fit_example, label='Gaussian Fit', color='orange', linestyle='--')
    plt.xlabel('Y Position [mm]')
    plt.ylabel('Normalized Dose [%]')
    plt.title(f'Gaussian fit for x bin {example_bin}')
    plt.legend()
    plt.grid(True)
    plt.show()


# Plot the depth-dose distribution for all z-bins (summarized result)
plt.figure(figsize=(12, 8))

# Plot the depth-dose distribution averaged across all z-bins
plt.plot(x_axis, integrated_dose_norm, color='blue')

# Add a horizontal line for 80% of the max dose
plt.axhline(y=r80_value, color='gray', linestyle='--', label='80% of Max. Dose')

# Add a red dot at the R80 position
plt.plot(r80_x, r80_value, 'ro', label=f'R$_{{80}}$ = {r80_x:.2f} mm')  # Add label to legend

# Add labels and title
plt.xlabel('Depth [mm]')
plt.ylabel('Relative Dose [%]')
plt.title('Percentage depth dose curve for protons (N = $10^5$, E = 200 MeV, B = 0 T) in water')
plt.legend()
plt.grid(True)

# Show the depth-dose distribution plot
plt.show()




# Interpolation to find the dose at the calculated R80 value
r80_s = r80_x  # Use the calculated R80 value
s_values = yshift_df_filtered['s (mm)']
dose_values = yshift_df_filtered['dose_norm']

# Create interpolation function
interp_function = interp1d(s_values, dose_values, bounds_error=False, fill_value="extrapolate")

# Get interpolated dose at r80_s
interpolated_dose = interp_function(r80_s)

# Plot the dose vs distance 's' (mm)
plt.figure(figsize=(12, 8))
plt.plot(yshift_df_filtered['s (mm)'], yshift_df_filtered['dose_norm'])
plt.xlabel('Distance (s) [mm]')
plt.ylabel('Relative Dose [%]')
plt.grid(True)

# Plot the interpolated R80 point
plt.plot(r80_s, interpolated_dose, 'ro', label=f'Dose = {interpolated_dose:.2f}% (R$_{{80}}$ = {r80_s:.2f} mm)')
#plt.plot(r80_s, interpolated_dose, 'ro', label=f'R$_{{80}}$ = {r80_s:.2f} mm (Dose = {interpolated_dose:.2f}%)')
plt.title('Distance (s) vs. Relative Dose: Searching dose')
plt.legend()

# Show the plot
plt.show()

# Search for `s` value based on input relative dose
search_rel_dose = True  # Toggle on or off

# Allow user to input the desired relative dose % (default: 83.69%)
target_rel_dose = 83.44 #interpolated_dose # ENTER HERE HOW MUCH r80 we had for B = 0 T !!!

# Find the maximum dose and its corresponding index
max_dose_idx = np.argmax(yshift_df_filtered['dose_norm'])
max_dose_value = yshift_df_filtered['dose_norm'].iloc[max_dose_idx]

# Now, slice the data after the maximum dose
falling_side_df = yshift_df_filtered.iloc[max_dose_idx:]  # Data after reaching 100%

# Check if the target relative dose exists on the falling side
if target_rel_dose <= max_dose_value:
    # Interpolation function to find `s` for the given relative dose, only after the peak
    interp_s_for_dose_falling = 
    interp1d(falling_side_df['dose_norm'], 
    falling_side_df['s (mm)'], 
    kind='linear',
    bounds_error=False,
    fill_value="extrapolate")
    
    # Get the corresponding `s` value for the target relative dose (83.69%)
    target_s_value_falling = interp_s_for_dose_falling(target_rel_dose)
    
    # Plot the dose vs distance 's' without the R80 point, but with the searched target dose point
    plt.figure(figsize=(12, 8))
    plt.plot(yshift_df_filtered['s (mm)'], yshift_df_filtered['dose_norm'], color='blue')
    
    # Plot the interpolated point at the target dose
    plt.plot(target_s_value_falling, target_rel_dose, 'ro', label=f's = {target_s_value_falling:.2f} mm for rel. dose = {target_rel_dose:.2f} %')
    
    plt.xlabel('Distance (s) [mm]')
    plt.ylabel('Relative Dose [%]')
    plt.title(f'Distance (s) vs. Relative Dose: Searching distance')
    plt.legend()
    plt.grid(True)
    plt.show()
else:
    print(f"Target relative dose {target_rel_dose}% exceeds the maximum dose.")

# Define the midpoint for the y-axis (example: 122.5 mm for B = 0 T)
y_midpoint = 122.5
    
# Find the two surrounding s-values for interpolation
s_values = yshift_df_filtered['s (mm)']
s_before_idx = np.where(s_values <= target_s_value_falling)[0][-1]  # closest s below target
s_after_idx = np.where(s_values >= target_s_value_falling)[0][0]    # closest s above target

# Get corresponding x and y values
x1, x2 = yshift_df_filtered['x (mm)'].iloc[s_before_idx], yshift_df_filtered['x (mm)'].iloc[s_after_idx]
y1, y2 = yshift_df_filtered['y (mm)'].iloc[s_before_idx], yshift_df_filtered['y (mm)'].iloc[s_after_idx]
s1, s2 = s_values.iloc[s_before_idx], s_values.iloc[s_after_idx]

# Perform linear interpolation to find exact x and y for the target_s_value_falling
interpolated_x = x1 + (target_s_value_falling - s1) * (x2 - x1) / (s2 - s1)
interpolated_y = y1 + (target_s_value_falling - s1) * (y2 - y1) / (s2 - s1)

# Check if y values need to be inverted based on the mirroring effect
if interpolated_y < y_midpoint:
    interpolated_y = 2 * y_midpoint - interpolated_y  # Flip the y position

print(f"Interpolated x = {interpolated_x:.2f} mm, y = {interpolated_y:.2f} mm for s = {target_s_value_falling:.2f} mm")
\end{python}
\newpage
\section{matRad}
\label{sec:matradappendix}
\subsection{Treatment plan setup}
\label{sec:matradtreatmentplanappendix}
\begin{python}
    %% Patient Data Import

    matRad_rc; %If this throws an error, run it from the parent directory first to set the paths

    
    load('2_04_P_CT_CST.mat'); 

    %% Objective & Constraints

    for i = 1:size(cst, 1)
        if strcmp(cst{i, 2}, 'Prostate')
            cst{i, 3}                        = 'TARGET';
            cst{i, 5}.Priority               = 1; 
            cst{i, 6}{1, 1}.className        = 'DoseObjectives.matRad_MeanDose';
            cst{i, 6}{1, 1}.parameters{1,1}  = 50; 
            cst{i, 6}{1, 1}.penalty          = 1; 
        end
    end

    %% Treatment Plan

    pln.radiationMode = 'protons';        
    pln.machine       = 'Generic';
    pln.propOpt.bioOptimization = 'const_RBExD';
    pln.propDoseCalc.calcLET = 1;

    pln.numOfFractions        = 1; 
    pln.propStf.gantryAngles  = [270];
    pln.propStf.couchAngles   = [0];
    pln.propStf.bixelWidth    = 5;
    pln.propStf.numOfBeams    = numel(pln.propStf.gantryAngles);
    pln.propStf.isoCenter     = ones(pln.propStf.numOfBeams,1) * matRad_getIsoCenter(cst,ct,0);

    csvwrite('isoCenter.csv', pln.propStf.isoCenter');
    disp(['Iso Center positions have been saved as isoCenter.csv!']);

    pln.propOpt.runDAO        = 0;
    pln.propOpt.runSequencing = 0;

    % dose calculation settings
    pln.propDoseCalc.doseGrid.resolution.x = ct.resolution.x; % [mm]
    pln.propDoseCalc.doseGrid.resolution.y = ct.resolution.y; % [mm]
    pln.propDoseCalc.doseGrid.resolution.z = ct.resolution.z; % [mm]

    %% Generate Beam Geometry STF
    stf = matRad_generateStf(ct,cst,pln);

    %% Dose Calculation
    % Lets generate dosimetric information by pre-computing dose influence 
    % matrices for unit beamlet intensities. Having dose influences available 
    % allows for subsequent inverse optimization. 
    dij = matRad_calcParticleDose(ct,stf,pln,cst);

    %% Inverse Optimization for IMPT
    % The goal of the fluence optimization is to find a set of bixel/spot 
    % weights which yield the best possible dose distribution according to the 
    % clinical objectives and constraints underlying the radiation treatment
    resultGUI = matRad_fluenceOptimization(dij,cst,pln);

    %% Plot the Resulting Dose Slice
    % Let's plot the transversal iso-center dose slice
    slice = round(pln.propStf.isoCenter(1,3)./ct.resolution.z);
    figure
    imagesc(resultGUI.RBExDose(:,:,slice)),colorbar,colormap(jet)

    plane = 3;
    doseWindow = [0 max([resultGUI.RBExDose(:);])];

    figure,title('original plan')
    matRad_plotSliceWrapper(gca,ct,cst,1,resultGUI.RBExDose,plane,slice,[],0.75,colorcube,[],doseWindow,[]);
\end{python}

\newpage
\subsection{Estimation of matRad proton stopping positions}
\label{sec:exporting_matRad_R80_positionsappendix}

\begin{python}
% Initialization for storing x, y values, 80% dose values, and energy
x_y_energy_values_at_80_percent = [];

% Counter for the current bixel (across all rays)
current_bixel_id = 1;

% Loop through all rays
for ray_id = 1:length(stf.ray)
    % Get the x and y position of the current ray relative to the isocenter
    ray_y_pos = stf.ray(ray_id).rayPos(2);
    
    % Get the isocenter
    isoCenter_y = stf.isoCenter(2);

    % Actual y position of the ray (relative to the isocenter)
    ray_y_absolute = isoCenter_y + ray_y_pos;

    % Number of bixels for this ray
    num_bixels = stf.numOfBixelsPerRay(ray_id);
    
    % Loop through all bixels in this ray
    for bixel_idx = 1:num_bixels
        % Step 1: Extract the 1D vector for the current bixel
        doseVector = full(dij.physicalDose{1,1}(:, current_bixel_id));
        LETVector = full(dij.mLETDose{1,1}(:, current_bixel_id));

        % Dimensions of the dose grid
        x_dim = dij.doseGrid.dimensions(2); % Number of x coordinates
        y_dim = dij.doseGrid.dimensions(1); % Number of y coordinates
        z_dim = dij.doseGrid.dimensions(3); % Number of z layers

        % Step 2: Convert the 1D vector into a 3D matrix
        doseMatrix = reshape(doseVector, [y_dim, x_dim, z_dim]);
        LETMatrix = reshape(LETVector, [y_dim, x_dim, z_dim]);

        % Step 3: Restrict to the second z layer
        doseMatrix_z2 = doseMatrix(:,:,2); % Dose values in the second z layer
        LETMatrix_z2 = LETMatrix(:,:,2); % LET values in the second z layer

        % Step 4: Sum the dose values along the y-axis for each x coordinate
        depthDoseCurve = sum(doseMatrix_z2, 1); % Sum over y for each x coordinate
        depthLETCurve = sum(LETMatrix_z2, 1); % Sum over y for each x coordinate

        % Step 5: Normalize the dose values to the maximum value (relative dose)
        max_dose = max(depthDoseCurve);
        relativeDepthDoseCurve = depthDoseCurve / max_dose;

        max_LET = max(depthLETCurve);
        relativeDepthLETCurve = depthLETCurve / max_LET;

        % Step 6: Find the x value where the dose falls to 80% of the maximum (after the peak)
        dose_threshold = 0.8; % 80% of the maximum value
        % Find the peak
        [~, max_index] = max(relativeDepthDoseCurve);
        % Find the first value after the peak that falls below 80%
        idx_above_threshold = find(relativeDepthDoseCurve(max_index:end) >= dose_threshold) + max_index - 1;
        idx_below_threshold = find(relativeDepthDoseCurve(max_index:end) < dose_threshold, 1, 'first') + max_index - 1;

        % Interpolation between the two points
        if ~isempty(idx_above_threshold) && ~isempty(idx_below_threshold) && idx_below_threshold > idx_above_threshold(end)
            x1 = idx_above_threshold(end); % Last index where dose >= 80%
            x2 = idx_below_threshold; % First index where dose < 80%
            D1 = relativeDepthDoseCurve(x1); % Dose at x1
            D2 = relativeDepthDoseCurve(x2); % Dose at x2

            % Linear interpolation for the exact x value where the dose falls to 80%
            x_80_percent = x1 + (dose_threshold - D1) / (D2 - D1) * (x2 - x1);
        else
            x_80_percent = NaN; % If no suitable point is found
        end
        
        % Extract the energy of the current bixel from the current ray
        energy = stf.ray(ray_id).energy(bixel_idx);
        
        % Store the x value (at 80% dose), the corrected y value (Ray + Isocenter), and the energy
        x_y_energy_values_at_80_percent = 
        [x_y_energy_values_at_80_percent; 
        x_80_percent*dij.ctGrid.resolution.x, ray_y_absolute, energy];
        
        % Increment the bixel counter
        current_bixel_id = current_bixel_id + 1;
    end
end

% Define the filename
filename = 'x_y_energy_values_at_80_percent.csv';

% Save the matrix to a CSV file
writematrix(x_y_energy_values_at_80_percent, filename);
\end{python}
\subsection{Hounsfield look-up table}
\label{sec:hlutmatradappendix}
\begin{python}
def matRad_conversion(HU):
    return np.interp(HU, [-1024, 200, 449, 2000, 2048, 3071],
            [0.00324, 1.2, 1.20001, 2.49066, 2.5306, 2.53061])
\end{python}


\section{Optimization}
\label{sec:optimizationappendix}
\subsection{Computational gradient descent implementation}
\label{sec:gradientdescentoptimization}
\begin{python}
    % Load data and initialize variables as before
data = readmatrix('x_y_energy_values_at_80_percent.csv');
y_positions = data(:, 2);  % Column 2 represents y-position
energy_values = data(:, 3);  % Column 3 represents energy values
new_x_positions = zeros(size(y_positions));
new_y_positions = zeros(size(y_positions));  % To store updated y-positions if adjusted
new_energy_values = zeros(size(energy_values));  % To store updated energy values

% Initialize arrays to store initial positions and adjusted starting positions
adjusted_starting_x = zeros(size(y_positions));
adjusted_starting_y = zeros(size(y_positions));
adjusted_starting_energy = zeros(size(energy_values));

% Initialize arrays to store the initial positions under the magnetic field (for verification)
initial_x_positions_B = zeros(size(y_positions));
initial_y_positions_B = zeros(size(y_positions));
initial_energy_values_B = zeros(size(energy_values));

magnetic_field = [0; 0; 1.5]; % Magnetic field in Tesla
SPR = readmatrix('SPR_values.csv');

% Parameters for gradient descent
learning_rate_E = 0.0005; % Step size for energy
learning_rate_y = 0.0005; % Step size for position (if y is adjusted)
tolerance = 1e-4;         % Convergence tolerance
max_steps = 200;          % Maximum iterations

% Error threshold definitions
small_error_threshold = 0.5;  % Reduce small difference threshold for finer control
big_error_threshold = 1.5;    % Reduce big difference threshold for finer control

% Loop over each row in the CSV data
for i = 1:length(y_positions)
    % Current y and energy
    y_position = y_positions(i) / 10.9375;  % Divide y-position by pixel factor
    E0_MeV = energy_values(i);  % Energy value

    % Target without magnetic field
    Target = analyzeProtonTrajectory(y_position, E0_MeV, [0; 0; 0], SPR);

    % Initial position with magnetic field (Target_start)
    Target_start = analyzeProtonTrajectory(y_position, E0_MeV, magnetic_field, SPR);

    % Save the initial positions under magnetic field for verification (before gradient descent)
    initial_x_positions_B(i) = Target_start(1) * 10;  % Scale back to original x-position
    initial_y_positions_B(i) = Target_start(2) * 10.9375;  % Scale back to original y-position
    initial_energy_values_B(i) = E0_MeV;  % Initial energy before gradient descent

    % Gradient Descent Loop
    Ei = E0_MeV;  % This energy will be adjusted
    yi = y_position;  % This y-position will be adjusted

    % Initialize variables to track the starting positions during the gradient descent
    initial_yi = yi;  % Keep track of the starting y_position
    initial_Ei = Ei;  % Keep track of the starting energy

    for step_i = 1:max_steps
        % Compute the difference between the current position and target
        position_this_step = analyzeProtonTrajectory(yi, Ei, magnetic_field, SPR);
        error_x = position_this_step(1) - Target(1);
        error_y = position_this_step(2) - Target(2);

        % Check for convergence
        if abs(error_x) < tolerance && abs(error_y) < tolerance
            break;
        end

        % Adjust learning rates based on error size
        if abs(error_x) >= big_error_threshold
            % Larger errors require larger steps, but within smaller thresholds now
            dE = -error_x * E0_MeV * learning_rate_E;
            dy = error_x * y_position * learning_rate_y;
        elseif abs(error_x) <= small_error_threshold
            % For small errors, fine-tune the adjustments
            dE = -error_x * E0_MeV * learning_rate_E * 0.5;
            dy = error_x * y_position * learning_rate_y * 0.5;
        end

        % Update energy and position using gradient descent
        Ei = Ei + dE; % Update energy
        yi = yi + dy; % Update y-position
    end

    % Save the final x-position, updated y-position, and energy after gradient descent
    new_x_positions(i) = position_this_step(1) * 10;  % Convert x to original scale
    new_y_positions(i) = yi * 10.9375;  % Convert back to original scale for saving
    new_energy_values(i) = Ei;  % Store the updated energy

    % Calculate the difference in y-position, x-position, and energy between Target_start and final values
    adjusted_starting_x(i) = Target_start(1) - position_this_step(1);  % Difference in x
    adjusted_starting_y(i) = Target_start(2) - yi;  % Difference in y-position
    adjusted_starting_energy(i) = E0_MeV - Ei;  % Difference in energy
end

% Combine new x-positions, updated y-positions, and updated energy values into a matrix
new_data = [new_x_positions, new_y_positions, new_energy_values];
writematrix(new_data, 'x_y_energy_values_at_80_percent_gradient.csv');

% Save the initial starting positions before gradient descent under the magnetic field (Target_start) in a CSV file
initial_B_data = [initial_x_positions_B, initial_y_positions_B, initial_energy_values_B];
writematrix(initial_B_data, 'x_y_energy_values_at_80_percent_gradient_B.csv');

% Save the adjusted starting positions (differences in x, y, and energy to reach the target) in a CSV file
adjusted_starting_data = [adjusted_starting_x * 10, adjusted_starting_y * 10.9375, adjusted_starting_energy];
writematrix(adjusted_starting_data, 'x_y_new_starting.csv');


%%%%%%%%%%%%%%%%%% Function to Analyze Proton Trajectory %%%%%%%%%%%%%%%%%%

function final_position = analyzeProtonTrajectory(y, E, B, SPR)
    % Constants (needed for velocity calculation for example)
    atomic_mass_unit_MeV_c2 = 931.494; % MeV/c^2
    c = 299792458 * 100; % [mm/s]
    
    grid_step = 0.00109375; % [m] This is the CT grid step /"PixelSpacing" 
    total_distance = 83.91796875; % [mm] CT rows * CT columns * CT grid step
    
    % Pre-calculations for initial velocity
    gamma0 = 1 + E / atomic_mass_unit_MeV_c2;
    beta0 = sqrt(1 - 1/gamma0^2);
    v0 = beta0 * c;
    
    % Set initial conditions based on input parameters
    initial_position = [0; y; 0]; % Initial y-position
    initial_velocity = [v0; 0; 0]; % Initial velocity
    
    % Create an instance of ProtonSimulation
    protonSim = ProtonSimulation_half(SPR, E, B, grid_step, total_distance, initial_position, initial_velocity);
    
    % Initialize and run the simulation
    protonSim = protonSim.initializeStep();
    protonSim = protonSim.simulate();
    protonSim.saveResults();

    % Get the final position using the number of non-zero steps
    num_non_zero_steps = protonSim.displayStepRange();
    if num_non_zero_steps > 0
        final_position = protonSim.positions(:, num_non_zero_steps); % Extract the last valid position
    else
        final_position = [0; 0; 0]; % Default if no valid positions found
    end
end
\end{python}
\printbibliography
\backmatter

\renewcommand*\listfigurename{List of figures}
\addcontentsline{toc}{chapter}{\listfigurename}
\listoffigures
\newpage
\renewcommand*\listtablename{List of tables}
\addcontentsline{toc}{chapter}{\listtablename}
\listoftables
\newpage
\thispagestyle{plain}

\chapter*{Achknowledgments}
I would like to express my deepest gratitude to Prof. Dr. Armin Lühr for his guidance, insightful discussions, and the opportunity to conduct my Master's thesis within his research group.
~\\

I extend my sincere thanks to Prof. Dr. Kevin Kröninger for kindly agreeing to serve as the second referee for my thesis.
~\\

I am deeply grateful to the entire Lühr research group for their invaluable feedback, support, and collaborative spirit. I would like to extend special thanks to Cihangir, Dennis, Lisa, Paulin, Robin and Roman for their dedicated assistance throughout the course of my work.
~\\

Finally, I would like to thank my family and friends for their unwavering emotional support during challenging times.
\addcontentsline{toc}{chapter}{Acknowledgments}
\includepdf{content/declaration.pdf}


%\cleardoublepage
% From https://www.tu-dortmund.de/studierende/im-studium/pruefungsangelegenheiten/allgemeine-vordrucke/
\includepdf{content/Eidesstaatliche_Versicherung.pdf}

\end{document}
